\chapter{Introduzione alla Programmazione Dinamica}

Dopo aver visto tecniche di design degli algoritmi quali Greedy e Divide et
Impera, è importante introdurre una tecnica più potente ma anche più complessa
da applicare: la Programmazione Dinamica (Dynamic Programming).\\

Prima di analizzarla in modo approfondito, spiegheremo a grandi linee il suo
funzionamento. L'idea di base si fonda sulla tecnica Divide et Impera ed è
essenzialmente l'opposto di una strategia Greedy, in sostanza si esplora
implicitamente tutto lo spazio delle soluzioni e si decompone in una serie di
sotto-problemi, grazie ai quali si costruiscono soluzioni corrette per
sotto-problemi sempre più grandi finché non si raggiunge il problema di
partenza.\\
\- Una tecnica di programmazione dinamica è quella della memoization che è utile
per risolvere una moltitudine di problemi e per applicare la programmazione
dinamica è necessario creare un sotto-set di problemi che soddisfano le seguenti
proprietà:

\begin{enumerate}
    \item Esistono solo un numero polinomiale di sotto-problemi
    \item La soluzione al
          problema originale può essere calcolata facilmente dalla soluzione dei
          sotto-problemi
    \item C'è un ordinamento naturale dei sotto-problemi dal più piccolo
          al più grande, insieme a una ricorsione facilmente calcolabile
\end{enumerate}
