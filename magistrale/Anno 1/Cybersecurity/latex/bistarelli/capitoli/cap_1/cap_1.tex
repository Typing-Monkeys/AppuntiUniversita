\chapter{ Cybersecurity }

\section{Introduzione alla sicurezza informatica}
Il rapporto interno/interagenzia NIST NISTIR 7298 (Glossario di informazioni chiave
Termini di sicurezza, maggio 2013) definisce il termine sicurezza informatica  come segue:

\begin{center}
    \textit{Misure e controlli che garantiscono riservatezza, integrità,
        e disponibilità delle risorse del sistema informativo inclusi hardware, software, firmware,
        e le informazioni che vengono elaborate, archiviate e comunicate.}
\end{center}


Questa definizione introduce tre obiettivi chiave che sono al centro della cybersecurity:

\begin{itemize}
    \item \textbf{Confidentiality} (Riservatezza):  conservazione delle restrizioni autorizzate all'accesso alle informazioni e divulgazione, compresi i mezzi per proteggere la privacy personale e le proprie informazioni. Una perdita di riservatezza è la divulgazione non autorizzata di informazioni. Questo termine copre due concetti correlati:
          \begin{itemize}
              \item  \textbf{Data Confidentiality} : garantisce che le informazioni private o riservate non siano disponibili o divulgate a soggetti non autorizzati.
              \item \textbf{Privacy}: assicura che le persone controllino o influenzino le informazioni
                    ad essi relativi, esse possono essere raccolte e conservate, inoltre si definisce da chi e a chi possono essere divulgate.
          \end{itemize}
    \item \textbf{Integrity} (Integrità): prevenire la modifica o la distruzione impropria delle informazioni, compresa la garanzia del non ripudio e dell'autenticità delle informazioni.
          La perdita d'integrità è la modifica o la distruzione non autorizzata di informazioni. Questo termine copre due concetti correlati:
          \begin{itemize}
              \item \textbf{Data integrity}:  garantisce che le informazioni e i programmi vengano modificati solo in modo determinato e autorizzato.
              \item \textbf{System integrity}: assicura che un sistema svolga la sua funzione prevista in modo inalterato, libero da intenzionali o involontarie manipolazioni non autorizzate del sistema.
          \end{itemize}
    \item \textbf{Availability}(Disponibilità): garantisce un accesso tempestivo e affidabile nell'utilizzo delle informazioni. Una perdita di disponibilità è l'interruzione dell'accesso o dell'uso di informazioni o un sistema informativo.
\end{itemize}

Questi tre concetti formano quella che viene spesso definita la triade della CIA. I tre
concetti incarnano gli obiettivi di sicurezza fondamentali sia per i dati che per le informazioni
e servizi informatici. Ad esempio, lo standard FIPS 199 del NIST (Standards for Security
Categorization of Federal Information and Information Systems , febbraio 2004) elenca la riservatezza, integrità e disponibilità come i tre obiettivi di sicurezza per le informazioni e
per i sistemi informativi.

\begin{figure}[H]
    \centering
    \includegraphics[width=7cm, keepaspectratio]{capitoli/cap_1/imgs/cia.png}
    \caption{Requisiti essenziali in Cybersecurity.}\label{fig:cia}
\end{figure}

Sebbene l'uso della triade della CIA per definire gli obiettivi di sicurezza sia ben consolidato,
alcuni nel campo della sicurezza ritengono che siano necessari concetti aggiuntivi per presentare
un quadro completo. Due dei più comunemente citati sono i seguenti:
\begin{itemize}
    \item \textbf{Authenticity} (Autenticità): la proprietà di essere genuini e di poter essere verificati e di essere trusted. Fiducia nella validità di una trasmissione di un messaggio o di un messaggio originatore. Ciò significa verificare che gli utenti siano chi dicono di essere e che ogni input che arriva al sistema proviene da una fonte attendibile.
    \item \textbf{Accountability} (Responsabilità): è la capacità di un sistema di identificare un singolo utente, di determinarne le azioni e il comportamento all'interno del sistema stesso. La rendicontabilità è un aspetto del controllo di accesso e si basa sulla concezione che gli individui siano responsabili delle loro azioni all'interno del sistema. Questo supporta il non ripudio, deterrenza, isolamento dei guasti, rilevamento e prevenzione delle intrusioni, il recupero post-azione in concomitanza con l'azione legale . Poiché i sistemi veramente sicuri non sono ancora un obiettivo realizzabile, dobbiamo essere in grado di tracciare una violazione della sicurezza al/ai responsabile/i. I sistemi devono tenere traccia delle loro attività per consentire successive analisi forensi per rintracciare violazioni della sicurezza o per aiutare nelle controversie sulle transazioni.
\end{itemize}

Si noti che FIPS 199 include l'autenticità sotto integrità.\\

La sicurezza informatica è allo stesso tempo affascinante e complessa, alcuni dei motivi sono:
\begin{enumerate}
    \item \textit{La sicurezza informatica non è così semplice come potrebbe sembrare a un principiante}. I requisiti sembrano essere semplici,in effetti, la maggior parte dei requisiti principali per i servizi di sicurezza possono essere definiti con etichette autoesplicative formate da una sola parola: riservatezza, autenticazione, non ripudio e integrità. Ma i meccanismi utilizzati per soddisfare tali requisiti possono essere piuttosto complessi, e capirli può portare a un ragionamento piuttosto sottile.
    \item \textit{Nello sviluppo di un particolare meccanismo di sicurezza o algoritmo, bisogna sempre considerare potenziali attacchi a tali funzionalità di sicurezza.} In molti casi gli attacchi di successo sono progettati guardando un problema in un modo completamente differente, dunque sfruttando una debolezza inaspettata del meccanismo.
    \item \textit{A causa del punto 2 , le procedure usate per fornire dei servizi particolari sono spesso controintuitive.} Tipicamente, un meccanismo di sicurezza è complesso e non è ovvio dalle dichiarazioni di una particolare esigenza che tali misure elaborate sono necessarie. Solo quando si prendono in considerazione i vari aspetti della minaccia si elaborano i meccanismi di sicurezza hanno un senso.
    \item \textit{I meccanismi di sicurezza in genere coinvolgono più di un particolare algoritmo o
              protocollo.}Richiedono inoltre che i partecipanti siano in possesso di un'informazione segreta (ad es. una chiave di crittografia), che sollevano domande sulla creazione, distribuzione e protezione di tali informazioni segrete. Potrebbe esserci anche una dipendenza sui protocolli di comunicazione il cui comportamento può complicare il compito di sviluppare il meccanismo di sicurezza. Ad esempio, se il corretto funzionamento del meccanismo di sicurezza richiede la definizione di limiti di tempo per il tempo di transito di un messaggio dal mittente al destinatario, allora qualsiasi protocollo o rete che introduce variabili e/o ritardi imprevedibili può rendere tali termini privi di significato.
    \item \textit{ La sicurezza informatica è essenzialmente una battaglia di ingegni tra un perpetratore che prova a  trovare buchi e il progettista o l'amministratore che tenta di chiuderli.} Il grande vantaggio che l'attaccante ha è che lei o lui ha solo bisogno di trovare una singola vulnerabilità, mentre il progettista deve trovare e eliminare tutte le vulnerabilità per ottenere una sicurezza perfetta.
    \item \textit{La sicurezza è ancora troppo spesso un'"aggiunta" (surplus) per essere incorporata in un sistema dopo che il progetto è completo, piuttosto che essere parte integrante del processo di progettazione.}
    \item \textit{La sicurezza richiede un monitoraggio regolare, anche costante, e questo è difficile nei tempi attuali.}
    \item \textit{C'è una naturale tendenza da parte di utenti e gestori di sistema a percepire pochi vantaggi nell'investimento sulla sicurezza fino a quando non si verifica un problema.}
    \item \textit{ Molti utenti e persino gli amministratori della sicurezza vedono una sicurezza forte come un ostacolo al funzionamento o all'uso efficiente di un sistema informativo o di un'informazione.}
\end{enumerate}

\subsection{Terminologie}

Di seguito andremo ad elencare alcuni termini essenziali definiti in ambito di Cybersecurity.

\subsubsection{Risorsa di sistema (Asset)}

Una applicazione, un sistema di supporto generale, un programma ad alto
impatto, un impianto fisico, un sistema
mission-critical, il personale, apparecchiature o un gruppo di sistemi logicamente
correlati.

\subsubsection{Minaccia}

Qualsiasi circostanza o evento che potrebbe avere un impatto negativo sulle
operazioni organizzative
(inclusi missione, funzioni, immagine o reputazione), risorse organizzative,
individui, altre organizzazioni o la
Nazione stessa attraverso un sistema informativo tramite accesso, distruzione,
divulgazione, modifica non autorizzata
delle informazioni , e/o negazione del servizio.

\subsubsection{Contromisure}

Dispositivo o tecniche che hanno come obiettivo la compromissione dell'efficacia
operativa di attività indesiderate/dannose, o la prevenzione di spionaggio,
sabotaggio, furto, accesso/utilizzo non autorizzato di informazioni
sensibili o di sistemi informativi.

\subsubsection{Rischio}

Una misura del grado in cui un'entità è minacciata da una potenziale circostanza o
evento. Tipicamente è una funzione di stima:

\begin{itemize}
    \item degli impatti negativi che si verificherebbero se la circostanza o
          l'evento si verificassero
    \item della probabilità che si verifichi.
\end{itemize}

\begin{figure}[H]
    \centering
    \includegraphics[width=11cm, keepaspectratio]{capitoli/cap_1/imgs/relazione_concetti_sicurezza.png}
    \caption{Concetti di sicurezza e le loro relazioni.}\label{fig:relazioni_concetti_sec}
\end{figure}

\subsection{Asset di un sistema informatico}

Gli asset di un sistema informatico possono essere suddivisi come di seguito:
\begin{itemize}
    \item \textbf{Hardware}: compresi i sistemi informatici e altri trattamenti di dati, archiviazione dei dati,e dispositivi di comunicazione dati;
    \item \textbf{Software}: include il sistema operativo, le utilità di sistema e le applicazioni;
    \item \textbf{Data}: inclusi file e database, nonché dati relativi alla sicurezza, ad esempio file di password.
    \item \textbf{Strutture e reti di comunicazione}: rete locale e geografica, collegamenti di comunicazione, bridge, router e così via.
\end{itemize}

\subsection{Vulnerabilità, minacce e attacchi}
Nel contesto della sicurezza, la nostra preoccupazione riguarda le vulnerabilità
delle risorse del sistema. Il National Reseach Council sulla Cybersicurezza del 2002
ha elencato le seguenti categorie generali di
vulnerabilità di un sistema informatico o di una risorsa di rete:

\begin{itemize}
    \item Il sistema può essere danneggiato (\textbf{corrupted}), quindi fa la cosa sbagliata o dà risposte sbagliate. Ad esempio, i valori dei dati memorizzati possono differire da quello che dovrebbero essere perché sono stati modificati in modo improprio.
    \item Il sistema può avere delle perdite (\textbf{be leaky}). Ad esempio,
          qualcuno che non dovrebbe avere accesso ad alcune o a tutte le informazioni
          disponibili attraverso la rete riesce ad ottenerlo ottengono.
    \item Il sistema può diventare non disponibile (\textbf{unavailable}) o molto
          lento. Cioè, utilizzare il sistema o la rete diventa impossibile o impraticabile.
\end{itemize}

Questi tre tipi generali di vulnerabilità corrispondono ai concetti di integrità,
riservatezza e disponibilità, descritti in precedenza.
Una \textbf{minaccia} rappresenta un potenziale danno alla sicurezza di una risorsa.
Un \textbf{attacco} rappresenta il concretarsi di  una minaccia e, in caso di
successo, comporta una violazione indesiderata della sicurezza.
L'agente che effettua l'attacco viene definito \textbf{attaccante} o
\textbf{agente di minaccia}. Possiamo distinguere due tipi ti attacchi differenti:

\begin{itemize}
    \item \textbf{Attacco attivo}: un tentativo di alterare le risorse del
          sistema o di influenzare il funzionamento.
    \item \textbf{Attacco passivo}: un tentativo di apprendere o
          utilizzare le informazioni di un sistema che però non verrà influenzato
          direttamente.
\end{itemize}

Possiamo anche classificare gli attacchi in base alla loro origine:

\begin{itemize}
    \item \textbf{Attacco interno}: iniziato da un entità interna al perimetro
          di sicurezza (chiamato "insider"). L'insider è autorizzato ad accedere alle
          risorse di sistema ma le utilizza in modi che non sono approvati da coloro che
          gli hanno forntio l'uatorizzazione.
    \item \textbf{Attacco esterno}: iniziato fuori dal perimetro, da un utente
          non autorizzato o illegittimo del sistema (un "outsider").
          Su Internet, potenziali aggressori esterni variano dai dilettanti "burloni"
          a criminali organizzati, terroristi internazionali o governi ostili. %%TODO: jhiad emoji
\end{itemize}

Infine, una \textbf{contromisura} è qualsiasi mezzo adottato per affrontare un
attacco alla sicurezza. Idealmente, una contromisura può essere escogitata per
prevenire un particolare tipo di attacco dall'avere successo.
Quando la prevenzione non è possibile, o in alcuni casi fallisce, l'obiettivo è
rilevare l'attacco e poi riprendersi dai suoi effetti. Una contromisura
può introdurre nuove vulnerabilità, e non è detto che copra da ogni possibile
attacco. Dunque rimarrà un fattore di rischio residuo che andrà minimizzato dai
proprietari dell'asset.

\paragraph{Unauthorized disclosure:} la divulgazione non autorizzata
, Unauthorized disclosure, è una minaccia alla riservatezza.

\begin{itemize}
    \item \textbf{Exposure}: quando un insider rilascia intenzionalmente
          informazioni sensibili a un estraneo.
    \item \textbf{Interception}: un'entità non autorizzata che accede
          direttamente a dati sensibili che viaggiano tra fonti autorizzate.
    \item \textbf{Inference}: un attacco in cui un entità non autorizzata
          accede indirettametne a dati sensibili, con l'analisi del traffico nella rete
          per ottenere informaioni.
    \item \textbf{Intrusion}: un entità non autorizzata ottiene accesso a dati
          snsibili dopo aver superato le ptoretzioni del sistema.
\end{itemize}

\paragraph{Deception: } l'inganno, Deception, è una minaccia per l'integrità
del sistema o per l'integrità dei dati.
I seguenti tipi di attacchi possono portare a queste conseguenze:

\begin{itemize}
    \item \textbf{Masquerade}: un entità non autorizzata ottiene accesso a un sistema
          o effettua azioni malevole spaccaindosi per un'entità autorizzata.
    \item \textbf{Falsification}: dati falsificati ingannano un'entità autorizzata.
    \item \textbf{Repudiation}: un'entità inganna un'altra negando la responsabilità
          di un'azione.
\end{itemize}

\paragraph{Disruption}: L'interruzione, Disruption, è una minaccia alla
disponibilità o all'integrità del sistema.
I seguenti tipi di attacchi possono portare a queste conseguenze:

\begin{itemize}
    \item \textbf{Incapacitation}: impedisce o interrompe le operazioni di un sistema
          disattivandone un componente.
    \item \textbf{Corruption}: alterazione non desiderata delle operaizoni di
          sistema causata dalla modifica di funzioni di sistema o dati di sistema.
    \item \textbf{Obstruction}: un attacco che interrompe la capacità di
          completare servizi di sistema ostacolando le operazioni di sistema.
\end{itemize}

\paragraph{Usurpation: } l'usurpazione, Usurpation, è una minaccia dell'integrità
del sistema che consiste in un'entità non autorizzata che ottiene il controllo
del sistema.
I seguenti tipi di attacchi possono portare a queste conseguenze:

\begin{itemize}
    \item \textbf{Misappropriation}: un'entità assume il controllo logico o fisico
          di una risorsa di sistema senza essere autorizzata.
    \item \textbf{Misuse}: causa un componente di sistema ad effettuare azioni o
          funzioni che sono dannose alla sicurezza del sistema.
\end{itemize}

\subsection{Asset e minacce}

Le risorse di un sistema informatico possono essere classificate come hardware,
software, dati, linee e reti di comunicazione. In questa sottosezione li
descriviamo brevemente mettendoli in relazione con i concetti di integrità,
riservatezza e disponibilità.

\begin{figure}[H]
    \centering
    \includegraphics[width=14cm, keepaspectratio]{capitoli/cap_1/imgs/asset_sec.png}
    \caption{ Scopo della sicurezza informatica.}\label{fig:asset_sec}
\end{figure}

\paragraph{Hardware.}
Una delle principali minacce per l'hardware  del computer è la minaccia alla
disponibilità. L'hardware è il più vulnerabile agli attacchi e il meno
suscettibile ai controlli automatizzati. Le minacce includono danni accidentali
e deliberati alle apparecchiature così come il furto.


\paragraph{Software.}
Una delle principali minacce al software è un attacco alla disponibilità.
Il Software, in particolare quello applicativo, è spesso facile da eliminare,
modificare o danneggiare per renderlo inutilizzabile.
Un problema più difficile da affrontare è la modifica del software che si ha in
un programma il quale funziona ancora ma che si comporta in modo diverso rispetto
a prima, questa è una minaccia per l'integrità/autenticità.
Rientrano in questa categoria i virus informatici e i relativi attacchi.
Un ultimo problema è la protezione contro la pirateria del software.
Sebbene siano disponibili alcune contromisure, in linea di massima il problema
di copie non autorizzate del software non è stata risolta.

\paragraph{Data.}
Un  problema molto più diffuso è la \textbf{sicurezza dei dati}, che
coinvolge file e
altre forme di dati controllati da individui, gruppi e organizzazioni aziendali.
I problemi di sicurezza relativi ai dati sono ampi e comprendono disponibilità,
segretezza e integrità. In caso di disponibilità, la preoccupazione è con la
distruzione di file di dati.
Una preoccupazione evidente per la \textbf{segretezza} è la lettura
non autorizzata di
file di dati o database, focus importante della sicurezza informatica.
Una minaccia meno ovvia alla segretezza comporta l'analisi dei dati e si
manifesta nell'utilizzo delle cosiddette banche dati statistiche, che forniscono
informazioni di sintesi o aggregate. Presumibilmente, l'esistenza delle
informazioni aggregate non minacciano la privacy delle persone coinvolte.
Tuttavia, con la crescita dell'uso delle banche dati statistiche, c'è un rischio
crescente per la divulgazione di informazioni personali. In sostanza, le
caratteristiche di un individuo possono essere identificate attraverso un'analisi
attenta.
Infine, l'\textbf{integrità dei dati} è una delle principali preoccupazioni
nella maggior parte delle installazioni.

\subsection{Attacchi Attivi e Passivi}

Un attacco attivo tenta di alterare le risorse di sistema o di influenzare il
loro funzionamento.
Un attacco passivo tenta di imparare o fare
uso delle informazioni del sistema, ma non influisce sulle risorse di quest'ultimo.

\subsubsection{Attacchi attivi}

Gli attacchi attivi comportano alcune modifiche del flusso di dati o la creazione
di un falso flusso, esso può essere suddiviso in quattro categorie: replay,
masquerade, modifica dei messaggi e denial of service.

\paragraph{Replay.}
Il replay comporta l'acquisizione passiva di un'unità di dati e la sua successiva
ritrasmissione per produrre un effetto non autorizzato.

\paragraph{Masquerade.}
Una masquerade ha luogo quando un'entità finge di essere un'entità diversa.
Un attacco di questo tipo di solito include una delle altre forme di attacco
attivo.

\paragraph{Modifica di un messaggio.} La modifica dei messaggi significa
semplicemente che una parte di un legittimo messaggio è alterato, o che i
messaggi sono ritardati o riordinati, per produrre un effetto non autorizzato.

\paragraph{DOS.}
Un attacco DoS impedisce o inibisce il normale utilizzo o gestione
delle strutture di comunicazione. Questo attacco può avere un obiettivo specifico.
Un'altra forma di DoS è l'interruzione di un'intera rete disabilitandola o
sovraccaricandola di messaggi in modo da degradarne le prestazioni.\\

Gli attacchi attivi presentano le caratteristiche opposte degli attacchi passivi.
Benchè gli attacchi passivi siano difficili da rilevare esistono misure per
evitarne il loro successo. Invece è molto difficile prevenrie gli attacchi
attivi in maniera assoluta perchè sarebbe necessaria una protezione fisica
costante di tutti i centri di comunicazione. Risulta essere più facile rilevarli
e guarire da qualsiasi tipo di disturbo o ritardo che hanno causato.

\subsubsection{Attacchi passivi}

Gli \textbf{attacchi passivi} generalmente riguardano l'intercettazione o il
monitoraggio di trasmissioni di dati. L'obiettivo dell'attaccante è ottenere
le informazioni che vengono trasmesse. Due tipi di attacchi passivi sono il
rilascio del contenuto dei messaggi e dell'analisi del traffico.

\paragraph{Rilascio dei contenuti di un messaggio}
Il \textbf{rilascio dei contenuti} del messaggio è facilmente comprensibile.
Una conversazione telefonica, un messaggio di posta elettronica e un file
trasferito possono contenere dati sensibili o informazioni confidenziali.
Vorremmo impedire a un avversario di comprendere il contenuto di queste
trasmissioni.

\paragraph{Analisi del traffico.}
Un secondo tipo di attacco passivo, \textbf{l'analisi del traffico}, è più
sottile. Supponiamo che noi abbiamo un modo per mascherare il contenuto dei
messaggi o altre informazioni del traffico di dati, in modo che gli oppositori,
anche se hanno catturato il messaggio, non possono estrarre le informazioni dal
messaggio. La tecnica comune per mascherare i contenuti è la crittografia.
Anche se disponiamo di una protezione crittografica, un avversario potrebbe
comunque essere in grado di osservare lo schema di questi messaggi.
L'avversario potrebbe determinare la posizione e l'identità degli host nella
comunicazione e potrebbe osservare la frequenza e la lunghezza dei messaggi
scambiati. Queste informazioni potrebbero essere utili per indovinare la natura
della comunicazione che stava avvenendo.\\

Gli attacchi passivi sono molto difficili da rilevare perché non coinvolgono
alterazione dei dati. In genere, il traffico dei messaggi viene inviato e
ricevuto in un modo apparentemente normale e né il mittente né il destinatario
sono consapevoli che una terza parte ha letto i messaggi o osservato l'andamento
del traffico. Tuttavia, è possibile prevenire il successo di questi attacchi,
di solito mediante crittografia. Pertanto, l'enfasi nell'affrontare gli attacchi
passivi è sulla prevenzione piuttosto che il rilevamento.


\subsection{Requisiti di sicurezza}

Esistono diversi modi per classificare e caratterizzare le contromisure
che possono essere utilizzate per ridurre le vulnerabilità e affrontare le
minacce alle risorse di sistema. In questa sottosezione, vediamo contromisure
in termini di requisiti funzionali, e seguiamo la classificazione definita in
FIPS 200.

\begin{enumerate}
    \item \textbf{Accesso controllato}: limitare l'accesso al sistema informativo
          agli utenti autorizzati, ai processi che agiscono per conto degli utenti
          autorizzati, o ai dispositivi (inclusi altri sistemi informativi) e alle
          tipologie di transazioni e funzioni che gli utenti autorizzati possono
          esercitare.
    \item \textbf{Consapevolezza e Formazione}: garantire che i gestori e gli
          utenti dei sistemi informativi siano consapevoli dei rischi
          per la sicurezza e delle norme da applicare e che il personale sia
          adeguatamente addestrato a svolgere i compiti e le responsabilità assegnate.
    \item \textbf{Audit e responsabilità}: creare, proteggere e conservare i
          record di audit del sistema informativo per consentire il monitoraggio,
          l'analisi, l'indagine e la segnalazione di atti illeciti. Garantire inoltre
          che le azioni dei singoli individui  nel sistema possano essere ricondotte
          in modo univoco a tali utenti in modo che possano essere ritenuti responsabili
          di esse.
    \item \textbf{Certificazione, accreditamento e valutazioni di sicurezza}:
          valutare periodicamente i controlli di sicurezza nei sistemi informativi
          per determinare se i controlli sono efficaci nella loro
          applicazione. Sviluppare e attuare piani d'azione volti a correggere le
          carenze e ridurre o eliminare le vulnerabilità in questi sistemi.
    \item \textbf{Gestione della configurazione}: stabilire e mantenere le
          configurazioni di base e gli inventari dei sistemi
          (inclusi hardware, software, firmware e documentazione) durante i
          rispettivi cicli di vita di sviluppo del sistema.
    \item \textbf{Pianificazione di emergenza}: stabilire, mantenere e
          implementare piani di risposta alle emergenze, operazioni di backup,
          e il ripristino post-disastro per i sistemi in modo da garantire la
          disponibilità di risorse informative critiche e continuità operativa
          in situazioni di emergenza.
    \item \textbf{Identificazione e autenticazione}: identificare gli utenti
          del sistema, i processi che agiscono per conto degli utenti o dei dispositivi
          e autenticare (o verificare) le identità di tali utenti, processi o
          dispositivi, come prerequisito per consentire l'accesso ai sistemi.
    \item \textbf{Risposta all'incidente}: stabilire una capacità operativa
          di gestione degli incidenti per le informazioni organizzative dei sistemi
          che includono un'adeguata preparazione, rilevamento, analisi, contenimento,
          recupero e un controllo delle attività di risposta dell'utente.
    \item \textbf{Manutenzione}: eseguire la manutenzione periodica e
          tempestiva dei sistemi, fornire controlli efficaci sugli strumenti,
          sulle tecniche, sui meccanismi e sul personale utilizzato per condurre
          una manutenzione del sistema informativo.
    \item \textbf{Protezione dei media}: proteggere i media dei sistemi, sia
          cartacei che digitali, limitare l'accesso alle informazioni dei media agli
          utenti autorizzati e sanificare o distruggere i supporti del sistema
          informativo prima dello smaltimento o del rilascio per il riutilizzo.
    \item \textbf{Protezione fisica e ambientale}: limitare l'accesso fisico
          dei soggetti autorizzati ai sistemi informativi, alle apparecchiature e ai
          rispettivi ambienti operativi.
    \item \textbf{Pianificazione}: sviluppare, documentare, aggiornare
          periodicamente e implementare piani di sicurezza per le informazioni
          organizzative dei sistemi che descrivono i controlli di sicurezza esistenti
          o previsti e le regole di comportamento dei soggetti che accedono ai sistemi.
    \item \textbf{Sicurezza del personale}: garantire che le persone che occupano
          posizioni di responsabilità all'interno delle organizzazioni
          (compresi i fornitori di servizi di terze parti) siano affidabili e
          soddisfino i criteri di sicurezza stabiliti per quelle posizioni.
          Garantire che le informazioni organizzative e i sistemi informativi siano
          protetti durante e dopo le azioni del personale quali licenziamenti e
          trasferimenti. Applicare sanzioni formali per il personale che non fa
          rispettare le politiche e le procedure di sicurezza dell'organizzazione.
    \item \textbf{Valutazione del rischio}: valutare periodicamente il rischio
          per le operazioni organizzative (inclusi missioni, funzioni, immagine o
          reputazione), risorse organizzative e individui, risultanti dal funzionamento
          del sistema e il relativo trattamento, archiviazione o trasmissione di
          informazioni organizzative.
    \item \textbf{Acquisizione di sistemi e servizi}: Allocare risorse sufficienti
          per proteggere adeguatamente l'organizzazione dei sistemi. Impiegare
          processi del ciclo di vita dello sviluppo del sistema che incorporano
          considerazioni sulla sicurezza. Imporre limitazioni all'utilizzo e
          all'installazione del software e garantire che i fornitori di terze parti
          adottino adeguate misure di sicurezza per proteggere le informazioni, le
          applicazioni e/o i servizi "esternalizzati" dall'organizzazione.
    \item \textbf{Protezione del sistema e delle comunicazioni}: monitorare,
          controllare e proteggere le comunicazioni organizzative ai confini
          esterni e interni. Impiegare progetti "architettonici", sviluppo software
          tecniche e principi di ingegneria dei sistemi che promuovono un'efficace
          sicurezza delle informazioni all'interno di dei sistemi organizzativi.
    \item \textbf{Integrità del sistema e delle informazioni}: identificare,
          segnalare e correggere le informazioni e le falle del sistema in modo
          tempestivo. Fornire protezione da codice dannoso in posizioni appropriate
          all'interno del sistema organizzativo e monitorare gli avvisi di sicurezza
          del sistema informativo e adottare le azioni appropriate in risposta.
\end{enumerate}