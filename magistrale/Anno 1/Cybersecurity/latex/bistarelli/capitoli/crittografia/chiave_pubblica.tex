\section{Crittografia a chiave Pubblica (asimmetrica)}

La crittografia asimmetrica, conosciuta anche come crittografia a coppia di chiavi,
crittografia a
chiave pubblica/privata o anche solo crittografia a chiave pubblica, è un tipo
di crittografia dove ad
ogni attore coinvolto nella comunicazione è associata una coppia di chiavi:

\begin{itemize}
    \item La chiave pubblica, che deve essere distribuita;
    \item La chiave privata, appunto personale e segreta;
\end{itemize}

Fra due interlocutori, non vi è dunque la necessità di scambiarsi le chiavi.
Se con una delle due
chiavi si cifra (codifica) un messaggio, allora quest'ultimo sarà decifrato solo
con l'altra.

\begin{figure}[H]
    \centering
    \includegraphics[width=\textwidth, keepaspectratio]{capitoli/crittografia/imgs/pubblica.png}
    \caption{Esempio del funzionamento della Crittografia Asimmetrica.}
    \label{fig:crittografia_asimmetrica}
\end{figure}

\newpage

In Figura~\ref{fig:crittografia_asimmetrica} va fatto notare come:
\begin{itemize}
	\item il mittente cifra il messaggio con la sua chiave privata così che tutti possano decifrarlo con la sua chiave pubblica. La cifratura in questo caso ha il solo scopo di garantire l'\textbf{integrità} del messaggio.
	\item il mittente cifra il messaggio con la chiave pubblica del destinatario così che solo il destinatario, con la propria chiave privata, potrà decifrarlo, garantendo la \textbf{confidenzialità} del messaggio.
\end{itemize}

Gli attacchi a sistemi di crittografici sono detti attacchi
di \textbf{Crittoanalisi} e
come obbiettivo hanno quello di provare a dedurre la chiave che gli
permetterebbe di decriptare il testo.
I principali attacchi crittografici si basano sulle seguenti caratteristiche:

\begin{itemize}
    \item \textbf{CypherText only}: è noto solo il testo codificato;
    \item \textbf{Known plaintext}: il messaggio è cifrato ma il testo in chiaro è noto;
    \item \textbf{Chosen plaintext}: l'attaccante può cifrare messaggi in chiaro da lui scelti per ottenere il ciphertext ed eseguirne l'analisi;
    \item \textbf{Brute-force}: forza bruta, tentativi di attacco alla chiave finché non si trova quella giusta;
\end{itemize}

\paragraph{Crittografia Perfetta: } la crittografia si di dice ``Perfetta'' quando
nessun
testo codificato rilascia informazione alcuna né
sulla chiave usata per la codifica, né sul testo in chiaro,
il quale può essere recuperato se e solo se
la chiave è disponibile.
Si tratta di una situazione ideale, in quanto ogni tipo di crittoanalisi sarebbe
reso inutile e la
probabilità di ricavare informazioni supplementari da un testo codificato
sarebbe piuttosto nulla.
Purtroppo però, la crittografia in pratica non è quasi mai perfetta.
