La Policy sulla Sicurezza Informatica è quel documento nel quale sono
contenute tutte le
disposizioni, comportamenti e misure organizzative richieste ai dipendenti
e/o collaboratori
aziendali per contrastare i rischi informatici. Si va a identificare tutte
quelle regole che possono
essere stabilite su un Server così che le workstation collegate vengano
"controllate" nella stessa
maniera e per fare in modo che su di esse siano presenti le stesse
caratteristiche.
L'obiettivo è quello di garantire i tre goal della Sicurezza:
confidenzialità, integrità e disponibilità.
Il compito della policy è quello di stabilire cosa è permesso e cosa non
lo è, cioè distinguere cosa è
reputato sicuro (e quindi autorizzato) da cosa invece può portare ad una
violazione del sistema. Il
sistema si muove da uno stato ad un altro. Ciò che deve fare la policy è
fare in modo che questo
non assuma mai in stati non sicuri. Un sistema si dice “sicuro” quando ciò
non accade mai.
Quando un sistema permette ad un utente o ad un processo di entrare in uno
stato non autorizzato
si dice che si è verificato un “security breach”.
Sia X un set di entità e I un informazione:
\begin{itemize}
    \item I ha confidenzialità con rispetto a X se nessun membro di X può
          ottenere alcuna
          informazione su I;
    \item I ha integrità con rispetto a X se tutti i membri di X si fidano
          di I. Le nozioni di “fiducia” e
          “integrità” sono collegate. Se un sistema ha integrità, dobbiamo
          fidarci del fatto che il suo
          comportamento è corretto. Se I è una risorsa, la sua integrità implica
          che essa funzioni
          come dovrebbe (assurance);
    \item I ha disponibilità con rispetto a X se tutti i membri di X hanno
          accesso ad I.
\end{itemize}


\paragraph{Confidentiality Policy.}

Il suo scopo è garantire che tutto lo staff comprenda i requisiti
dell'organizzazione in relazione alla
divulgazione di dati personali e informazioni riservate.
Riguardo al flusso delle informazioni è possibile possedere diversi
diritti. Si può, infatti:
\begin{itemize}
    \item Trasferire i diritti di accesso;
    \item Trasferire le informazioni senza però trasferire i diritti;
    \item Avere diritto di accesso alle informazioni solo per un certo
          periodo di tempo.
\end{itemize}

Il modello della policy spesso dipende dalla fiducia.

\paragraph{Integrity Policy.}

Definisce come l'informazione può essere modificata o alterata.
Specifica:
\begin{itemize}
    \item Chi può effettivamente compiere queste operazioni;
    \item Sotto quali condizioni i dati possono essere alterati;
    \item Eventuali limiti sulle modifiche dei dati.
\end{itemize}

Un mezzo per ottenere l'integrità è la separazione dei compiti.
Si deve fare in modo che tutte le operazioni compiute (transazioni) mantengano il sistema in uno
stato “consistente”.

\paragraph{Availability.}

Tipi di disponibilità:
\begin{itemize}
    \item Tradizionale: x ha o no l'accesso;
    \item Quality of Service: viene promesso un certo livello di accesso
          che però non è garantito (ad esempio, un livello
          specifico di larghezza di banda).
\end{itemize}

\paragraph{La policy e i suoi meccanismi: }
La policy descrive ciò che è permesso e cosa no, i meccanismi,
invece, controllano come questa viene implementata. Chiaramente gli utenti devono verificarsi
della policy, ma anche dei meccanismi utilizzati.
Le policy prese in considerazione per il controllo sugli accessi sono:
\begin{itemize}
    \item Discretionary Access Control (DAC): il proprietario determina i diritti di accesso.
          Solitamente sono identity-based access control (IBAC), ovvero il proprietario indica anche
          quali altri utenti possono avere l'accesso;
    \item Mandatory Access Control (MAC): policy più restrittive. Stabiliscono a priori quali sono i
          comportamenti da evitare e quelli permessi. Ciò non viene specificato da chi crea la risorse,
          ma anche dalle regole generali che vengono messe in atto, per esempio quelle aziendali;
    \item Originator Controlled Access Control (ORCON): policy dove colui che assegna i diritti è il
          creatore. I possessori dei file non detengono diritti e non possono cederli a loro volta;
    \item Role Based Access Control (RBAC): usate in ambito commerciale per la gestione delle
          risorse a livello amministrativo e aziendale.
          Quando si definisce una policy si devono sempre tenere presenti:
    \item Gli utenti;
    \item I ruoli che essi ricoprono: utente, utente segreto, sistemista, utente negligente..
    \item Le operazioni che possono essere compiute o no: leggere, scrivere, “downgrade”, cambio
          password...
    \item Le modalità con cui vietare o consentire determinate operazioni: obbligo, permesso, divieto,
          discrezionalità..
\end{itemize}