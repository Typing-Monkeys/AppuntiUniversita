\chapter{Heap Overflows}

\section{Alcuni problemi dello heap}

Il problema riguardante la heap memory occorre quando essa non viene adeguatamente
liberata dopo che non è più necessaria. Le memory leaks possono essere problematiche
in processi a lungo termine
o in attacchi di esaurimento delle risorse. La memoria può
essere \textbf{esausta} quando un malintenzionato identifica delle azioni esterne
che possono allocare memoria ma non liberarla. Di conseguenza le allocazioni successive
falliscono e l'applicazione è incapace di processare delle richieste valide dello user
senza \textbf{crashare}. Inoltre è possibile accedere alla memoria liberata a meno
che tutti puntatori che puntano a quella memoria sono settati a NULL o sovrascritti.
Perciò quando si libera, bisogna impostare anche il puntatore alla memoria liberata
a NULL.

\paragraph{Dereferencing Null or Invalid Pointers.} Se l'operando non punta a un
oggetto o una funzione, il comportamento dell'operatore unario * non è definito.

\paragraph{Double free.} Consiste nel liberare lo stesso blocco di memoria più di una volta.

\newpage

\section{Heap overflow}

\subsection{Dlmalloc}

Nella dlmalloc, i blocchi di memoria (chunks) sono sia allocati a un processo che liberi.
I primi 4 byte dei chunk allocati e liberi contengono la dimensione del precedente
blocco adiacente, se è libero, ovvero gli ultimi 4 byte di dati utente del
precedente pezzo, se è allocato.

\begin{figure}[H]
    \centering
    \includegraphics[width=13cm, keepaspectratio]{capitoli/secure_coding//img/cap_4/chuncks.png}
    \caption{Chunk libero e allocato.}\label{fig:chunk_lib_alloc}
\end{figure}

\subsubsection{chunk liberi}

In dlmalloc, i blocchi liberi sono disposti in linked list
\footnote{In informatica, una lista concatenata (o linked list) è una struttura dati
    dinamica, tra quelle fondamentali usate nella programmazione.
    Consiste di una sequenza di nodi, ognuno contenente campi di dati arbitrari ed
    uno o due riferimenti ("link") che puntano al nodo successivo e/o
    precedente.}
circolari a doppio collegamento, detti anche \textbf{bin}.
Ogni linked list a doppio collegamento ha un'intestazione che contiene un puntatore
in avanti e indietro rispettivamente al primo e all'ultimo blocco
nella lista. Sia il puntatore in avanti dell'ultimo chunk che quello all'indietro nel
primo chunk della lista punta all'elemento testa.
Quando la lista è vuota, i puntatori della testa fanno riferimento alla testa stessa.

\subsubsection{Bin}

Ogni bin ha una \textit{head}(testa) che contiene il puntatore in avanti e indietro
che puntano rispettivamente al primo e all'ultimo blocco nella lista. Sia il chunk
allocato che quello libero fanno uso di un bit \verb|PREV_INUSE|
(rappresentato da P in Figura \ref{fig:chunk_lib_alloc}) che indica se il chunk precedente è allocato o
meno.

\begin{figure}[H]
    \centering
    \includegraphics[width=12cm, keepaspectratio]{capitoli/secure_coding/img/cap_4/bin.png}
    \caption{Esempio di un bin.}\label{fig:bin}
\end{figure}


\subsubsection{UNLINK}

\verb|unlink()| è una macro usata per rimuovere un chunk dalla sua lista doppiamente
linkata. Essa è usata quando la memoria è consolidata e quando un chunk è tolto della
lista libera perché è stato allocato da un utente.

\begin{verbatim}
    #define unlink(P, BK, FD) {
        FD = P -> fd;
        BK = P -> bk;
        FD -> bk = BK;
        BK -> fd = FD;
    }
\end{verbatim}


\paragraph{Funzionamento macro.}
Dalla Figura \ref{fig:ulink} si può capire bene il funzionamento della macro,
essa prende in input tre puntatori

\begin{itemize}
    \item \textbf{P} puntatore al blocco da rimuovere;
    \item \textbf{BK} puntatore al blocco precedente;
    \item \textbf{FD} puntatore al blocco successivo.
\end{itemize}

\begin{figure}[H]
    \centering
    \includegraphics[width=12cm, keepaspectratio]{capitoli/secure_coding/img/cap_4/ulink.png}
    \caption{Funzionamento macro unlink.}\label{fig:ulink}
\end{figure}

In Figura \ref{fig:ulink} possiamo vedere un esempio del funzionamento della macro. Come accennato in precedenza il puntatore P si riferisce al chunk da togliere, esso contiene due puntatori uno che punta al blocco precedente e uno a quello successivo. Nel primo step della \verb|unlink()| si assegna FD in modo da farlo puntare al chunk successivo nella lista rispetto a quello indicato da P. Facciamo la stessa cosa nel secondo step solo che assegniamo a BK il puntatore al chunk precedente. Nel terzo step, il puntatore in avanti (FD)  sostituisce il puntatore all'indietro del blocco successivo nella lista con il puntatore al blocco che precede quello che è stato scollegato. Nell'ultimo step il puntatore all'indietro (BK) sostituisce il puntatore in avanti del precedente chunk nella lista con il puntatore al blocco successivo.

\subsection{Tecnica unlink}

La tecnica unlink è stata introdotta la prima volta da Solar Designer e usata con
successo contro alcune versioni dei browser di Netscape, traceroute e slocate che
utilizzavano dlmalloc. Questa tecnica è usata per fare un buffer overflow in modo da
manipolare i tag di confine su un chunk di memoria per ingannare la macro \verb|unlink()|
facendole scrivere 4 byte di dati in una zona arbitraria.

\begin{figure}[H]
    \centering
    \includegraphics[width=10cm, keepaspectratio]{capitoli/secure_coding/img/cap_4/unlink_buff_over.png}
    \caption{Esempio di codice vulnerabile a tecnica unlink.}\label{fig:ulink_buff_over}
\end{figure}

Il programma vulnerabile alloca 3 chunk di memoria (riga 5-7). Il programma accetta
una singola stringa come argomento che è copiata all'interno della malloc \textit{first} (linea 8).
Questa operazione strcpy() illimitata è soggetta a un buffer overflow.
Il tag di confine può essere sovrascritto da un argomento stringa che supera la
lunghezza di first perché il tag di confine per il secondo si trova direttamente dopo
il primo buffer. Il problema di questo programma accade alla seconda free (linea 10).
Vediamo come è strutturato l'heap prima di fare la seconda free.

\begin{figure}[H]
    \centering
    \includegraphics[width=10cm, keepaspectratio]{capitoli/secure_coding/img/cap_4/heap_prima_free.png}
    \caption{Contenuto dell'heap alla prima chiamata di free().}\label{fig:heap_prima_free}
\end{figure}

\vspace{-1em}

Se il secondo blocco non è allocato, l'operazione \verb|free()| prova a consolidarlo
con il primo blocco. Per determinare se il secondo chunk è allocato o meno bisogna
guardare il \verb|PREV_INUSE| bit del terzo blocco. La locazione di ogni blocco è
determinata aggiungendo la grandezza del blocco all'indirizzo iniziale.
Durante le operazioni normali, il bit P del terzo chunk è settato perché il secondo
chunk è allocato come si vede in Figura \ref{fig:heap_prima_free}.
Poiché il buffer vulnerabile è allocato nell'heap e non nello stack, l'attaccante non
può solamente sovrascrivere l'indirizzo di ritorno per sfruttare la vulnerabilità ed
eseguire codice malevolo. L'attaccante può sovrascrivere i boundary tag associati
con il secondo chunk della memoria, perché questo tag di confine è collocato
immediatamente dopo la fine del primo blocco. La grandezza del primo chunk (672 byte)
è il risultato della grandezza richiesta di 666 byte, più 4 byte per la grandezza,
arrotondato al multiplo più vicino a 8 poiché tutti i chunk devono essere divisibili
per 8 byte.

\begin{figure}[H]
    \centering
    \includegraphics[width=13cm, keepaspectratio]{capitoli/secure_coding/img/cap_4/funzionamento_unlink.png}
    \caption{Funzionamento tecnica unlink.}\label{fig:funzionamento_unlink}
\end{figure}

Come si vede in Figura \ref{fig:funzionamento_unlink} un argomento malevolo può essere
usato per sovrascrivere i tag del secondo chunk. Questo argomento sovrascrive il
campo della dimensione del precedente blocco, grandezza del chunk, e i puntatori in
avanti e indietro del secondo chunk, alterando così il comportamento della free().
In particolare il campo per la dimensione è modificato inserendo come valore -4 byte,
in questo modo quando la free() prova a determinare la locazione del terzo chunk
aggiungendo la grandezza appena modificata all'indirizzo iniziale del secondo chunk
invece di aggiungere sottrae 4 byte. Così la dlmalloc pensa che l'inizio del successivo
chunk è 4 byte prima dell'inizio del secondo chunk. L'argomento malevolo garantisce
che la collocazione dove la dlmalloc trova il bit P è libera, ingannando la dlmalloc
facendole credere che il secondo chunk non è allocato così l'operazione di free()
invoca la unlink macro per consolidare i due blocchi liberi consecutivi. Come si vede
in Figura \ref{fig:funzionamento_unlink} viene inserito al posto della grandezza
effettiva del chunk un numero pari finto, deve essere pari poiché l'ultimo bit deve
essere 0.  In fd inseriamo \verb|fp-12| che è l'indirizzo dove voglio compiere l'attacco.
In bk c'è il dato che vogliamo scrivere all'indirizzo fd.
Per come è scritta la unlink, è lei stessa che fa questa cosa quando viene chiamata
la seconda free.
Noi scriviamo solo il payload, poi il lavoro lo fanno free e di conseguenza unlink.
L'obiettivo è scrivere l'indirizzo addr in fp.

