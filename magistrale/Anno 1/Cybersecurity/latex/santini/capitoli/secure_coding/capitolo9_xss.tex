\chapter{XSS (Cross-Site Scripting)}

Gli attacchi Cross-Site Scripting (XSS) sono un tipo di injection, in cui gli
script maligni
vengono iniettati in siti web in realtà benigni e affidabili.
Si verificano quando un aggressore
utilizza un'applicazione web per inviare codice malevolo, generalmente sotto
forma di script
(javascript) lato browser, ad un altro utente finale.
I flaws che permettono a questi attacchi di avere successo sono abbastanza diffusi
e si
verificano ovunque un'applicazione web utilizza l'input di un utente all'interno
dell'output che
genera, senza effettivamente validarlo o codificarlo.
Gli attacchi Cross-Site Scripting si verificano quando:

\begin{itemize}
    \item I dati entrano in un'applicazione web attraverso una fonte non
          attendibile, il più delle
          volte tramite una richiesta web.
    \item I dati sono inclusi in contenuti dinamici che vengono inviati ad un
          utente web senza
          essere convalidati per contenuti dannosi.
\end{itemize}

Il contenuto dannoso inviato al browser web spesso assume la forma di un segmento
di JavaScript, ma può anche includere HTML, Flash o qualsiasi altro tipo di codice
che il browser può eseguire.
La varietà di attacchi basati su XSS è molto vasta, ma comunemente includono:

\begin{itemize}
    \item trasmettere all'aggressore dati privati, come cookies o altre
          informazioni di sessione (password),
    \item Dirigere la vittima a contenuti web controllati dall'aggressore,
    \item Effettuare altre operazioni dannose sulla macchina dell'utente,
          simulando il sito vulnerabile.
\end{itemize}

Gli attacchi XSS possono essere generalmente classificati in tre categorie:

\begin{itemize}
    \item \textbf{Stored}
    \item \textbf{Reflected}
    \item \textbf{DOM Based}
\end{itemize}

\section{Attacchi}

\subsection{Stored}

\subsection{Reflected}

\subsection{DOM Based}

\subsection{Un'altra Classificazione}

\section{Mitigation}