\chapter{SAML}

\begin{figure}[H]
    \centering
    \includegraphics[width=10cm, keepaspectratio]{capitoli/id_managing/imgs/samuel.jpg}
\end{figure}

Il \textit{Security Assertion Markup Language} (\textbf{SAML}) 2.0 fornisce due
funzioni molto importanti: \textit{Cross-domain Single Sign-On} (\textbf{SSO}) e
l'\textit{Identity Federation}. Questo è largamente utilizzato in ambito aziendale
in quanto permette di avere applicazioni che delegano l'autenticazione degli impiegati,
clienti e partner ad un identity provider centralizzato nell'azienda.\\

Il caso d'uso più comune per SAML 2.0 è la Cross-domain single sign-on (SSO) in cui
un utente ha la necessità di accedere a molteplici applicazioni che risiedono in
domini differenti (e.g.: application1.com, application2.com, ecc.).
Senza SSO l'utente avrebbe dovuto creare un account per ognuna di queste applicazioni
e loggarsi su ognuna individualmente, il che si traduce in molte credenziali che
l'utente deve ricordarsi e tenere al sicuro. Per un'azienda questo potrebbe essere
un problema molto importante in quanto dovrebbe trovarsi a gestire un grandissimo
numero di account. SAML permette alle applicazioni di delegare la fase dell'autenticazione
dell'utente ad un'entità remota chiamata Identity Provider.
Questa identità autentica l'utente e ritorna all'applicazione informazioni riguardanti
l'utente autenticato e la fase di autenticazione. Se l'utente accede ad una seconda
applicazione che delega l'autenticazione allo stesso Identity Provider,
non gli verrà richiesto di effettuare nuovamente il login ma potrà utilizzare
immediatamente il servizio.
SAML offre anche un altro meccanismo, chiamato Federated Identity, che permette
alle applicazioni e all'identity provider di utilizzare un unico identificatore
condiviso per un utente in modo da scambiare informazioni riguardo questo.

\section{Terminologia}

SAML definisce i seguenti termini:

\begin{itemize}
    \item \textbf{Subject}: un entità le cui informazioni verranno scambiate.
          Generalmente si riferisce a una persona che deve autenticarsi,
          ma può anche essere
          del software. In generale è un'entità in grado di autenticarsi.
    \item \textbf{SAML Assertion}: un messaggio sotto forma XML che contiene
          contiene informazioni sulla sicurezza riguardo un subject.
    \item \textbf{SAML Profile}: un'insieme di regole di come usare i messaggi SAML
          per un business use case (come cross-domain single sign-on).
    \item \textbf{Identity Provider}: un ruolo definito per il profilo SAML di SSO.
          È un server che invia SAML Assertion riguardo un authenticated subject, nel
          contesto di SSO.
    \item \textbf{Service Provider}: un altro ruolo definito per il profilo SAML di
          SSO. Un service provider delega la fase di autenticazione ad un Identity
          Provider e si affida alle informazioni relative ad un authenticated
          subject ricevute dall'identity provider.
    \item \textbf{Trust Relationship}: un accordo tra un SAML Service Provider e un
          SAML Identity Provider dove il Service Provider si fida delle informazioni
          ricevute dall'Identity Provider.
    \item \textbf{SAML Protocol Binding}: una descrizione di come gli elementi
          dei messaggi SAML sono mappati in protocolli di comunicazione standard,
          come HTTP, per trasmetter informazioni tra il Service Provider e l'Identity
          Provider. In pratica, le richieste e risposte SAML sono inviate utilizzando
          il protocollo HTTPS tramite HTTP-Redirect o HTTP-POST, usando i relativi
          bindings, HTTP-Redirect Binding e HTTP-POST binding.
\end{itemize}

\section{Come Funziona ?}