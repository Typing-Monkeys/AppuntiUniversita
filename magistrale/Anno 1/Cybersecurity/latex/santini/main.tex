\documentclass[a4paper,12 pt]{report}
\usepackage[italian]{babel}
\usepackage[T1]{fontenc}
\usepackage[utf8]{inputenc}
\usepackage{lmodern}
\usepackage{listings}
\usepackage{graphicx}
\usepackage{float}
\usepackage{subcaption}
\usepackage{hyperref}
\usepackage{wrapfig}
\usepackage{fancyhdr}

% forza le footnote a stare il più in basso possibile
\usepackage[bottom]{footmisc}


%% STILE LISTINGS

\usepackage{xcolor}

\definecolor{codegreen}{rgb}{0,0.6,0}
\definecolor{codegray}{rgb}{0.5,0.5,0.5}
\definecolor{codepurple}{rgb}{0.58,0,0.82}
\definecolor{backcolour}{rgb}{0.95,0.95,0.92}

\lstdefinestyle{mystyle}{
    backgroundcolor=\color{backcolour},   
    commentstyle=\color{codegreen},
    keywordstyle=\color{magenta},
    numberstyle=\tiny\color{codegray},
    stringstyle=\color{codepurple},
    basicstyle=\ttfamily\footnotesize,
    breakatwhitespace=false,         
    breaklines=true,                 
    captionpos=b,                    
    keepspaces=true,                 
    numbers=left,                    
    numbersep=5pt,                  
    showspaces=false,                
    showstringspaces=false,
    showtabs=false,                  
    tabsize=2
}

\lstset{style=mystyle}

%% SOLIDITY Settings

% Copyright 2017 Sergei Tikhomirov, MIT License
% https://github.com/s-tikhomirov/solidity-latex-highlighting/

%\usepackage{listings, xcolor}

\definecolor{verylightgray}{rgb}{.97,.97,.97}

\lstdefinelanguage{Solidity}{
	keywords=[1]{anonymous, assembly, assert, balance, break, call, callcode, case, catch, class, constant, continue, constructor, contract, debugger, default, delegatecall, delete, do, else, emit, event, experimental, export, external, false, finally, for, function, gas, if, implements, import, in, indexed, instanceof, interface, internal, is, length, library, log0, log1, log2, log3, log4, memory, modifier, new, payable, pragma, private, protected, public, pure, push, require, return, returns, revert, selfdestruct, send, solidity, storage, struct, suicide, super, switch, then, this, throw, transfer, true, try, typeof, using, value, view, while, with, addmod, ecrecover, keccak256, mulmod, ripemd160, sha256, sha3}, % generic keywords including crypto operations
	keywordstyle=[1]\color{blue}\bfseries,
	keywords=[2]{address, bool, byte, bytes, bytes1, bytes2, bytes3, bytes4, bytes5, bytes6, bytes7, bytes8, bytes9, bytes10, bytes11, bytes12, bytes13, bytes14, bytes15, bytes16, bytes17, bytes18, bytes19, bytes20, bytes21, bytes22, bytes23, bytes24, bytes25, bytes26, bytes27, bytes28, bytes29, bytes30, bytes31, bytes32, enum, int, int8, int16, int24, int32, int40, int48, int56, int64, int72, int80, int88, int96, int104, int112, int120, int128, int136, int144, int152, int160, int168, int176, int184, int192, int200, int208, int216, int224, int232, int240, int248, int256, mapping, string, uint, uint8, uint16, uint24, uint32, uint40, uint48, uint56, uint64, uint72, uint80, uint88, uint96, uint104, uint112, uint120, uint128, uint136, uint144, uint152, uint160, uint168, uint176, uint184, uint192, uint200, uint208, uint216, uint224, uint232, uint240, uint248, uint256, var, void, ether, finney, szabo, wei, days, hours, minutes, seconds, weeks, years},	% types; money and time units
	keywordstyle=[2]\color{teal}\bfseries,
	keywords=[3]{block, blockhash, coinbase, difficulty, gaslimit, number, timestamp, msg, data, gas, sender, sig, value, now, tx, gasprice, origin},	% environment variables
	keywordstyle=[3]\color{violet}\bfseries,
	identifierstyle=\color{black},
	sensitive=false,
	comment=[l]{//},
	morecomment=[s]{/*}{*/},
	commentstyle=\color{gray}\ttfamily,
	stringstyle=\color{red}\ttfamily,
	morestring=[b]',
	morestring=[b]"
}

%\lstset{
%	language=Solidity,
%	backgroundcolor=\color{verylightgray},
%	extendedchars=true,
%	basicstyle=\footnotesize\ttfamily,
%	showstringspaces=false,
%	showspaces=false,
%	numbers=left,
%	numberstyle=\footnotesize,
%	numbersep=9pt,
%	tabsize=2,
%	breaklines=true,
%	showtabs=false,
%	captionpos=b
%}

%% -----


% Resetta la numerazione dei chapter quando
% una nuova part viene creata
\makeatletter
\@addtoreset{chapter}{part} 
\makeatother

% Rimuove l'indentazione quando si crea un nuovo paragrafo
\setlength{\parindent}{0pt}

% footer
\pagestyle{fancyplain}
% rimuove la riga nell'header
\fancyhf{} % sets both header and footer to nothing
\renewcommand{\headrulewidth}{0pt}
\fancyfoot[L]{\href{https://github.com/Typing-Monkeys/AppuntiUniversita}{Typing Monkeys}}
\fancyfoot[C]{\emoji{gorilla}}
\fancyfoot[R]{\thepage}

% configurazione emoji
\usepackage{fontspec}
\usepackage{emoji}
% \setemojifont{NotoColorEmoji.ttf}[Path=/usr/share/fonts/truetype/noto/]
\setemojifont{NotoColorEmoji.ttf}[Path=fonts/]

\begin{document}
\documentclass[a4paper,12 pt]{report}
\usepackage[T1]{fontenc}
\usepackage[utf8]{inputenc}
\usepackage{lmodern}
\usepackage{listings}
\usepackage{graphicx}
\usepackage{float}
\usepackage{subcaption}
\usepackage{hyperref}

% Resetta la numerazione dei chapter quando
% una nuova part viene creata
\makeatletter
\@addtoreset{chapter}{part}
\makeatother

% Rimuove l'indentazione quando si crea un nuovo paragrafo
\setlength{\parindent}{0pt}

\begin{document}
\tableofcontents

%%TODO: definire prima pagina

\documentclass[a4paper,12 pt]{report}
\usepackage[T1]{fontenc}
\usepackage[utf8]{inputenc}
\usepackage{lmodern}
\usepackage{listings}
\usepackage{graphicx}
\usepackage{float}
\usepackage{subcaption}
\usepackage{hyperref}

% Resetta la numerazione dei chapter quando
% una nuova part viene creata
\makeatletter
\@addtoreset{chapter}{part}
\makeatother

% Rimuove l'indentazione quando si crea un nuovo paragrafo
\setlength{\parindent}{0pt}

\begin{document}
\tableofcontents

%%TODO: definire prima pagina

\documentclass[a4paper,12 pt]{report}
\usepackage[T1]{fontenc}
\usepackage[utf8]{inputenc}
\usepackage{lmodern}
\usepackage{listings}
\usepackage{graphicx}
\usepackage{float}
\usepackage{subcaption}
\usepackage{hyperref}

% Resetta la numerazione dei chapter quando
% una nuova part viene creata
\makeatletter
\@addtoreset{chapter}{part}
\makeatother

% Rimuove l'indentazione quando si crea un nuovo paragrafo
\setlength{\parindent}{0pt}

\begin{document}
\tableofcontents

%%TODO: definire prima pagina

\include{capitoli/secure_coding/main.tex}
\include{capitoli/ethereum/main.tex}
\include{capitoli/id_managing/main.tex}
\end{document}
\documentclass[a4paper,12 pt]{report}
\usepackage[T1]{fontenc}
\usepackage[utf8]{inputenc}
\usepackage{lmodern}
\usepackage{listings}
\usepackage{graphicx}
\usepackage{float}
\usepackage{subcaption}
\usepackage{hyperref}

% Resetta la numerazione dei chapter quando
% una nuova part viene creata
\makeatletter
\@addtoreset{chapter}{part}
\makeatother

% Rimuove l'indentazione quando si crea un nuovo paragrafo
\setlength{\parindent}{0pt}

\begin{document}
\tableofcontents

%%TODO: definire prima pagina

\include{capitoli/secure_coding/main.tex}
\include{capitoli/ethereum/main.tex}
\include{capitoli/id_managing/main.tex}
\end{document}
\documentclass[a4paper,12 pt]{report}
\usepackage[T1]{fontenc}
\usepackage[utf8]{inputenc}
\usepackage{lmodern}
\usepackage{listings}
\usepackage{graphicx}
\usepackage{float}
\usepackage{subcaption}
\usepackage{hyperref}

% Resetta la numerazione dei chapter quando
% una nuova part viene creata
\makeatletter
\@addtoreset{chapter}{part}
\makeatother

% Rimuove l'indentazione quando si crea un nuovo paragrafo
\setlength{\parindent}{0pt}

\begin{document}
\tableofcontents

%%TODO: definire prima pagina

\include{capitoli/secure_coding/main.tex}
\include{capitoli/ethereum/main.tex}
\include{capitoli/id_managing/main.tex}
\end{document}
\end{document}
\documentclass[a4paper,12 pt]{report}
\usepackage[T1]{fontenc}
\usepackage[utf8]{inputenc}
\usepackage{lmodern}
\usepackage{listings}
\usepackage{graphicx}
\usepackage{float}
\usepackage{subcaption}
\usepackage{hyperref}

% Resetta la numerazione dei chapter quando
% una nuova part viene creata
\makeatletter
\@addtoreset{chapter}{part}
\makeatother

% Rimuove l'indentazione quando si crea un nuovo paragrafo
\setlength{\parindent}{0pt}

\begin{document}
\tableofcontents

%%TODO: definire prima pagina

\documentclass[a4paper,12 pt]{report}
\usepackage[T1]{fontenc}
\usepackage[utf8]{inputenc}
\usepackage{lmodern}
\usepackage{listings}
\usepackage{graphicx}
\usepackage{float}
\usepackage{subcaption}
\usepackage{hyperref}

% Resetta la numerazione dei chapter quando
% una nuova part viene creata
\makeatletter
\@addtoreset{chapter}{part}
\makeatother

% Rimuove l'indentazione quando si crea un nuovo paragrafo
\setlength{\parindent}{0pt}

\begin{document}
\tableofcontents

%%TODO: definire prima pagina

\include{capitoli/secure_coding/main.tex}
\include{capitoli/ethereum/main.tex}
\include{capitoli/id_managing/main.tex}
\end{document}
\documentclass[a4paper,12 pt]{report}
\usepackage[T1]{fontenc}
\usepackage[utf8]{inputenc}
\usepackage{lmodern}
\usepackage{listings}
\usepackage{graphicx}
\usepackage{float}
\usepackage{subcaption}
\usepackage{hyperref}

% Resetta la numerazione dei chapter quando
% una nuova part viene creata
\makeatletter
\@addtoreset{chapter}{part}
\makeatother

% Rimuove l'indentazione quando si crea un nuovo paragrafo
\setlength{\parindent}{0pt}

\begin{document}
\tableofcontents

%%TODO: definire prima pagina

\include{capitoli/secure_coding/main.tex}
\include{capitoli/ethereum/main.tex}
\include{capitoli/id_managing/main.tex}
\end{document}
\documentclass[a4paper,12 pt]{report}
\usepackage[T1]{fontenc}
\usepackage[utf8]{inputenc}
\usepackage{lmodern}
\usepackage{listings}
\usepackage{graphicx}
\usepackage{float}
\usepackage{subcaption}
\usepackage{hyperref}

% Resetta la numerazione dei chapter quando
% una nuova part viene creata
\makeatletter
\@addtoreset{chapter}{part}
\makeatother

% Rimuove l'indentazione quando si crea un nuovo paragrafo
\setlength{\parindent}{0pt}

\begin{document}
\tableofcontents

%%TODO: definire prima pagina

\include{capitoli/secure_coding/main.tex}
\include{capitoli/ethereum/main.tex}
\include{capitoli/id_managing/main.tex}
\end{document}
\end{document}
\documentclass[a4paper,12 pt]{report}
\usepackage[T1]{fontenc}
\usepackage[utf8]{inputenc}
\usepackage{lmodern}
\usepackage{listings}
\usepackage{graphicx}
\usepackage{float}
\usepackage{subcaption}
\usepackage{hyperref}

% Resetta la numerazione dei chapter quando
% una nuova part viene creata
\makeatletter
\@addtoreset{chapter}{part}
\makeatother

% Rimuove l'indentazione quando si crea un nuovo paragrafo
\setlength{\parindent}{0pt}

\begin{document}
\tableofcontents

%%TODO: definire prima pagina

\documentclass[a4paper,12 pt]{report}
\usepackage[T1]{fontenc}
\usepackage[utf8]{inputenc}
\usepackage{lmodern}
\usepackage{listings}
\usepackage{graphicx}
\usepackage{float}
\usepackage{subcaption}
\usepackage{hyperref}

% Resetta la numerazione dei chapter quando
% una nuova part viene creata
\makeatletter
\@addtoreset{chapter}{part}
\makeatother

% Rimuove l'indentazione quando si crea un nuovo paragrafo
\setlength{\parindent}{0pt}

\begin{document}
\tableofcontents

%%TODO: definire prima pagina

\include{capitoli/secure_coding/main.tex}
\include{capitoli/ethereum/main.tex}
\include{capitoli/id_managing/main.tex}
\end{document}
\documentclass[a4paper,12 pt]{report}
\usepackage[T1]{fontenc}
\usepackage[utf8]{inputenc}
\usepackage{lmodern}
\usepackage{listings}
\usepackage{graphicx}
\usepackage{float}
\usepackage{subcaption}
\usepackage{hyperref}

% Resetta la numerazione dei chapter quando
% una nuova part viene creata
\makeatletter
\@addtoreset{chapter}{part}
\makeatother

% Rimuove l'indentazione quando si crea un nuovo paragrafo
\setlength{\parindent}{0pt}

\begin{document}
\tableofcontents

%%TODO: definire prima pagina

\include{capitoli/secure_coding/main.tex}
\include{capitoli/ethereum/main.tex}
\include{capitoli/id_managing/main.tex}
\end{document}
\documentclass[a4paper,12 pt]{report}
\usepackage[T1]{fontenc}
\usepackage[utf8]{inputenc}
\usepackage{lmodern}
\usepackage{listings}
\usepackage{graphicx}
\usepackage{float}
\usepackage{subcaption}
\usepackage{hyperref}

% Resetta la numerazione dei chapter quando
% una nuova part viene creata
\makeatletter
\@addtoreset{chapter}{part}
\makeatother

% Rimuove l'indentazione quando si crea un nuovo paragrafo
\setlength{\parindent}{0pt}

\begin{document}
\tableofcontents

%%TODO: definire prima pagina

\include{capitoli/secure_coding/main.tex}
\include{capitoli/ethereum/main.tex}
\include{capitoli/id_managing/main.tex}
\end{document}
\end{document}
\end{document}

\tableofcontents

\documentclass[a4paper,12 pt]{report}
\usepackage[T1]{fontenc}
\usepackage[utf8]{inputenc}
\usepackage{lmodern}
\usepackage{listings}
\usepackage{graphicx}
\usepackage{float}
\usepackage{subcaption}
\usepackage{hyperref}

% Resetta la numerazione dei chapter quando
% una nuova part viene creata
\makeatletter
\@addtoreset{chapter}{part}
\makeatother

% Rimuove l'indentazione quando si crea un nuovo paragrafo
\setlength{\parindent}{0pt}

\begin{document}
\tableofcontents

%%TODO: definire prima pagina

\documentclass[a4paper,12 pt]{report}
\usepackage[T1]{fontenc}
\usepackage[utf8]{inputenc}
\usepackage{lmodern}
\usepackage{listings}
\usepackage{graphicx}
\usepackage{float}
\usepackage{subcaption}
\usepackage{hyperref}

% Resetta la numerazione dei chapter quando
% una nuova part viene creata
\makeatletter
\@addtoreset{chapter}{part}
\makeatother

% Rimuove l'indentazione quando si crea un nuovo paragrafo
\setlength{\parindent}{0pt}

\begin{document}
\tableofcontents

%%TODO: definire prima pagina

\documentclass[a4paper,12 pt]{report}
\usepackage[T1]{fontenc}
\usepackage[utf8]{inputenc}
\usepackage{lmodern}
\usepackage{listings}
\usepackage{graphicx}
\usepackage{float}
\usepackage{subcaption}
\usepackage{hyperref}

% Resetta la numerazione dei chapter quando
% una nuova part viene creata
\makeatletter
\@addtoreset{chapter}{part}
\makeatother

% Rimuove l'indentazione quando si crea un nuovo paragrafo
\setlength{\parindent}{0pt}

\begin{document}
\tableofcontents

%%TODO: definire prima pagina

\include{capitoli/secure_coding/main.tex}
\include{capitoli/ethereum/main.tex}
\include{capitoli/id_managing/main.tex}
\end{document}
\documentclass[a4paper,12 pt]{report}
\usepackage[T1]{fontenc}
\usepackage[utf8]{inputenc}
\usepackage{lmodern}
\usepackage{listings}
\usepackage{graphicx}
\usepackage{float}
\usepackage{subcaption}
\usepackage{hyperref}

% Resetta la numerazione dei chapter quando
% una nuova part viene creata
\makeatletter
\@addtoreset{chapter}{part}
\makeatother

% Rimuove l'indentazione quando si crea un nuovo paragrafo
\setlength{\parindent}{0pt}

\begin{document}
\tableofcontents

%%TODO: definire prima pagina

\include{capitoli/secure_coding/main.tex}
\include{capitoli/ethereum/main.tex}
\include{capitoli/id_managing/main.tex}
\end{document}
\documentclass[a4paper,12 pt]{report}
\usepackage[T1]{fontenc}
\usepackage[utf8]{inputenc}
\usepackage{lmodern}
\usepackage{listings}
\usepackage{graphicx}
\usepackage{float}
\usepackage{subcaption}
\usepackage{hyperref}

% Resetta la numerazione dei chapter quando
% una nuova part viene creata
\makeatletter
\@addtoreset{chapter}{part}
\makeatother

% Rimuove l'indentazione quando si crea un nuovo paragrafo
\setlength{\parindent}{0pt}

\begin{document}
\tableofcontents

%%TODO: definire prima pagina

\include{capitoli/secure_coding/main.tex}
\include{capitoli/ethereum/main.tex}
\include{capitoli/id_managing/main.tex}
\end{document}
\end{document}
\documentclass[a4paper,12 pt]{report}
\usepackage[T1]{fontenc}
\usepackage[utf8]{inputenc}
\usepackage{lmodern}
\usepackage{listings}
\usepackage{graphicx}
\usepackage{float}
\usepackage{subcaption}
\usepackage{hyperref}

% Resetta la numerazione dei chapter quando
% una nuova part viene creata
\makeatletter
\@addtoreset{chapter}{part}
\makeatother

% Rimuove l'indentazione quando si crea un nuovo paragrafo
\setlength{\parindent}{0pt}

\begin{document}
\tableofcontents

%%TODO: definire prima pagina

\documentclass[a4paper,12 pt]{report}
\usepackage[T1]{fontenc}
\usepackage[utf8]{inputenc}
\usepackage{lmodern}
\usepackage{listings}
\usepackage{graphicx}
\usepackage{float}
\usepackage{subcaption}
\usepackage{hyperref}

% Resetta la numerazione dei chapter quando
% una nuova part viene creata
\makeatletter
\@addtoreset{chapter}{part}
\makeatother

% Rimuove l'indentazione quando si crea un nuovo paragrafo
\setlength{\parindent}{0pt}

\begin{document}
\tableofcontents

%%TODO: definire prima pagina

\include{capitoli/secure_coding/main.tex}
\include{capitoli/ethereum/main.tex}
\include{capitoli/id_managing/main.tex}
\end{document}
\documentclass[a4paper,12 pt]{report}
\usepackage[T1]{fontenc}
\usepackage[utf8]{inputenc}
\usepackage{lmodern}
\usepackage{listings}
\usepackage{graphicx}
\usepackage{float}
\usepackage{subcaption}
\usepackage{hyperref}

% Resetta la numerazione dei chapter quando
% una nuova part viene creata
\makeatletter
\@addtoreset{chapter}{part}
\makeatother

% Rimuove l'indentazione quando si crea un nuovo paragrafo
\setlength{\parindent}{0pt}

\begin{document}
\tableofcontents

%%TODO: definire prima pagina

\include{capitoli/secure_coding/main.tex}
\include{capitoli/ethereum/main.tex}
\include{capitoli/id_managing/main.tex}
\end{document}
\documentclass[a4paper,12 pt]{report}
\usepackage[T1]{fontenc}
\usepackage[utf8]{inputenc}
\usepackage{lmodern}
\usepackage{listings}
\usepackage{graphicx}
\usepackage{float}
\usepackage{subcaption}
\usepackage{hyperref}

% Resetta la numerazione dei chapter quando
% una nuova part viene creata
\makeatletter
\@addtoreset{chapter}{part}
\makeatother

% Rimuove l'indentazione quando si crea un nuovo paragrafo
\setlength{\parindent}{0pt}

\begin{document}
\tableofcontents

%%TODO: definire prima pagina

\include{capitoli/secure_coding/main.tex}
\include{capitoli/ethereum/main.tex}
\include{capitoli/id_managing/main.tex}
\end{document}
\end{document}
\documentclass[a4paper,12 pt]{report}
\usepackage[T1]{fontenc}
\usepackage[utf8]{inputenc}
\usepackage{lmodern}
\usepackage{listings}
\usepackage{graphicx}
\usepackage{float}
\usepackage{subcaption}
\usepackage{hyperref}

% Resetta la numerazione dei chapter quando
% una nuova part viene creata
\makeatletter
\@addtoreset{chapter}{part}
\makeatother

% Rimuove l'indentazione quando si crea un nuovo paragrafo
\setlength{\parindent}{0pt}

\begin{document}
\tableofcontents

%%TODO: definire prima pagina

\documentclass[a4paper,12 pt]{report}
\usepackage[T1]{fontenc}
\usepackage[utf8]{inputenc}
\usepackage{lmodern}
\usepackage{listings}
\usepackage{graphicx}
\usepackage{float}
\usepackage{subcaption}
\usepackage{hyperref}

% Resetta la numerazione dei chapter quando
% una nuova part viene creata
\makeatletter
\@addtoreset{chapter}{part}
\makeatother

% Rimuove l'indentazione quando si crea un nuovo paragrafo
\setlength{\parindent}{0pt}

\begin{document}
\tableofcontents

%%TODO: definire prima pagina

\include{capitoli/secure_coding/main.tex}
\include{capitoli/ethereum/main.tex}
\include{capitoli/id_managing/main.tex}
\end{document}
\documentclass[a4paper,12 pt]{report}
\usepackage[T1]{fontenc}
\usepackage[utf8]{inputenc}
\usepackage{lmodern}
\usepackage{listings}
\usepackage{graphicx}
\usepackage{float}
\usepackage{subcaption}
\usepackage{hyperref}

% Resetta la numerazione dei chapter quando
% una nuova part viene creata
\makeatletter
\@addtoreset{chapter}{part}
\makeatother

% Rimuove l'indentazione quando si crea un nuovo paragrafo
\setlength{\parindent}{0pt}

\begin{document}
\tableofcontents

%%TODO: definire prima pagina

\include{capitoli/secure_coding/main.tex}
\include{capitoli/ethereum/main.tex}
\include{capitoli/id_managing/main.tex}
\end{document}
\documentclass[a4paper,12 pt]{report}
\usepackage[T1]{fontenc}
\usepackage[utf8]{inputenc}
\usepackage{lmodern}
\usepackage{listings}
\usepackage{graphicx}
\usepackage{float}
\usepackage{subcaption}
\usepackage{hyperref}

% Resetta la numerazione dei chapter quando
% una nuova part viene creata
\makeatletter
\@addtoreset{chapter}{part}
\makeatother

% Rimuove l'indentazione quando si crea un nuovo paragrafo
\setlength{\parindent}{0pt}

\begin{document}
\tableofcontents

%%TODO: definire prima pagina

\include{capitoli/secure_coding/main.tex}
\include{capitoli/ethereum/main.tex}
\include{capitoli/id_managing/main.tex}
\end{document}
\end{document}
\end{document}

\documentclass[a4paper,12 pt]{report}
\usepackage[T1]{fontenc}
\usepackage[utf8]{inputenc}
\usepackage{lmodern}
\usepackage{listings}
\usepackage{graphicx}
\usepackage{float}
\usepackage{subcaption}
\usepackage{hyperref}

% Resetta la numerazione dei chapter quando
% una nuova part viene creata
\makeatletter
\@addtoreset{chapter}{part}
\makeatother

% Rimuove l'indentazione quando si crea un nuovo paragrafo
\setlength{\parindent}{0pt}

\begin{document}
\tableofcontents

%%TODO: definire prima pagina

\documentclass[a4paper,12 pt]{report}
\usepackage[T1]{fontenc}
\usepackage[utf8]{inputenc}
\usepackage{lmodern}
\usepackage{listings}
\usepackage{graphicx}
\usepackage{float}
\usepackage{subcaption}
\usepackage{hyperref}

% Resetta la numerazione dei chapter quando
% una nuova part viene creata
\makeatletter
\@addtoreset{chapter}{part}
\makeatother

% Rimuove l'indentazione quando si crea un nuovo paragrafo
\setlength{\parindent}{0pt}

\begin{document}
\tableofcontents

%%TODO: definire prima pagina

\documentclass[a4paper,12 pt]{report}
\usepackage[T1]{fontenc}
\usepackage[utf8]{inputenc}
\usepackage{lmodern}
\usepackage{listings}
\usepackage{graphicx}
\usepackage{float}
\usepackage{subcaption}
\usepackage{hyperref}

% Resetta la numerazione dei chapter quando
% una nuova part viene creata
\makeatletter
\@addtoreset{chapter}{part}
\makeatother

% Rimuove l'indentazione quando si crea un nuovo paragrafo
\setlength{\parindent}{0pt}

\begin{document}
\tableofcontents

%%TODO: definire prima pagina

\include{capitoli/secure_coding/main.tex}
\include{capitoli/ethereum/main.tex}
\include{capitoli/id_managing/main.tex}
\end{document}
\documentclass[a4paper,12 pt]{report}
\usepackage[T1]{fontenc}
\usepackage[utf8]{inputenc}
\usepackage{lmodern}
\usepackage{listings}
\usepackage{graphicx}
\usepackage{float}
\usepackage{subcaption}
\usepackage{hyperref}

% Resetta la numerazione dei chapter quando
% una nuova part viene creata
\makeatletter
\@addtoreset{chapter}{part}
\makeatother

% Rimuove l'indentazione quando si crea un nuovo paragrafo
\setlength{\parindent}{0pt}

\begin{document}
\tableofcontents

%%TODO: definire prima pagina

\include{capitoli/secure_coding/main.tex}
\include{capitoli/ethereum/main.tex}
\include{capitoli/id_managing/main.tex}
\end{document}
\documentclass[a4paper,12 pt]{report}
\usepackage[T1]{fontenc}
\usepackage[utf8]{inputenc}
\usepackage{lmodern}
\usepackage{listings}
\usepackage{graphicx}
\usepackage{float}
\usepackage{subcaption}
\usepackage{hyperref}

% Resetta la numerazione dei chapter quando
% una nuova part viene creata
\makeatletter
\@addtoreset{chapter}{part}
\makeatother

% Rimuove l'indentazione quando si crea un nuovo paragrafo
\setlength{\parindent}{0pt}

\begin{document}
\tableofcontents

%%TODO: definire prima pagina

\include{capitoli/secure_coding/main.tex}
\include{capitoli/ethereum/main.tex}
\include{capitoli/id_managing/main.tex}
\end{document}
\end{document}
\documentclass[a4paper,12 pt]{report}
\usepackage[T1]{fontenc}
\usepackage[utf8]{inputenc}
\usepackage{lmodern}
\usepackage{listings}
\usepackage{graphicx}
\usepackage{float}
\usepackage{subcaption}
\usepackage{hyperref}

% Resetta la numerazione dei chapter quando
% una nuova part viene creata
\makeatletter
\@addtoreset{chapter}{part}
\makeatother

% Rimuove l'indentazione quando si crea un nuovo paragrafo
\setlength{\parindent}{0pt}

\begin{document}
\tableofcontents

%%TODO: definire prima pagina

\documentclass[a4paper,12 pt]{report}
\usepackage[T1]{fontenc}
\usepackage[utf8]{inputenc}
\usepackage{lmodern}
\usepackage{listings}
\usepackage{graphicx}
\usepackage{float}
\usepackage{subcaption}
\usepackage{hyperref}

% Resetta la numerazione dei chapter quando
% una nuova part viene creata
\makeatletter
\@addtoreset{chapter}{part}
\makeatother

% Rimuove l'indentazione quando si crea un nuovo paragrafo
\setlength{\parindent}{0pt}

\begin{document}
\tableofcontents

%%TODO: definire prima pagina

\include{capitoli/secure_coding/main.tex}
\include{capitoli/ethereum/main.tex}
\include{capitoli/id_managing/main.tex}
\end{document}
\documentclass[a4paper,12 pt]{report}
\usepackage[T1]{fontenc}
\usepackage[utf8]{inputenc}
\usepackage{lmodern}
\usepackage{listings}
\usepackage{graphicx}
\usepackage{float}
\usepackage{subcaption}
\usepackage{hyperref}

% Resetta la numerazione dei chapter quando
% una nuova part viene creata
\makeatletter
\@addtoreset{chapter}{part}
\makeatother

% Rimuove l'indentazione quando si crea un nuovo paragrafo
\setlength{\parindent}{0pt}

\begin{document}
\tableofcontents

%%TODO: definire prima pagina

\include{capitoli/secure_coding/main.tex}
\include{capitoli/ethereum/main.tex}
\include{capitoli/id_managing/main.tex}
\end{document}
\documentclass[a4paper,12 pt]{report}
\usepackage[T1]{fontenc}
\usepackage[utf8]{inputenc}
\usepackage{lmodern}
\usepackage{listings}
\usepackage{graphicx}
\usepackage{float}
\usepackage{subcaption}
\usepackage{hyperref}

% Resetta la numerazione dei chapter quando
% una nuova part viene creata
\makeatletter
\@addtoreset{chapter}{part}
\makeatother

% Rimuove l'indentazione quando si crea un nuovo paragrafo
\setlength{\parindent}{0pt}

\begin{document}
\tableofcontents

%%TODO: definire prima pagina

\include{capitoli/secure_coding/main.tex}
\include{capitoli/ethereum/main.tex}
\include{capitoli/id_managing/main.tex}
\end{document}
\end{document}
\documentclass[a4paper,12 pt]{report}
\usepackage[T1]{fontenc}
\usepackage[utf8]{inputenc}
\usepackage{lmodern}
\usepackage{listings}
\usepackage{graphicx}
\usepackage{float}
\usepackage{subcaption}
\usepackage{hyperref}

% Resetta la numerazione dei chapter quando
% una nuova part viene creata
\makeatletter
\@addtoreset{chapter}{part}
\makeatother

% Rimuove l'indentazione quando si crea un nuovo paragrafo
\setlength{\parindent}{0pt}

\begin{document}
\tableofcontents

%%TODO: definire prima pagina

\documentclass[a4paper,12 pt]{report}
\usepackage[T1]{fontenc}
\usepackage[utf8]{inputenc}
\usepackage{lmodern}
\usepackage{listings}
\usepackage{graphicx}
\usepackage{float}
\usepackage{subcaption}
\usepackage{hyperref}

% Resetta la numerazione dei chapter quando
% una nuova part viene creata
\makeatletter
\@addtoreset{chapter}{part}
\makeatother

% Rimuove l'indentazione quando si crea un nuovo paragrafo
\setlength{\parindent}{0pt}

\begin{document}
\tableofcontents

%%TODO: definire prima pagina

\include{capitoli/secure_coding/main.tex}
\include{capitoli/ethereum/main.tex}
\include{capitoli/id_managing/main.tex}
\end{document}
\documentclass[a4paper,12 pt]{report}
\usepackage[T1]{fontenc}
\usepackage[utf8]{inputenc}
\usepackage{lmodern}
\usepackage{listings}
\usepackage{graphicx}
\usepackage{float}
\usepackage{subcaption}
\usepackage{hyperref}

% Resetta la numerazione dei chapter quando
% una nuova part viene creata
\makeatletter
\@addtoreset{chapter}{part}
\makeatother

% Rimuove l'indentazione quando si crea un nuovo paragrafo
\setlength{\parindent}{0pt}

\begin{document}
\tableofcontents

%%TODO: definire prima pagina

\include{capitoli/secure_coding/main.tex}
\include{capitoli/ethereum/main.tex}
\include{capitoli/id_managing/main.tex}
\end{document}
\documentclass[a4paper,12 pt]{report}
\usepackage[T1]{fontenc}
\usepackage[utf8]{inputenc}
\usepackage{lmodern}
\usepackage{listings}
\usepackage{graphicx}
\usepackage{float}
\usepackage{subcaption}
\usepackage{hyperref}

% Resetta la numerazione dei chapter quando
% una nuova part viene creata
\makeatletter
\@addtoreset{chapter}{part}
\makeatother

% Rimuove l'indentazione quando si crea un nuovo paragrafo
\setlength{\parindent}{0pt}

\begin{document}
\tableofcontents

%%TODO: definire prima pagina

\include{capitoli/secure_coding/main.tex}
\include{capitoli/ethereum/main.tex}
\include{capitoli/id_managing/main.tex}
\end{document}
\end{document}
\end{document}
\documentclass[a4paper,12 pt]{report}
\usepackage[T1]{fontenc}
\usepackage[utf8]{inputenc}
\usepackage{lmodern}
\usepackage{listings}
\usepackage{graphicx}
\usepackage{float}
\usepackage{subcaption}
\usepackage{hyperref}

% Resetta la numerazione dei chapter quando
% una nuova part viene creata
\makeatletter
\@addtoreset{chapter}{part}
\makeatother

% Rimuove l'indentazione quando si crea un nuovo paragrafo
\setlength{\parindent}{0pt}

\begin{document}
\tableofcontents

%%TODO: definire prima pagina

\documentclass[a4paper,12 pt]{report}
\usepackage[T1]{fontenc}
\usepackage[utf8]{inputenc}
\usepackage{lmodern}
\usepackage{listings}
\usepackage{graphicx}
\usepackage{float}
\usepackage{subcaption}
\usepackage{hyperref}

% Resetta la numerazione dei chapter quando
% una nuova part viene creata
\makeatletter
\@addtoreset{chapter}{part}
\makeatother

% Rimuove l'indentazione quando si crea un nuovo paragrafo
\setlength{\parindent}{0pt}

\begin{document}
\tableofcontents

%%TODO: definire prima pagina

\documentclass[a4paper,12 pt]{report}
\usepackage[T1]{fontenc}
\usepackage[utf8]{inputenc}
\usepackage{lmodern}
\usepackage{listings}
\usepackage{graphicx}
\usepackage{float}
\usepackage{subcaption}
\usepackage{hyperref}

% Resetta la numerazione dei chapter quando
% una nuova part viene creata
\makeatletter
\@addtoreset{chapter}{part}
\makeatother

% Rimuove l'indentazione quando si crea un nuovo paragrafo
\setlength{\parindent}{0pt}

\begin{document}
\tableofcontents

%%TODO: definire prima pagina

\include{capitoli/secure_coding/main.tex}
\include{capitoli/ethereum/main.tex}
\include{capitoli/id_managing/main.tex}
\end{document}
\documentclass[a4paper,12 pt]{report}
\usepackage[T1]{fontenc}
\usepackage[utf8]{inputenc}
\usepackage{lmodern}
\usepackage{listings}
\usepackage{graphicx}
\usepackage{float}
\usepackage{subcaption}
\usepackage{hyperref}

% Resetta la numerazione dei chapter quando
% una nuova part viene creata
\makeatletter
\@addtoreset{chapter}{part}
\makeatother

% Rimuove l'indentazione quando si crea un nuovo paragrafo
\setlength{\parindent}{0pt}

\begin{document}
\tableofcontents

%%TODO: definire prima pagina

\include{capitoli/secure_coding/main.tex}
\include{capitoli/ethereum/main.tex}
\include{capitoli/id_managing/main.tex}
\end{document}
\documentclass[a4paper,12 pt]{report}
\usepackage[T1]{fontenc}
\usepackage[utf8]{inputenc}
\usepackage{lmodern}
\usepackage{listings}
\usepackage{graphicx}
\usepackage{float}
\usepackage{subcaption}
\usepackage{hyperref}

% Resetta la numerazione dei chapter quando
% una nuova part viene creata
\makeatletter
\@addtoreset{chapter}{part}
\makeatother

% Rimuove l'indentazione quando si crea un nuovo paragrafo
\setlength{\parindent}{0pt}

\begin{document}
\tableofcontents

%%TODO: definire prima pagina

\include{capitoli/secure_coding/main.tex}
\include{capitoli/ethereum/main.tex}
\include{capitoli/id_managing/main.tex}
\end{document}
\end{document}
\documentclass[a4paper,12 pt]{report}
\usepackage[T1]{fontenc}
\usepackage[utf8]{inputenc}
\usepackage{lmodern}
\usepackage{listings}
\usepackage{graphicx}
\usepackage{float}
\usepackage{subcaption}
\usepackage{hyperref}

% Resetta la numerazione dei chapter quando
% una nuova part viene creata
\makeatletter
\@addtoreset{chapter}{part}
\makeatother

% Rimuove l'indentazione quando si crea un nuovo paragrafo
\setlength{\parindent}{0pt}

\begin{document}
\tableofcontents

%%TODO: definire prima pagina

\documentclass[a4paper,12 pt]{report}
\usepackage[T1]{fontenc}
\usepackage[utf8]{inputenc}
\usepackage{lmodern}
\usepackage{listings}
\usepackage{graphicx}
\usepackage{float}
\usepackage{subcaption}
\usepackage{hyperref}

% Resetta la numerazione dei chapter quando
% una nuova part viene creata
\makeatletter
\@addtoreset{chapter}{part}
\makeatother

% Rimuove l'indentazione quando si crea un nuovo paragrafo
\setlength{\parindent}{0pt}

\begin{document}
\tableofcontents

%%TODO: definire prima pagina

\include{capitoli/secure_coding/main.tex}
\include{capitoli/ethereum/main.tex}
\include{capitoli/id_managing/main.tex}
\end{document}
\documentclass[a4paper,12 pt]{report}
\usepackage[T1]{fontenc}
\usepackage[utf8]{inputenc}
\usepackage{lmodern}
\usepackage{listings}
\usepackage{graphicx}
\usepackage{float}
\usepackage{subcaption}
\usepackage{hyperref}

% Resetta la numerazione dei chapter quando
% una nuova part viene creata
\makeatletter
\@addtoreset{chapter}{part}
\makeatother

% Rimuove l'indentazione quando si crea un nuovo paragrafo
\setlength{\parindent}{0pt}

\begin{document}
\tableofcontents

%%TODO: definire prima pagina

\include{capitoli/secure_coding/main.tex}
\include{capitoli/ethereum/main.tex}
\include{capitoli/id_managing/main.tex}
\end{document}
\documentclass[a4paper,12 pt]{report}
\usepackage[T1]{fontenc}
\usepackage[utf8]{inputenc}
\usepackage{lmodern}
\usepackage{listings}
\usepackage{graphicx}
\usepackage{float}
\usepackage{subcaption}
\usepackage{hyperref}

% Resetta la numerazione dei chapter quando
% una nuova part viene creata
\makeatletter
\@addtoreset{chapter}{part}
\makeatother

% Rimuove l'indentazione quando si crea un nuovo paragrafo
\setlength{\parindent}{0pt}

\begin{document}
\tableofcontents

%%TODO: definire prima pagina

\include{capitoli/secure_coding/main.tex}
\include{capitoli/ethereum/main.tex}
\include{capitoli/id_managing/main.tex}
\end{document}
\end{document}
\documentclass[a4paper,12 pt]{report}
\usepackage[T1]{fontenc}
\usepackage[utf8]{inputenc}
\usepackage{lmodern}
\usepackage{listings}
\usepackage{graphicx}
\usepackage{float}
\usepackage{subcaption}
\usepackage{hyperref}

% Resetta la numerazione dei chapter quando
% una nuova part viene creata
\makeatletter
\@addtoreset{chapter}{part}
\makeatother

% Rimuove l'indentazione quando si crea un nuovo paragrafo
\setlength{\parindent}{0pt}

\begin{document}
\tableofcontents

%%TODO: definire prima pagina

\documentclass[a4paper,12 pt]{report}
\usepackage[T1]{fontenc}
\usepackage[utf8]{inputenc}
\usepackage{lmodern}
\usepackage{listings}
\usepackage{graphicx}
\usepackage{float}
\usepackage{subcaption}
\usepackage{hyperref}

% Resetta la numerazione dei chapter quando
% una nuova part viene creata
\makeatletter
\@addtoreset{chapter}{part}
\makeatother

% Rimuove l'indentazione quando si crea un nuovo paragrafo
\setlength{\parindent}{0pt}

\begin{document}
\tableofcontents

%%TODO: definire prima pagina

\include{capitoli/secure_coding/main.tex}
\include{capitoli/ethereum/main.tex}
\include{capitoli/id_managing/main.tex}
\end{document}
\documentclass[a4paper,12 pt]{report}
\usepackage[T1]{fontenc}
\usepackage[utf8]{inputenc}
\usepackage{lmodern}
\usepackage{listings}
\usepackage{graphicx}
\usepackage{float}
\usepackage{subcaption}
\usepackage{hyperref}

% Resetta la numerazione dei chapter quando
% una nuova part viene creata
\makeatletter
\@addtoreset{chapter}{part}
\makeatother

% Rimuove l'indentazione quando si crea un nuovo paragrafo
\setlength{\parindent}{0pt}

\begin{document}
\tableofcontents

%%TODO: definire prima pagina

\include{capitoli/secure_coding/main.tex}
\include{capitoli/ethereum/main.tex}
\include{capitoli/id_managing/main.tex}
\end{document}
\documentclass[a4paper,12 pt]{report}
\usepackage[T1]{fontenc}
\usepackage[utf8]{inputenc}
\usepackage{lmodern}
\usepackage{listings}
\usepackage{graphicx}
\usepackage{float}
\usepackage{subcaption}
\usepackage{hyperref}

% Resetta la numerazione dei chapter quando
% una nuova part viene creata
\makeatletter
\@addtoreset{chapter}{part}
\makeatother

% Rimuove l'indentazione quando si crea un nuovo paragrafo
\setlength{\parindent}{0pt}

\begin{document}
\tableofcontents

%%TODO: definire prima pagina

\include{capitoli/secure_coding/main.tex}
\include{capitoli/ethereum/main.tex}
\include{capitoli/id_managing/main.tex}
\end{document}
\end{document}
\end{document}
\documentclass[a4paper,12 pt]{report}
\usepackage[T1]{fontenc}
\usepackage[utf8]{inputenc}
\usepackage{lmodern}
\usepackage{listings}
\usepackage{graphicx}
\usepackage{float}
\usepackage{subcaption}
\usepackage{hyperref}

% Resetta la numerazione dei chapter quando
% una nuova part viene creata
\makeatletter
\@addtoreset{chapter}{part}
\makeatother

% Rimuove l'indentazione quando si crea un nuovo paragrafo
\setlength{\parindent}{0pt}

\begin{document}
\tableofcontents

%%TODO: definire prima pagina

\documentclass[a4paper,12 pt]{report}
\usepackage[T1]{fontenc}
\usepackage[utf8]{inputenc}
\usepackage{lmodern}
\usepackage{listings}
\usepackage{graphicx}
\usepackage{float}
\usepackage{subcaption}
\usepackage{hyperref}

% Resetta la numerazione dei chapter quando
% una nuova part viene creata
\makeatletter
\@addtoreset{chapter}{part}
\makeatother

% Rimuove l'indentazione quando si crea un nuovo paragrafo
\setlength{\parindent}{0pt}

\begin{document}
\tableofcontents

%%TODO: definire prima pagina

\documentclass[a4paper,12 pt]{report}
\usepackage[T1]{fontenc}
\usepackage[utf8]{inputenc}
\usepackage{lmodern}
\usepackage{listings}
\usepackage{graphicx}
\usepackage{float}
\usepackage{subcaption}
\usepackage{hyperref}

% Resetta la numerazione dei chapter quando
% una nuova part viene creata
\makeatletter
\@addtoreset{chapter}{part}
\makeatother

% Rimuove l'indentazione quando si crea un nuovo paragrafo
\setlength{\parindent}{0pt}

\begin{document}
\tableofcontents

%%TODO: definire prima pagina

\include{capitoli/secure_coding/main.tex}
\include{capitoli/ethereum/main.tex}
\include{capitoli/id_managing/main.tex}
\end{document}
\documentclass[a4paper,12 pt]{report}
\usepackage[T1]{fontenc}
\usepackage[utf8]{inputenc}
\usepackage{lmodern}
\usepackage{listings}
\usepackage{graphicx}
\usepackage{float}
\usepackage{subcaption}
\usepackage{hyperref}

% Resetta la numerazione dei chapter quando
% una nuova part viene creata
\makeatletter
\@addtoreset{chapter}{part}
\makeatother

% Rimuove l'indentazione quando si crea un nuovo paragrafo
\setlength{\parindent}{0pt}

\begin{document}
\tableofcontents

%%TODO: definire prima pagina

\include{capitoli/secure_coding/main.tex}
\include{capitoli/ethereum/main.tex}
\include{capitoli/id_managing/main.tex}
\end{document}
\documentclass[a4paper,12 pt]{report}
\usepackage[T1]{fontenc}
\usepackage[utf8]{inputenc}
\usepackage{lmodern}
\usepackage{listings}
\usepackage{graphicx}
\usepackage{float}
\usepackage{subcaption}
\usepackage{hyperref}

% Resetta la numerazione dei chapter quando
% una nuova part viene creata
\makeatletter
\@addtoreset{chapter}{part}
\makeatother

% Rimuove l'indentazione quando si crea un nuovo paragrafo
\setlength{\parindent}{0pt}

\begin{document}
\tableofcontents

%%TODO: definire prima pagina

\include{capitoli/secure_coding/main.tex}
\include{capitoli/ethereum/main.tex}
\include{capitoli/id_managing/main.tex}
\end{document}
\end{document}
\documentclass[a4paper,12 pt]{report}
\usepackage[T1]{fontenc}
\usepackage[utf8]{inputenc}
\usepackage{lmodern}
\usepackage{listings}
\usepackage{graphicx}
\usepackage{float}
\usepackage{subcaption}
\usepackage{hyperref}

% Resetta la numerazione dei chapter quando
% una nuova part viene creata
\makeatletter
\@addtoreset{chapter}{part}
\makeatother

% Rimuove l'indentazione quando si crea un nuovo paragrafo
\setlength{\parindent}{0pt}

\begin{document}
\tableofcontents

%%TODO: definire prima pagina

\documentclass[a4paper,12 pt]{report}
\usepackage[T1]{fontenc}
\usepackage[utf8]{inputenc}
\usepackage{lmodern}
\usepackage{listings}
\usepackage{graphicx}
\usepackage{float}
\usepackage{subcaption}
\usepackage{hyperref}

% Resetta la numerazione dei chapter quando
% una nuova part viene creata
\makeatletter
\@addtoreset{chapter}{part}
\makeatother

% Rimuove l'indentazione quando si crea un nuovo paragrafo
\setlength{\parindent}{0pt}

\begin{document}
\tableofcontents

%%TODO: definire prima pagina

\include{capitoli/secure_coding/main.tex}
\include{capitoli/ethereum/main.tex}
\include{capitoli/id_managing/main.tex}
\end{document}
\documentclass[a4paper,12 pt]{report}
\usepackage[T1]{fontenc}
\usepackage[utf8]{inputenc}
\usepackage{lmodern}
\usepackage{listings}
\usepackage{graphicx}
\usepackage{float}
\usepackage{subcaption}
\usepackage{hyperref}

% Resetta la numerazione dei chapter quando
% una nuova part viene creata
\makeatletter
\@addtoreset{chapter}{part}
\makeatother

% Rimuove l'indentazione quando si crea un nuovo paragrafo
\setlength{\parindent}{0pt}

\begin{document}
\tableofcontents

%%TODO: definire prima pagina

\include{capitoli/secure_coding/main.tex}
\include{capitoli/ethereum/main.tex}
\include{capitoli/id_managing/main.tex}
\end{document}
\documentclass[a4paper,12 pt]{report}
\usepackage[T1]{fontenc}
\usepackage[utf8]{inputenc}
\usepackage{lmodern}
\usepackage{listings}
\usepackage{graphicx}
\usepackage{float}
\usepackage{subcaption}
\usepackage{hyperref}

% Resetta la numerazione dei chapter quando
% una nuova part viene creata
\makeatletter
\@addtoreset{chapter}{part}
\makeatother

% Rimuove l'indentazione quando si crea un nuovo paragrafo
\setlength{\parindent}{0pt}

\begin{document}
\tableofcontents

%%TODO: definire prima pagina

\include{capitoli/secure_coding/main.tex}
\include{capitoli/ethereum/main.tex}
\include{capitoli/id_managing/main.tex}
\end{document}
\end{document}
\documentclass[a4paper,12 pt]{report}
\usepackage[T1]{fontenc}
\usepackage[utf8]{inputenc}
\usepackage{lmodern}
\usepackage{listings}
\usepackage{graphicx}
\usepackage{float}
\usepackage{subcaption}
\usepackage{hyperref}

% Resetta la numerazione dei chapter quando
% una nuova part viene creata
\makeatletter
\@addtoreset{chapter}{part}
\makeatother

% Rimuove l'indentazione quando si crea un nuovo paragrafo
\setlength{\parindent}{0pt}

\begin{document}
\tableofcontents

%%TODO: definire prima pagina

\documentclass[a4paper,12 pt]{report}
\usepackage[T1]{fontenc}
\usepackage[utf8]{inputenc}
\usepackage{lmodern}
\usepackage{listings}
\usepackage{graphicx}
\usepackage{float}
\usepackage{subcaption}
\usepackage{hyperref}

% Resetta la numerazione dei chapter quando
% una nuova part viene creata
\makeatletter
\@addtoreset{chapter}{part}
\makeatother

% Rimuove l'indentazione quando si crea un nuovo paragrafo
\setlength{\parindent}{0pt}

\begin{document}
\tableofcontents

%%TODO: definire prima pagina

\include{capitoli/secure_coding/main.tex}
\include{capitoli/ethereum/main.tex}
\include{capitoli/id_managing/main.tex}
\end{document}
\documentclass[a4paper,12 pt]{report}
\usepackage[T1]{fontenc}
\usepackage[utf8]{inputenc}
\usepackage{lmodern}
\usepackage{listings}
\usepackage{graphicx}
\usepackage{float}
\usepackage{subcaption}
\usepackage{hyperref}

% Resetta la numerazione dei chapter quando
% una nuova part viene creata
\makeatletter
\@addtoreset{chapter}{part}
\makeatother

% Rimuove l'indentazione quando si crea un nuovo paragrafo
\setlength{\parindent}{0pt}

\begin{document}
\tableofcontents

%%TODO: definire prima pagina

\include{capitoli/secure_coding/main.tex}
\include{capitoli/ethereum/main.tex}
\include{capitoli/id_managing/main.tex}
\end{document}
\documentclass[a4paper,12 pt]{report}
\usepackage[T1]{fontenc}
\usepackage[utf8]{inputenc}
\usepackage{lmodern}
\usepackage{listings}
\usepackage{graphicx}
\usepackage{float}
\usepackage{subcaption}
\usepackage{hyperref}

% Resetta la numerazione dei chapter quando
% una nuova part viene creata
\makeatletter
\@addtoreset{chapter}{part}
\makeatother

% Rimuove l'indentazione quando si crea un nuovo paragrafo
\setlength{\parindent}{0pt}

\begin{document}
\tableofcontents

%%TODO: definire prima pagina

\include{capitoli/secure_coding/main.tex}
\include{capitoli/ethereum/main.tex}
\include{capitoli/id_managing/main.tex}
\end{document}
\end{document}
\end{document}
\documentclass[a4paper,12 pt]{report}
\usepackage[T1]{fontenc}
\usepackage[utf8]{inputenc}
\usepackage{lmodern}
\usepackage{listings}
\usepackage{graphicx}
\usepackage{float}
\usepackage{subcaption}
\usepackage{hyperref}

% Resetta la numerazione dei chapter quando
% una nuova part viene creata
\makeatletter
\@addtoreset{chapter}{part}
\makeatother

% Rimuove l'indentazione quando si crea un nuovo paragrafo
\setlength{\parindent}{0pt}

\begin{document}
\tableofcontents

%%TODO: definire prima pagina

\documentclass[a4paper,12 pt]{report}
\usepackage[T1]{fontenc}
\usepackage[utf8]{inputenc}
\usepackage{lmodern}
\usepackage{listings}
\usepackage{graphicx}
\usepackage{float}
\usepackage{subcaption}
\usepackage{hyperref}

% Resetta la numerazione dei chapter quando
% una nuova part viene creata
\makeatletter
\@addtoreset{chapter}{part}
\makeatother

% Rimuove l'indentazione quando si crea un nuovo paragrafo
\setlength{\parindent}{0pt}

\begin{document}
\tableofcontents

%%TODO: definire prima pagina

\documentclass[a4paper,12 pt]{report}
\usepackage[T1]{fontenc}
\usepackage[utf8]{inputenc}
\usepackage{lmodern}
\usepackage{listings}
\usepackage{graphicx}
\usepackage{float}
\usepackage{subcaption}
\usepackage{hyperref}

% Resetta la numerazione dei chapter quando
% una nuova part viene creata
\makeatletter
\@addtoreset{chapter}{part}
\makeatother

% Rimuove l'indentazione quando si crea un nuovo paragrafo
\setlength{\parindent}{0pt}

\begin{document}
\tableofcontents

%%TODO: definire prima pagina

\include{capitoli/secure_coding/main.tex}
\include{capitoli/ethereum/main.tex}
\include{capitoli/id_managing/main.tex}
\end{document}
\documentclass[a4paper,12 pt]{report}
\usepackage[T1]{fontenc}
\usepackage[utf8]{inputenc}
\usepackage{lmodern}
\usepackage{listings}
\usepackage{graphicx}
\usepackage{float}
\usepackage{subcaption}
\usepackage{hyperref}

% Resetta la numerazione dei chapter quando
% una nuova part viene creata
\makeatletter
\@addtoreset{chapter}{part}
\makeatother

% Rimuove l'indentazione quando si crea un nuovo paragrafo
\setlength{\parindent}{0pt}

\begin{document}
\tableofcontents

%%TODO: definire prima pagina

\include{capitoli/secure_coding/main.tex}
\include{capitoli/ethereum/main.tex}
\include{capitoli/id_managing/main.tex}
\end{document}
\documentclass[a4paper,12 pt]{report}
\usepackage[T1]{fontenc}
\usepackage[utf8]{inputenc}
\usepackage{lmodern}
\usepackage{listings}
\usepackage{graphicx}
\usepackage{float}
\usepackage{subcaption}
\usepackage{hyperref}

% Resetta la numerazione dei chapter quando
% una nuova part viene creata
\makeatletter
\@addtoreset{chapter}{part}
\makeatother

% Rimuove l'indentazione quando si crea un nuovo paragrafo
\setlength{\parindent}{0pt}

\begin{document}
\tableofcontents

%%TODO: definire prima pagina

\include{capitoli/secure_coding/main.tex}
\include{capitoli/ethereum/main.tex}
\include{capitoli/id_managing/main.tex}
\end{document}
\end{document}
\documentclass[a4paper,12 pt]{report}
\usepackage[T1]{fontenc}
\usepackage[utf8]{inputenc}
\usepackage{lmodern}
\usepackage{listings}
\usepackage{graphicx}
\usepackage{float}
\usepackage{subcaption}
\usepackage{hyperref}

% Resetta la numerazione dei chapter quando
% una nuova part viene creata
\makeatletter
\@addtoreset{chapter}{part}
\makeatother

% Rimuove l'indentazione quando si crea un nuovo paragrafo
\setlength{\parindent}{0pt}

\begin{document}
\tableofcontents

%%TODO: definire prima pagina

\documentclass[a4paper,12 pt]{report}
\usepackage[T1]{fontenc}
\usepackage[utf8]{inputenc}
\usepackage{lmodern}
\usepackage{listings}
\usepackage{graphicx}
\usepackage{float}
\usepackage{subcaption}
\usepackage{hyperref}

% Resetta la numerazione dei chapter quando
% una nuova part viene creata
\makeatletter
\@addtoreset{chapter}{part}
\makeatother

% Rimuove l'indentazione quando si crea un nuovo paragrafo
\setlength{\parindent}{0pt}

\begin{document}
\tableofcontents

%%TODO: definire prima pagina

\include{capitoli/secure_coding/main.tex}
\include{capitoli/ethereum/main.tex}
\include{capitoli/id_managing/main.tex}
\end{document}
\documentclass[a4paper,12 pt]{report}
\usepackage[T1]{fontenc}
\usepackage[utf8]{inputenc}
\usepackage{lmodern}
\usepackage{listings}
\usepackage{graphicx}
\usepackage{float}
\usepackage{subcaption}
\usepackage{hyperref}

% Resetta la numerazione dei chapter quando
% una nuova part viene creata
\makeatletter
\@addtoreset{chapter}{part}
\makeatother

% Rimuove l'indentazione quando si crea un nuovo paragrafo
\setlength{\parindent}{0pt}

\begin{document}
\tableofcontents

%%TODO: definire prima pagina

\include{capitoli/secure_coding/main.tex}
\include{capitoli/ethereum/main.tex}
\include{capitoli/id_managing/main.tex}
\end{document}
\documentclass[a4paper,12 pt]{report}
\usepackage[T1]{fontenc}
\usepackage[utf8]{inputenc}
\usepackage{lmodern}
\usepackage{listings}
\usepackage{graphicx}
\usepackage{float}
\usepackage{subcaption}
\usepackage{hyperref}

% Resetta la numerazione dei chapter quando
% una nuova part viene creata
\makeatletter
\@addtoreset{chapter}{part}
\makeatother

% Rimuove l'indentazione quando si crea un nuovo paragrafo
\setlength{\parindent}{0pt}

\begin{document}
\tableofcontents

%%TODO: definire prima pagina

\include{capitoli/secure_coding/main.tex}
\include{capitoli/ethereum/main.tex}
\include{capitoli/id_managing/main.tex}
\end{document}
\end{document}
\documentclass[a4paper,12 pt]{report}
\usepackage[T1]{fontenc}
\usepackage[utf8]{inputenc}
\usepackage{lmodern}
\usepackage{listings}
\usepackage{graphicx}
\usepackage{float}
\usepackage{subcaption}
\usepackage{hyperref}

% Resetta la numerazione dei chapter quando
% una nuova part viene creata
\makeatletter
\@addtoreset{chapter}{part}
\makeatother

% Rimuove l'indentazione quando si crea un nuovo paragrafo
\setlength{\parindent}{0pt}

\begin{document}
\tableofcontents

%%TODO: definire prima pagina

\documentclass[a4paper,12 pt]{report}
\usepackage[T1]{fontenc}
\usepackage[utf8]{inputenc}
\usepackage{lmodern}
\usepackage{listings}
\usepackage{graphicx}
\usepackage{float}
\usepackage{subcaption}
\usepackage{hyperref}

% Resetta la numerazione dei chapter quando
% una nuova part viene creata
\makeatletter
\@addtoreset{chapter}{part}
\makeatother

% Rimuove l'indentazione quando si crea un nuovo paragrafo
\setlength{\parindent}{0pt}

\begin{document}
\tableofcontents

%%TODO: definire prima pagina

\include{capitoli/secure_coding/main.tex}
\include{capitoli/ethereum/main.tex}
\include{capitoli/id_managing/main.tex}
\end{document}
\documentclass[a4paper,12 pt]{report}
\usepackage[T1]{fontenc}
\usepackage[utf8]{inputenc}
\usepackage{lmodern}
\usepackage{listings}
\usepackage{graphicx}
\usepackage{float}
\usepackage{subcaption}
\usepackage{hyperref}

% Resetta la numerazione dei chapter quando
% una nuova part viene creata
\makeatletter
\@addtoreset{chapter}{part}
\makeatother

% Rimuove l'indentazione quando si crea un nuovo paragrafo
\setlength{\parindent}{0pt}

\begin{document}
\tableofcontents

%%TODO: definire prima pagina

\include{capitoli/secure_coding/main.tex}
\include{capitoli/ethereum/main.tex}
\include{capitoli/id_managing/main.tex}
\end{document}
\documentclass[a4paper,12 pt]{report}
\usepackage[T1]{fontenc}
\usepackage[utf8]{inputenc}
\usepackage{lmodern}
\usepackage{listings}
\usepackage{graphicx}
\usepackage{float}
\usepackage{subcaption}
\usepackage{hyperref}

% Resetta la numerazione dei chapter quando
% una nuova part viene creata
\makeatletter
\@addtoreset{chapter}{part}
\makeatother

% Rimuove l'indentazione quando si crea un nuovo paragrafo
\setlength{\parindent}{0pt}

\begin{document}
\tableofcontents

%%TODO: definire prima pagina

\include{capitoli/secure_coding/main.tex}
\include{capitoli/ethereum/main.tex}
\include{capitoli/id_managing/main.tex}
\end{document}
\end{document}
\end{document}
\end{document}