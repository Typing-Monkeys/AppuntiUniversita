\section{Obbiettivo}

L'obiettivo di questa esercitazione \`{e}  quello di realizzare un cluster di 2 nodi nel quale si dovrà avere \textit{Pacemaker} come Cluster Resource Manager (\textit{CRM}), \textit{Corosync} come \textit{Cluster Engine}, \textit{Apache} come \textit{Web Server} e \textit{DRBD} per creare una risorsa replicata in tutti i nodi del cluster (\textit{Distributed Replicated Storage System}).

\section{Ambiente di Lavoro}

Questa esercitazione \`{e} stata svolta all'interno del seguente ambiente di lavoro:

\begin{itemize}
	\item \textbf{Hardware}: 
		\begin{itemize}
			\item \textbf{CPU}: AMD Ryzen 9 5900x
			\item \textbf{RAM}: 32 GB DDR4 @3200 MHz
		\end{itemize}
	\item \textbf{Software}:
		\begin{itemize}
			\item \textbf{Host OS:} Arch Linux
			\item \textbf{Guest OS}: Fedora 34 Server
			\item \textbf{Virtualization Software}: VirtualBox 6.1
		\end{itemize}
\end{itemize}

\section{Configurazione Macchine}

Per questa esercitazione sono necessarie 2 macchine virtuali che saranno i due nodi del nostro cluster. Ogni macchina \`{e} stata configurata come segue:

\begin{itemize}
	\item \textbf{Cores:} 5 Core
	\item \textbf{RAM:} 5GB
	\item \textbf{Dischi di archiviazione:}
		\begin{itemize}
			\item Disco Principale da 25 GB
			\item Disco per risorsa condivisa da 1 GB
		\end{itemize}
	\item \textbf{Scheda di Rete:} Scheda di rete con Bridge
\end{itemize}

\`{E} consigliato configurare una sola macchina, installare e configurare il sistema operativo e i software necessari, per poi utilizzare la funzione 'Clona' di VirtualBox per generare una copia identica senza dover ripetere tali operazioni nuovamente. Durante la fase di clonazione della macchina \`{e} necessario selezionare l'opzione "Generare nuovi Mac Address per ogni Network Adapter" come policy per la gestione dei Mac Address.

\begin{center}
	%%\includegraphics[scale=0.4]{screens/vb_macpolicy.png}
	\includegraphics[width=\textwidth]{screens/vb_macpolicy.png}
\end{center}
 
\section{Configurazione Software}

\subsection{Sistema Operativo}

Una volta configurata la macchina virtuale e aggiunta la iso di Fedora, avvio la macchina e scelgo, dal menu, la voce \lstinline[style=cmd]|Install Fedora 34|.

\begin{center}
	\includegraphics[width=0.8\textwidth]{screens/fedora_install_1.png}
\end{center}
\ \\
Seleziono la lingua del sistema. In Fedora  34 c'\`{e} un bug per cui se la lingua italiana viene selezionata non viene mostrata la sezione per configurare un nuovo utente nel menu successivo. \`{e} anche per questo motivo che scelgo la lingua inglese.

\begin{center}
	\includegraphics[width=0.8\textwidth]{screens/fedora_install_lingua.png}
\end{center}
\ \\
Dal successivo menu, vado a configurare la formattazione del disco.

% allineo le immagini una affianco all'altra
\includegraphics[width=0.5\textwidth]{screens/fedora_install_menu.png}
\includegraphics[width=0.5\textwidth]{screens/fedora_install_disk.png}
\ \\
A questo punto comparir\`{a} una nuova voce per configurare anche l'utente normale. Procedo prima a scegliere una password per \lstinline[style=cmd]|root| e poi imposto l'utente base.

\begin{center}
	\includegraphics[width=0.8\textwidth]{screens/fedora_install_menu2.png}
\end{center}

% allineo le immagini una affianco all'altra
\includegraphics[width=0.5\textwidth]{screens/fedora_install_rootpwd.png}
\includegraphics[width=0.5\textwidth]{screens/fedora_install_userpwd.png}
\ \\
Avvio il processo di installazione del SO.

\begin{center}
	\includegraphics[width=0.8\textwidth]{screens/fedora_install_installing.png}
\end{center}
\ \\
L'installazione \`{e} completata e riavvio il sistema.

\begin{center}
	\includegraphics[width=0.8\textwidth]{screens/fedora_install_end.png}
\end{center}

\subsection{Software Necessario}

Appena ho terminato la fase di installazione del SO, ho provveduto ad aggiornarlo con il seguente comando, per evitare problemi di compatibilit\`{a} e software obsoleto:

\begin{lstlisting}[style=cmd]
 sudo dnf -y upgrade
\end{lstlisting} 
\ \\
Poi ho installato i pacchetti necessari (\textit{Pacemaker}, \textit{Corosync}, \textit{Apache} e \textit{DRBD}) con: 

\begin{lstlisting}[style=cmd]
 sudo dnf -y install pacemaker corosync pcs
 sudo dnf -y install drbd-pacemaker drbd-udev
 sudo dnf -y install httpd
 sudo dnf -y install iptables-services
\end{lstlisting} 
\pagebreak
\subsection{IP}

Ho assegnato IP statici alle macchine per essere sicuro di poterle sempre raggiungere e che il DHCP del mio router non gli assegni indirizzi diversi col passare del tempo. Per farlo ho utilizzato i seguenti comandi:

\begin{itemize}
	\item Macchina 1: 
	\begin{lstlisting}[style=cmd]
 sudo nmcli connection modify enp0s3 IPv4.address 192.168.178.52/24
 sudo nmcli connection modify enp0s3 IPv4.gateway 192.168.178.1
 sudo nmcli connection modify enp0s3 IPv4.dns 8.8.8.8
 sudo nmcli connection modify enp0s3 IPv4.method manual
	\end{lstlisting}
	\item Macchina 2:
		\begin{lstlisting}[style=cmd]
 sudo nmcli connection modify enp0s3 IPv4.address 192.168.178.53/24
 sudo nmcli connection modify enp0s3 IPv4.gateway 192.168.178.1
 sudo nmcli connection modify enp0s3 IPv4.dns 8.8.8.8
 sudo nmcli connection modify enp0s3 IPv4.method manual
	\end{lstlisting}
\end{itemize}
\ \\
Con la prima riga di ogni set di istruzioni vado a specificare l'indirizzo IP che voglio assegnare alla macchina con tanto di subnet mask, la seconda riga indica l'indirizzo del default gateway e la terza specifica il DNS che voglio utilizzare. \lstinline[style=cmd]|enps0s3| \`{e} il nome della scheda di rete che sto configurando.\\
Per rendere effettive le modifiche basta riavviare il sistema: \lstinline[style=cmd]|sudo reboot|.\\
Si pu\`{o} controllare il successo di questa operazione analizzando l'output del comando
\begin{lstlisting}[style=cmd]
 route -n
\end{lstlisting}
se restituisce qualcosa vuol dire che il tutto \`{e} andato a buon fine.\\

\begin{lstlisting}[style=output]
 Kernel IP routing table
 Destination     Gateway         Genmask       ...   Iface
 0.0.0.0         192.168.178.1   0.0.0.0       ...   enp0s3
 192.168.178.0   0.0.0.0         255.255.255.0 ...   enp0s3
\end{lstlisting}
Un possibile esempio di output.
\pagebreak

\subsection{hosts \& hostname}
\label{sec:hosts}

Ho modificato il file \lstinline[style=cmd]|/etc/hostname| definendo un nome diverso per ogni macchina in modo tale da poterle distinguere pi\`{u} facilmente durante le sessioni SSH: 

\begin{itemize}
	\item Nome Macchina 1: fedoraman
	\item Nome Macchina 2: fedoragirl
\end{itemize}
\ \\
In entrambe le macchine, alla fine del file \lstinline[style=cmd]|/etc/hosts| ho aggiunto le seguenti righe per facilitare poi la configurazione degli altri servizi:

\begin{lstlisting}[style=cmd]
 192.1168.178.52 Fedoraman
 192.1168.178.53 Fedoragirl
\end{lstlisting}

\subsection{Firewall}

Per evitare problemi di comunicazione tra le varie macchine ho deciso di disabilitare il firewall:

\begin{lstlisting}[style=cmd]
 sudo systemctl stop firewalld
 sudo systemctl disable firewalld
\end{lstlisting}

\subsection{PCSD}
\label{sec:pcsd}

In ogni macchina ho cambiato la password dell'utente \lstinline[style=cmd]|hacluster| in quanto servir\`{a} per l'autenticazione dei vari nodi in alcuni passaggi successivi:

\begin{lstlisting}[style=cmd]
 sudo passwd hacluster
\end{lstlisting}
\ \\
Successivamente ho abilitato il servizio e l'ho avviato con:

\begin{lstlisting}[style=cmd]
 sudo systemctl enable pcsd
 sudo systemctl start pcsd
\end{lstlisting}
\pagebreak

\subsection{Corosync}

In entrambe le macchine ho provveduto alla configurazione di \lstinline[style=cmd]|corosync| andando a popolare il file \lstinline[style=cmd]|/etc/corosync/corosync.conf| con il seguente contenuto:

\begin{lstlisting}[style=cmd]
 totem {
   version: 2
   cluster_name: ExampleCluster
   transport: knet
   crypto_cipher: aes256
   crypto_hash: sha256
 }

 nodelist {
   node {
      ring0_addr: Fedoraman
      name: node1
      nodeid: 1
   }
	
   node {
      ring0_addr: Fedoragirl
      name: node2
      nodeid: 2
  }
 }

 quorum {
   provider: corosync_votequorum
   two_node: 1
 }

 logging {
   to_logfile: yes
   logfile: /var/log/cluster/corosync.log
   to_syslog: yes
   timestamp: on
 }
\end{lstlisting}
\pagebreak
Nella sezione \lstinline[style=cmd]|totem| ho impostato il nome del cluster modificando l'attributo \lstinline[style=cmd]|cluster_name|. La sezione \lstinline[style=cmd]|nodelist| conterr\`{a} l'elenco di tutti i nodi del cluster, per ogni nodo ho aggiunto la sottosezione \lstinline[style=cmd]|node{}| con gli attributi:

\begin{itemize}
	\item \lstinline[style=cmd]|ring0_addr|: indica l'indirizzo del nodo, nel mio caso \`{e} \lstinline[style=cmd]|Fedoraman/Fedoragilr| per via della configurazione in \autoref{sec:hosts}
	\item \lstinline[style=cmd]|name|: il nome da assegnare al nodo
	\item \lstinline[style=cmd]|nodeid|: un numero progressivo che identifica il nodo
\end{itemize}

\subsection{Autenticazione dei Nodi}

In ogni macchina ho effettuato l'autenticazione del nodo al cluster con i seguenti comandi:

\begin{lstlisting}[style=cmd]
 sudo pcs client local-auth -u hacluster
 sudo pcs cluster auth -u hacluster
\end{lstlisting}
\ \\
La password da utilzzare in questa fase \`{e} quella dell'utente \lstinline[style=cmd]|hacluster| scelta al punto \autoref{sec:pcsd}

\subsection{Avvio Cluster}

Configuro l'avvio automatico del cluster e lo faccio partire con i seguenti comandi:

\begin{lstlisting}[style=cmd]
 sudo pcs cluster setup ExampleCluster node1 node2 --force
 sudo pcs cluster start --all
 sudo pcs cluster enable --all
\end{lstlisting}
\ \\
Questa operazione deve essere eseguita in una sola macchina, non importa quale, io per comodit\`{a} ho scelto \lstinline[style=cmd]|Fedoraman(node1)|.
\ \\
Controllo il corretto funzionamento del cluster tramite il comando:

\begin{lstlisting}[style=cmd]
 sudo pcs status
\end{lstlisting}
\pagebreak
Se ho un output come il seguente vuol dire che il tutto \`{e} andato a buon fine (\`{e} importante che i nodi risultino \lstinline[style=cmd]|Online|):

\begin{lstlisting}[style=output]
 Cluster name: ExampleCluster
 
 WARNINGS:
 No stonith devices and stonith-enabled is not false
 
 Cluster Summary:
    * Stack: corosync
    * Current DC: node1 (version 2.1.1-9.fc34-77db578727) - partition with quorum
    * Last updated: Wed Nov  3 18:17:32 2021
    * Last change:  Wed Nov  3 18:02:47 2021 by hacluster via crmd on node1
    * 2 nodes configured
    * 0 resource instances configured
 
 Node List:
    * Online: [ node1 node2 ]
 
 Full List of Resources:
    * No resources
 
 Daemon Status:
    corosync: active/enabled
    pacemaker: active/enabled
    pcsd: active/enabled
\end{lstlisting}

\subsection{Cluster Property}

Ho disabilitato le property \lstinline[style=cmd]|stonith| e \lstinline[style=cmd]|quorum| in quanto la prima non \`{e} necessaria ai fini di questa esercitazione e la seconda dato che non ho un numero sufficiente di macchine per utilizzare questa propriet\`{a} (ne servono minimo 3).

\begin{lstlisting}[style=cmd]
 sudo pcs property set stonith-enabled=false
 sudo pcs property set no-quorum-policy=ignore
\end{lstlisting}

\subsection{Cluster IP}

Assegno un IP al cluster in modo tale da raggiungere il server web, che verr\`{a} configurato in seguito, e altri possibili servizi:

\begin{lstlisting}[style=cmd]
 sudo pcs resource create floating_ip ocf:heartbeat:IPaddr2 ip=192.168.178.55 cidr_netmask=24 op monitor interval=60s
\end{lstlisting}
\ \\
L'indirizzo viene specificato con il parametro \lstinline[style=cmd]|ip=| e la subnet mask con \lstinline[style=cmd]|cidr_netmask=|. \`{E} importante notare che l'IP scelto deve appartenere alla stessa rete delle macchine e non deve essere utilizzato da nessun altro dispositivo !

\subsection{Risorsa HTTP}

Aggiungo la risorsa HTTP al cluster in modo tale da avere un server web sempre attivo anche se il nodo principale cade:

\begin{lstlisting}[style=cmd]
 sudo pcs resource create http_server ocf:heartbeat:apache configfile="/etc/httpd/conf/httpd.conf" op monitor timeout="20s" interval="60s"
\end{lstlisting}

\subsection{Partizionamento Disco Condiviso}
\label{sec:partizione}

Procedo al partizionamento del secondo disco presente in tutte e due le macchine.\\
Con il seguente comando posso controllare il nome che gli \`{e} stato assegnato dal SO (di solito \lstinline[style=cmd]|/dev/sdb|):

\begin{lstlisting}[style=cmd]
 sudo fdisk -l
\end{lstlisting}
\ \\
Partiziono effettivamente il disco con:

\begin{lstlisting}[style=cmd]
 sudo fdisk /dev/sdb
\end{lstlisting}
\ \\
Si aprir\`{a} un menu interattivo e dovr\`{o} digirare i seguenti comandi:

\begin{itemize}
	\item \lstinline[style=cmd]|n|: nuova partizione
	\item \lstinline[style=cmd]|p|: partizione primaria
	\item \lstinline[style=cmd]|1|: numero di partizioni
	\item \lstinline[style=cmd]|ENTER|: primo settore (utilizzare il valore di default)
	\item \lstinline[style=cmd]|ENTER|: ultimo settore (utilizzare il valore di default)
	\item \lstinline[style=cmd]|w|: scrive le modifiche
	\item \lstinline[style=cmd]|q|: esce dal programma
\end{itemize}
\pagebreak
Oppure posso effettuare il tutto in maniera autoamtica con:

\begin{lstlisting}[style=cmd]
 sed -e 's/\s*\([\+0-9a-zA-Z]*\).*/\1/' << EOF | fdisk /dev/sdb
    n # new partition
    p # primary partition
    1 # partition number 1
      # default - start at beginning of disk 
      # default - stop at ending of disk 
    w # write the partition table
    q # and we're done
 EOF
\end{lstlisting}

\subsection{Configurazione DRBD}

Prima di tutto, devo andare a modificare la policy di sicurezza di SELinux per DRBD, dato che ne impedisce (di defualt) il corretto funzionamento, tramite il seguetne comando:

\begin{lstlisting}[style=cmd]
 sudo semanage permissive -a drbd_t
\end{lstlisting}
\ \\
Fatto ci\`{o}, posso andare a configurare DRBD popolando il file di configurazione \lstinline[style=cmd]|/etc/drbd.d/wwwdata.res| con il seguente contenuto:

\begin{lstlisting}[style=cmd]
 resource wwwdata {
    protocol C;
    device /dev/drbd0;

    syncer{
        verify-alg sha1;
    }

    net {
        cram-hmac-alg sha1;
        shared-secret "barisoni";
    }
    on fedoraman {
        disk /dev/sdb1;
        address 192.168.178.52:7788;
        meta-disk internal;
    }
    on fedoragirl {
        disk /dev/sdb1;
        address 192.168.178.53:7788;
        meta-disk internal;
    }
 }
\end{lstlisting}
\ \\
Ho scelto una chiave per l'algoritmo di crittografia tramite la propriet\`{a} \lstinline[style=cmd]|shared-secret|. \`{E} importante notare che \lstinline[style=cmd]|on fedoraman| e \lstinline[style=cmd]|on fedoragirl| sono i nomi che ho impostato nel punto \autoref{sec:hosts} e nelle relative sezioni ho specificato il nome del disco da utilizzare (sar\`{a} sempre \lstinline[style=cmd]|/dev/sdb1| per via della configurazione in \autoref{sec:partizione}), l'indirizzo IP della macchina (non sembra possibile utilizzare gli alias definiti in \autoref{sec:hosts}) e una porta a scelta per la comunicazione (ho utilizzato la \lstinline[style=cmd]|7788|)\ \\
\ \\
Creo dunque la risorsa drbd:

\begin{lstlisting}[style=cmd]
 sudo drbdadm create-md wwwdata
\end{lstlisting}
\ \\
Abilito il modulo kernel per rendere sempre utilizzabile la risorsa, anche ai prossimi riavvii:

\begin{lstlisting}[style=cmd]
 sudo modprobe drbd
 echo "drbd" | sudo tee -a /etc/modules-load.d/drbd.conf
\end{lstlisting}
\ \\
Completo la configurazione della risorsa con:

\begin{lstlisting}[style=cmd]
 sudo drbdadm up wwwdata
 sudo drbdadm -- --overwrite-data-of-peer primary all
 
 sudo drbdadm primary --force wwwdata
\end{lstlisting}
\ \\
Il terzo comando deve essere eseguito solo sulla prima macchina !\\
Per monitorare l'avanzamento del processo ho utilizzato il seguente comando: \lstinline[style=cmd]|watch cat /proc/drbd|.\\
\ \\
Infine la abilito per far si che si avvii in automatico:

\begin{lstlisting}[style=cmd]
 sudo systemctl enable drbd
 sudo systemctl start drbd
\end{lstlisting}

\subsection{Formattazione Risorsa DRBD}

Procedo con la formattazione della risorsa DRBD creata in precedenza e con l'aggiunta di una pagina web per controllare il corretto funzionamento:

\begin{lstlisting}[style=cmd]
 mkfs.xfs /dev/drbd0
 sudo mount /dev/drbd0 /mnt
 echo "<h1>Hello World</h1>" | sudo tee /mnt/index.html
 sudo umount /dev/drbd0
\end{lstlisting}

\subsection{Risorsa DRBD}

Aggiungo al cluster la risorsa DRBD creata in precedenza:

\begin{lstlisting}[style=cmd]
 sudo pcs cluster cib drbd_cfg
 sudo pcs -f drbd_cfg resource create WebData ocf:linbit:drbd drbd_resource=wwwdata op monitor interval=60s
 sudo pcs -f drbd_cfg resource promotable WebData promoted-max=1 promoted-node-max=1 clone-max=2 clone-node-max=1 notify=true
 sudo pcs cluster cib-push drbd_cfg --config
\end{lstlisting}
\ \\
Controllo il risultato di questa operazione con il comando:

\begin{lstlisting}[style=cmd]
 sudo pcs status
\end{lstlisting}
\ \\
Mi aspetto un outpit simile al seguente:

\begin{lstlisting}[style=output]
 Cluster name: ExampleCluster
 Cluster Summary:
    * Stack: corosync
    * Current DC: node1 (version 2.1.1-9.fc34-77db578727) - partition with quorum
    * Last updated: Tue Dec 14 10:39:46 2021
    * Last change:  Mon Nov  8 10:35:32 2021 by root via cibadmin on node1
    * 2 nodes configured
    * 5 resource instances configured

 Node List:
    * Online: [ node1 node2 ]

 Full List of Resources:
    * Resource Group: Fedora_group:
        * floating_ip	(ocf::heartbeat:IPaddr2):	 Started node1
        * http_server	(ocf::heartbeat:apache):	 Started node1
    * Clone Set: WebData-clone [WebData] (promotable):
        * Masters: [ node1 ]
        * Slaves: [ node2 ]

 Daemon Status:
    corosync: active/enabled
    pacemaker: active/enabled
    pcsd: active/enabled

\end{lstlisting}
%%incollare outpt con tutte le risorse pronte

\subsection{Risorsa WebFS}

Per ultimo aggiungo al cluster la risorsa WebFS per far si che i dati del Web Server siano replicati in tutti i nodi:

\begin{lstlisting}[style=cmd]
 sudo pcs cluster cib fs_cfg
 sudo pcs -f fs_cfg resource create WebFS Filesystem device="/dev/drbd0" directory="/var/www/html" fstype="xfs"
 sudo pcs -f fs_cfg constraint colocation add WebFS with WebData-clone INFINITY with-rsc-role=Master
 sudo pcs -f fs_cfg constraint order promote WebData-clone then start WebFS
 sudo pcs -f fs_cfg constraint colocation add http_server with WebFS INFINITY
 sudo pcs -f fs_cfg constraint order WebFS then http_server
 sudo pcs cluster cib-push fs_cfg --config
\end{lstlisting}
\ \\
Nei comandi dove compare la clausula \lstinline[style=cmd]|order| vado a specificare l'ordine con cui le risorse devono essere avviate, per esempio con \lstinline[style=cmd]|order WebFS then http_server| sto dicendo al cluster di avviare prima la risorsa WebFs e poi il server web.
\pagebreak
\subsection{Failover Test}

Come test finale ho simulato un failover del nodo principale con il seguente comando:

\begin{lstlisting}[style=cmd]
 sudo pcs node standby node1
\end{lstlisting}
\ \\
Controllo quindi che \lstinline[style=cmd]|node2| venga promosso a master e si avvii la risorsa WebFs:

\begin{lstlisting}[style=cmd]
 sudo pcs status
\end{lstlisting}

\begin{lstlisting}[style=output]
 Cluster name: ExampleCluster
 Cluster Summary:
    * Stack: corosync
    * Current DC: node1 (version 2.1.1-9.fc34-77db578727) - partition with quorum
    * Last updated: Tue Dec 14 10:53:27 2021
    * Last change:  Tue Dec 14 10:53:10 2021 by root via cibadmin on node1
    * 2 nodes configured
    * 5 resource instances configured

 Node List:
    * Node node1: standby
    * Online: [ node2 ]

 Full List of Resources:
    * Resource Group: Fedora_group:
        * floating_ip	(ocf::heartbeat:IPaddr2):	 Started node2
        * http_server	(ocf::heartbeat:apache):	 Started node2
    * Clone Set: WebData-clone [WebData] (promotable):
        * Masters: [ node2 ]
        * Stopped: [ node1 ]
    * WebFS	(ocf::heartbeat:Filesystem):	 Started node2

 Daemon Status:
 corosync: active/enabled
 pacemaker: active/enabled
 pcsd: active/enabled
\end{lstlisting}
\ \\
Una volta assicurato il successo di questa operazione riporto in vita \lstinline[style=cmd]|nodo1| con:

\begin{lstlisting}[style=cmd]
 sudo pcs node unstandby node1
\end{lstlisting}
