% A (minimal) template for problem sets and solutions using the exam document class

\documentclass[answers, 12 pt]{exam}

\usepackage[italian]{babel}
\usepackage{graphicx}
\usepackage[utf8]{inputenc}
\usepackage{float}

\usepackage[table,xcdraw]{xcolor}
\usepackage{setspace}
\usepackage{amssymb}

% Pacchetti matematica
\usepackage{amsmath}
\usepackage{amsthm}
\usepackage{amsfonts}
\usepackage{amssymb}
\usepackage{mathrsfs}

% Commenti multiriga
\usepackage{comment}

\renewcommand{\qedsymbol}{$\blacksquare$}

\renewcommand{\solutiontitle}
{\noindent\textbf{\large Soluzione}\enspace} 

% Sezioni in soluzione es
\newcommand{\mysection}[1]% #1 = title
{\stepcounter{section}%
\setcounter{question}{0}%
\fullwidth{\smallskip\textbf{\normalsize #1}}}

% Mappe di karnaugh
\usepackage{karnaugh-map}

% Disegno del circuito - Usato sito draw.io
% \usepackage{circuitikz}

%%%%%%%%%%%%%%%%%%%%%%%%%%%%%%%%%%%%%%%%%%%%%%%%%

%% Segnalazione manuale di possibili errori
\newcommand\myworries[1]{\textcolor{red}{#1}}

%% \union - Example: \union{j \in J}{A_j}
\newcommand{\union}[2]{\underset{#1}\bigcup #2}

%% \inter - like \union, but with \bigcap
\newcommand{\inter}[2]{\underset{#1}\bigcap #2}

%%%%%%%%%%%%%%%%%%%%%%%%%%%%%%%%%%%%%%%%%%%%%%%%%


\title{Esercitazione Didattica degli Elaboratori}

\begin{document}

% Thesis frontmatter --------------------------------------------

\thispagestyle{empty} %suppress page number

	\noindent % just to prevent indentation narrowing the line width for this line
	\includegraphics[width=0.15\textwidth]{img/logoUniPg}
	\begin{minipage}[b]{0.7\textwidth}
		\centering
		{\Large {\textsc{Universit{\`a} di Perugia}}}\\
		\vspace{0.4 em}
		{\large {Dipartimento di Matematica e Informatica}}
		\vspace{0.6 em}
	\end{minipage}%
	\includegraphics[width=0.15\textwidth]{img/logoDMI}
	
	\vspace{5 em}

	\begin{center}
		
		{\large {\textsc{Esame di Didattica dell'Architettura degli Elaboratori}}}
		\vspace{8 em}
		
		{\Huge {Esercitazione su circuiti combinatori}}
		\vspace{10 em}
		
		\makebox[380pt][c]{{\textit{Professore} \hfill \textit{Studenti}}}
		\makebox[380pt][c]{{\textbf{Prof. Arturo Carpi \hfill Chiara Luchini}}}
		\makebox[380pt][c]{{\textbf{ \hfill  Nicolò Posta}}}
		\makebox[380pt][c]{{\textbf{ \hfill  Tommaso Romani}}}
		\makebox[380pt][c]{{\textbf{ \hfill  Nicolò Vescera}}}
%		\makebox[380pt][c]{\textcolor{blu_dmi}{\textit{Advisor} \hfill \textit{}}}
%		\makebox[380pt][c]{\textcolor{blu_dmi}{\textbf{Dott. Francesco Santini \hfill}}}
		
		\vspace{6 em}
		\vfill
		
	    {\rule{380pt}{.4pt}}\\
		\vspace{1.2 em}
		\large{{Anno Accademico 2021-2022}}
		
		
		
		
	\end{center}

% ------------------------------------------------------------------
\newpage

\section{Prerequisiti}
Per poter svolgere l'esercitazione proposta è necessario che lo studente abbia prima appreso e compreso i seguenti prerequisiti: 

\begin{itemize}
    \item Conoscenza delle notazioni algebriche di base; 
    \item Familiarità con le nozioni base dell'algebra Booleana; 
    \item Saper costruire le tabelle di verità per funzioni booleane;
    \item Conoscenza e applicazione delle mappe di Karnaugh; 
	\item Saper interpretare problemi relativi ai circuiti combinatori; 
    \item Rappresentazione mediante diagrammi dei circuiti combinatori. 
\end{itemize}

\section{Obiettivi}
Scopo dell'esercitazione è quello di condurre lo studente all'applicazione degli strumenti e dei concetti appresi sui circuiti combinatori nel corso di Architettura degli Elaboratori. In particolare, lo studente avrà acquisito le seguenti capacità: 
\begin{enumerate}
    \item Trasformazione delle specifiche verbali nella corrispondente tabella di verità; 
    \item Minimizzazione dell'espressione booleana (attraverso l'uso delle mappe di Karnaugh); 
    \item Disegnare, tramite appositi diagrammi, i circuiti combinatori. 
    
\end{enumerate}
\newpage

\section{Esercizio}
\begin{questions}
\question{
	In un'azienda sono disponibili tre auto aziendali destinate agli spostamenti di lavoro. L'assegnamento avviene attraverso la firma di un documento da parte del soggetto che le utilizza, diventando così unico responsabile dell'auto assegnata. Le quattro figure autorizzate al loro impiego hanno ruoli e responsabilità specifici che determinano la priorità nell'utilizzo dei veicoli.
	\begin{itemize}
		\item Amministratore delegato (A)
		\item Manager (M)
		\item Venditore (V)
		\item Stagista (S)
	\end{itemize}
	
	Si chiede di progettare un sistema automatico per la presa in carico delle automobili, considerando che tutte le figure potrebbero avere tale necessità in contemporanea, considerando il seguente ordine di priorità: A $>$ M $>$ V $>$ S.



  Il sistema di accesso è gestito tramite 4 differenti variabili ``C'', ``D'', ``E'' e ``F'' secondo la codifica riportata in Tabella \ref{tab:codifica}.

    \begin{table}[h!]
        \centering
            \begin{tabular}{ |p{1cm}|p{1cm} |p{1cm}|p{1cm}| p{5cm}|  }
                 \hline
                 \multicolumn{5}{|c|}{Codifica assegnazione auto} \\
                 \hline
                 C & D & E & F & Codice\\
                 \hline
                 0 & 0 & 0 & 0 & Nessun assegnamento\\
                 1 & 0 & 0 & 0 & A\\
                 0 & 1 & 0 & 0 & M\\
                 0 & 0 & 1 & 0 & V\\
                 0 & 0 & 0 & 1 & S\\
                 1 & 1 & 0 & 0 & AM\\
                 1 & 0 & 1 & 0 & AV\\
                 1 & 0 & 0 & 1 & AS\\
                 0 & 1 & 1 & 0 & MV\\
                 0 & 1 & 0 & 1 & MS\\
                 0 & 0 & 1 & 1 & VS\\
                 1 & 1 & 1 & 0 & AMV\\
                 1 & 1 & 0 & 1 & AMS\\
                 1 & 0 & 1 & 1 & AVS\\
                 0 & 1 & 1 & 1 & MVS\\
                 \hline
            \end{tabular}

        \label{tab:codifica}
    \end{table}

}

\newpage
    \begin{solution}
       \mysection{Formulazione}
        Creazione della tabella di verità.
        
            \begin{center}
              \begin{tabular}{cccc|cccc|c}
                A & M & V & S 	& F & E & D & C & Assegnamento\\
                \hline
                0 & 0 & 0 & 0 	& 0 & 0 & 0 & 0 & Nessuno\\
                0 & 0 & 0 & 1 	& 1 & 0 & 0 & 0 & S\\
                0 & 0 & 1 & 0 	& 0 & 1 & 0 & 0 & V\\
                0 & 0 & 1 & 1 	& 1 & 1 & 0 & 0 & VS\\
                0 & 1 & 0 & 0 	& 0 & 0 & 1 & 0 & M\\ 
                0 & 1 & 0 & 1 	& 1 & 0 & 1 & 0 & MS\\
                0 & 1 & 1 & 0 	& 0 & 1 & 1 & 0 & MV\\
                0 & 1 & 1 & 1 	& 1 & 1 & 1 & 0 & MVS\\
                1 & 0 & 0 & 0 	& 0 & 0 & 0 & 1 & A\\
                1 & 0 & 0 & 1 	& 1 & 0 & 0 & 1 & AS\\
                1 & 0 & 1 & 0 	& 0 & 1 & 0 & 1 & AV\\
                1 & 0 & 1 & 1 	& 1 & 1 & 0 & 1 & AVS\\
                1 & 1 & 0 & 0 	& 0 & 0 & 1 & 1 & AM\\
                1 & 1 & 0 & 1 	& 1 & 0 & 1 & 1 & AMS\\
                1 & 1 & 1 & 0 	& 0 & 1 & 1 & 1 & AMV\\
                1 & 1 & 1 & 1 	& 0 & 1 & 1 & 1 & AMV\\
              \end{tabular}
            \end{center}
            
        \vspace{-1em}
            
        \mysection{Ottimizzazione}
            Creazione delle Mappe di Karnaugh.
            \begin{enumerate}
                    \item Mappa di Karnaugh per F.
                    
                        \begin{center}
                            \begin{karnaugh-map}[4][4][1][$VS$][$AM$]
                                \manualterms{
                                	0,1,0,1,
                                	0,1,0,1,
                                	0,1,0,1,
                                	0,1,0,0}
                                \implicant{1}{7}
                                \implicant{1}{9}
                                \implicantedge{1}{3}{9}{11}
                             \end{karnaugh-map}
                        \end{center}
                        \[ F = \overline{V}S  +  \overline{A}S + \overline{M}S  \]
                    

                    \item Mappa di Karnaugh per E.
                    
                        \begin{center}
                            \begin{karnaugh-map}[4][4][1][$VS$][$AM$]
                                \manualterms{                  0,0,1,1,
                                	0,0,1,1,
                                	0,0,1,1,
                                	0,0,1,1}
                                \implicant{3}{10}
                             \end{karnaugh-map}
                        \end{center}
                    \[ E = V \]
             

                    \item Mappa di Karnaugh per D.
                    
                        \begin{center}
                            \begin{karnaugh-map}[4][4][1][$VS$][$AM$]
                                \manualterms{                  0,0,0,0,
									1,1,1,1,
									0,0,0,0,
									1,1,1,1}
                                \implicant{4}{14}
                             \end{karnaugh-map}
                        \end{center}
                    \[ D = M \]
                    
                    \newpage
              
                    \item Mappa di Karnaugh per C.
                    
	                    \begin{center}
	                    	\begin{karnaugh-map}[4][4][1][$VS$][$AM$]
	                    		\manualterms{                  0,0,0,0,
	                    			0,0,0,0,
	                    			1,1,1,1,
	                    			1,1,1,1}
	                    		\implicant{12}{10}
	                    	\end{karnaugh-map}
	                    \end{center}
	                    \[ C = A \]
                    
                    
            \end{enumerate}
            
            \mysection{Disegno del circuito}
            Disegno del circuito combinatorio non minimizzato.

            \begin{center}
                \includegraphics[width=13cm, keepaspectratio]{img/circuito.png}
            \end{center}
            
            \newpage
            
            Le formule usate nella realizzazione del circuito combinatorio sono risultate essere già minimizzate quindi abbiamo cercato di lavorare sull'utilizzo di porte logiche equivalenti ma che ci permettessero di risparmiare sulle componenti. Alla fine, siamo arrivati inizialmente al seguente risultato...
            
            \begin{center}
				\includegraphics[width=9cm, keepaspectratio]{img/circuito_minimizzato_1.png}
            \end{center}
            
            ... per poi accorgerci che poteva essere minimizzato ulteriormente nel seguente:
            
            \begin{center}
            	\includegraphics[width=9cm, keepaspectratio]{img/circuito_minimizzato_2.png}
            \end{center}
                 

    \end{solution}
\end{questions}

\end{document}