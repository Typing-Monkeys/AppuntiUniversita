\section{Prerequisiti}
Per poter svolgere l'esercitazione proposta è necessario che lo studente abbia prima appreso e compreso i seguenti prerequisiti: 

\begin{itemize}
    \item Conoscenza delle notazioni algebriche di base; 
    \item Familiarità con le nozioni base dell'algebra Booleana; 
    \item Saper costruire le tabelle di verità per funzioni booleane;
    \item Conoscenza e applicazione delle mappe di Karnaugh; 
	\item Saper interpretare problemi relativi ai circuiti combinatori; 
    \item Rappresentazione mediante diagrammi dei circuiti combinatori. 
\end{itemize}

\section{Obiettivi}
Scopo dell'esercitazione è quello di condurre lo studente all'applicazione degli strumenti e dei concetti appresi sui circuiti combinatori nel corso di Architettura degli Elaboratori. In particolare, lo studente avrà acquisito le seguenti capacità: 
\begin{enumerate}
    \item Trasformazione delle specifiche verbali nella corrispondente tabella di verità; 
    \item Minimizzazione dell'espressione booleana (attraverso l'uso delle mappe di Karnaugh); 
    \item Disegnare, tramite appositi diagrammi, i circuiti combinatori. 
    
\end{enumerate}