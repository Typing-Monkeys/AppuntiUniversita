\section{Prerequisiti}
Per poter svolgere adeguatamente questa esercitazione è necessario possedere una conoscenza di base dell'algebra Booleana, avere familiarità con la costruzione delle mappe di Karnaugh, capacità di formalizzare problemi relativi ai circuiti combinatori e saperli rappresentare mediante diagrammi.

\section{Obiettivi per l'esercitazione}
L'obiettivo di questa esercitazione è consentire agli studenti di interpretare problemi reali e di estrarre gli elementi fondamentali per la loro risoluzione, utilizzando gli strumenti acquisiti nel corso di Architettura degli Elaboratori. Gli studenti saranno in grado di progettare e minimizzare circuiti logici combinatori, definendone le specifiche, implementandoli e ottimizzandoli.