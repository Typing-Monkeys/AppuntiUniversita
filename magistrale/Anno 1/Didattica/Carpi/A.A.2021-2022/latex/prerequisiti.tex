\section{Prerequisiti}
Per questa esercitazione sono necessari i seguenti prerequisiti:

\begin{itemize}
    \item Nozioni base dell'algebra Booleana;
    \item Conoscenza e applicazione delle mappe di Karnaugh;
    \item Saper formalizzare problemi riguardanti circuiti combinatori;
    \item Saper rappresentare circuiti combinatori tramite diagrammi;
\end{itemize}

\section{Obiettivi}
Alla fine dell'esercitazione lo studente saprà interpretare il problema proposto e risolverlo con gli strumenti appresi durante il corso di Architettura degli Elaboratori. Inoltre avrà acquisito la capacità di progettare e minimizzare un circuito logico combinatorio a partire da dei dati iniziali suddividendolo in diversi passaggi fra cui:
\begin{enumerate}
    \item Definizione delle specifiche;
    \item Sintesi;
    \item Ottimizzazione;
    \item Implementazione;
    \item Verifica.
\end{enumerate}