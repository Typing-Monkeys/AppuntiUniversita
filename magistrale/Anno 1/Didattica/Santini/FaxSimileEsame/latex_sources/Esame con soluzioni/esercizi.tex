
%------------------------- INTESTAZIONE COMPITO -----------------------------------
\begin{center}
	\fbox{\fbox{\parbox{7in}{\centering
				Prova scritta Programmazione Procedurale con Lab. - Cerami Cristian}}}
\end{center}

\vspace{5mm}

\noindent\makebox[\textwidth]{Nome e Cognome: \rule{8cm}{.1pt} \hspace{1cm} Matricola:  \rule{5cm}{.1pt}}

\begin{questions}

%------------------------- ESERCIZIO 1 --------------------------------------------

\question[5]
Cosa stampa il seguente frammento di codice?

\begin{minipage}[t]{0.4\linewidth}
	\begin{lstlisting}[language=C]
	int a = 0123 ^ 0x056;
	double b = 2.59;

	printf ("%d\n", a);
	
	while ((++a || a++) ? a-=1 : 0) {
		if (!(a-- && --a ))
			break;
		else {
			printf("%d\n", a); 
		}
	}
	
	a<=a, a+=b, a++;
	printf("a: %d\n", a); 
	\end{lstlisting}
\end{minipage}
\begin{minipage}[t]{0.6\linewidth}
	\makenonemptybox{120pt}{~\\
	5\\
	3\\
	1\\
	-1\\
	a: 4}
\end{minipage}

%------------------------- FINE ESERCIZIO 1 ---------------------------------------

%------------------------- ESERCIZIO 2 --------------------------------------------

\question[6]
Elencare le conversioni di tipo \underline{implicite} (\emph{... da ... a}). Scrivere cosa viene stampato a schermo sapendo che: \emph{UCHAR\_MAX = 255} , \emph{'a' = 97} .

\begin{minipage}[t]{0.4\linewidth}
	\begin{lstlisting}[language=C]
	double fun (float a) {
		char b = ('x' * 3) - 'g';
		return (a / b);
	}
	
	int main (void) {
		unsigned int a = 'g' - 3UL;
		float b = fun(a);
		unsigned char c = -(int) (b+53);
		printf("c: %c, %d\n", c, c);
		return 0;
	}
	\end{lstlisting}
\end{minipage}
\begin{minipage}[t]{0.6\linewidth}
	\makenonemptybox{200pt}{~\\
linea 7: 'g' convertito da int a unsigned long int\\
linea 7: il valore dopo l'uguale è convertito da unsigned long int ad unsigned int\\
linea 8: parametro "a" di fun convertito da unsigned int a float\\
linea 2: il valore dopo l'uguale è convertito da int a char\\
linea 3: "b" è convertito da char a float per la divisione\\
linea 3: il risultato della divisione è convertito da float a double\\
linea 8: il valore di ritorno è convertito da double a float\\
linea 9: 53 è convertito da int a float\\
linea 9: il valore dopo l'uguale è converito da int (dopo la conversione esplicita) a unsigned char\\

A schermo viene stampato ``c: g, 103'' perchè: \\c = (UCHAR\_MAX + 1) - 153 = 103 = 'g' in ASCII\\
}
\end{minipage}

%------------------------- FINE ESERCIZIO 2 ---------------------------------------

%------------------------- ESERCIZIO 3 --------------------------------------------

\question[6]
Data la seguente \emph{struct}, scrivere la definizione di una funzione di nome \emph{ritorna\_dispari} che prende come parametro una lista (\emph{lista\_input}) e ritorna un'altra lista (\emph{lista\_output}, creata nella funzione) che contiene, nello stesso ordine della lista passata, solamente gli elementi in posizione \emph{dispari} (se presenti). Se la lista originale è \textbf{5-2-9}, la lista ritornata sarà \textbf{5-9}.

\begin{minipage}[t]{0.4\linewidth}
	\begin{lstlisting}[language=C]
	typedef struct node Node;

	struct node {
		int info;
		struct node* pNext;
	};
	\end{lstlisting}
\end{minipage}
\begin{minipage}[t]{0.6\linewidth}
	\makenonemptybox{30pt}{
		\centering 
		\vspace{0.8em} 
		Guarda soluzione in fondo al compito
	}
\end{minipage}

%------------------------- FINE ESERCIZIO 3 ---------------------------------------

\newpage

%------------------------- INTESTAZIONE COMPITO -----------------------------------
\begin{center}
	\fbox{\fbox{\parbox{7in}{\centering
				Prova scritta Programmazione Procedurale con Lab. - Cerami Cristian}}}
\end{center}

\vspace{5mm}

\noindent\makebox[\textwidth]{Nome e Cognome: \rule{8cm}{.1pt} \hspace{1cm} Matricola:  \rule{5cm}{.1pt}}


%------------------------- ESERCIZIO 4 --------------------------------------------

\question[7]
Dire quali compilazioni provocano errore a causa del linker (e perchè):\\
1) gcc -o write write.c\\
2) gcc -c main.c\\
3) gcc -o main main.c\\
4) gcc -o execute main.c write.c\\
In caso il punto 4) ritorni un errore, descrivere come può essere corretto. Infine, \textbf{dopo la correzione} eventualmente applicata, elencare tutte le definizioni, dichiarazioni e tipologie di linkage, presenti in ogni file, per \emph{count}, \emph{i}, \emph{a}, e \emph{mywrite}. Cosa stampa il programma?

\begin{minipage}[h]{0.5\linewidth}
	\includegraphics[width=9cm, keepaspectratio]{immagini/es_linkage_uniti}
	\begin{minipage}[h]{0.95\linewidth}
		\makenonemptybox{240pt}{
			~\\
			Stampa:\\
			3\\
			4\\
			5\\
			6\\
		}
	\end{minipage}
\end{minipage}
\begin{minipage}[h]{0.5\linewidth}
	\makenonemptybox{385pt}{
1) Manca definizione \emph{main} ed \emph{i}\\
3) Manca definizione \emph{mywrite} e \emph{count}\\
4) Manca definizione \emph{count} in \emph{main.c} perchè \emph{count} ha linkage interno in \emph{write.c} quindi non è visibile. Manca poi la dichiarazione con linkage esterno di \emph{i} in \emph{write.c}\\

Il punto 4) può essere corretto cambiando la tipologia del linkage di \emph{count} (globale) in \emph{write.c} da interno a esterno. Si fa eliminando la keyword ``\emph{static}'': \\\texttt{int count = -3;}\\
Va inoltre inserita in \emph{write.c} la dichiarazione di \emph{i} con linkage esterno: \texttt{extern int i;} \\

In main.c:\\
- \emph{mywrite} è dichiarata ed ha linkage esterno\\
- \emph{count} è dichiarata ed ha linkage esterno\\
- \emph{i} a riga 3 è un tentativo di definizione e ha linkage esterno\\
- \emph{i} a riga 4 è ora definita e ha linkage esterno\\

In write.c:\\
- DOPO LA CORREZIONE: \emph{count} a riga 3 è definita e ha linkage esterno\\
- \emph{i} è dichiarata ed ha linkage esterno\\
- \emph{mywrite} è definita e ha linkage esterno\\
- \emph{a} locale in mywrite è definita e ha no linkage\\
- \emph{count} locale in mywrite è definita e ha no linkage\\
}
\end{minipage}


%------------------------- FINE ESERCIZIO 4 ---------------------------------------

%------------------------- ESERCIZIO 5 --------------------------------------------

\question[6]
Cerchiare le affermazioni vere dato:\\ 
\underline{\emph{int a[7]= \{21,-21,[3]=INT\_MAX, 65537, [6]=511\}; short *ptr = (short*) a; char *n = (char*) a;}} ~~sapendo che i tre tipi usati occupano 4, 2 e 1 byte e 65536 = $2^{16}$ (valori rappresentati in complemento a due e \emph{little endian}). Rappresentare la zona di memoria in cui è memorizzato l'array.\\~

\textbf{A.} n+5 >= \&ptr[3]; \textbf{B.} *(n+5) > *(n+4); \textbf{C.} \&ptr[8] == ptr+9; \textbf{D.} ((int)(ptr+8)-(int)(\&a[2]) < 8);\\ \textbf{E.} *(ptr+1) == *(a+2)


\end{questions}
