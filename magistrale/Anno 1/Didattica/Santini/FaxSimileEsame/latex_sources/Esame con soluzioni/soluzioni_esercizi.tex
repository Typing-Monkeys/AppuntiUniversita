%------------------------- SOLUZIONE ESERCIZIO 3 --------------------------------------------

\begin{minipage}[h]{\linewidth}
	SOLUZIONE ESERCIZIO 3
    \begin{lstlisting}[language=C]
Node* ritorna_dispari(Node* lista_input) {
	
	if (lista_input == NULL) {
		return NULL;
	} else {
		int counter = 0;
		Node* pScan = lista_input;
		
		Node* lista_output = NULL;
		Node* lista_output_pLast = NULL;
		
		while (pScan != NULL) {
			if (counter % 2 == 0) {
				Node* pNew = (Node*) malloc(sizeof(Node));
				pNew -> info = pScan -> info;
				pNew -> pNext = NULL;
				
				if (lista_output == NULL) {
					lista_output = pNew;
					lista_output_pLast = pNew;
				} else {
					lista_output_pLast -> pNext = pNew;
					lista_output_pLast = pNew;
				}	    		
			}
			
			pScan = pScan -> pNext;
			counter++;
		}
		
		return lista_output;
	}
}
	\end{lstlisting}
\end{minipage}

\begin{minipage}[h]{\linewidth}
SOLUZIONE ESERCIZIO 5\\


10101000\hspace{2em}    a[0]\\
00000000\\    
00000000\hspace{2em}    *(ptr+1)\\
00000000\\
\\
11010111\hspace{2em}    *(n+4)\\
11111111\hspace{2em}    *(n+5)\\
11111111\hspace{2em}    \&ptr[3]\\
11111111\\
\\
00000000\hspace{2em}    *(a+2) e a[2]\\
00000000\\
00000000\\
00000000\\
\\
11111111\\
11111111\\
11111111\\
11111110\\
\\
10000000\hspace{2em}    \&ptr[8] o ptr+8\\
00000000\\
10000000\hspace{2em}    ptr+9\\
00000000\\
\\
00000000\\
00000000\\
00000000\\
00000000\\
\\
11111111\\
10000000\\
00000000\\
00000000\\
\\

A = FALSO, (n+5) sta in una cella di memoria inferiore a (\&ptr[3])\\
B = VERO, ∗(n+5) == -1, ∗(n+4) == -21 quindi -1 > -21\\
C = FALSO, ricordati che si sta guardando l'indirizzo di memoria e non il loro contenuto\\
D = FALSO, la differenza in byte tra i due puntatori è 8\\
E = VERO, *(ptr+1) == 0, *(a+2) == 0 quindi *(ptr+1) == *(a+2)\\

\end{minipage}
