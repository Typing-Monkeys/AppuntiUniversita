%\chapter{Algoritmo di Edmonds-Karp o Dinits(?)}
\chapter{Algoritmo di Edmonds-Karp}
\section{Introduzione}
L'algoritmo di Ford-Fulkerson più che un algoritmo è un procedimento
concettuale, dato che molte caratteristiche non sono specificate.\\

Altri hanno studiato delle implementazioni di Ford-Fulkerson specificando
l'oridne ci scelta dei nodi su cui effettuare l'augment, due di queste sono lo
\textbf{shortest path max flow} e il \textbf{fat flow}.


\section{Shortest Path Max Flow}
Migliora la complessità di Ford-Fulkerson scegliendo il cammino aumentante in
base al risultato di una ricerca in ampiezza, preferendo quindi il cammino più
corto tra il nodo e il sink.\\

Il cammino corretto può essere trovato in tempo $O(E)$ eseguendo una
\texttt{BFS} nel grafo residuale, e ci garantisce di terminare in tempo
polinomiale, tagliando quindi la dipendenza dell'algoritmo di Ford-Fulkerson
dalle capacità degli archi.

\subsection{Analisi dell'algoritmo}
Ora andremo a dimostrare la terminazione polinomiale dell'algoritmo con le
seguenti 2 osservazioni, ma prima ci definiamo $\delta_f(u,v)$ come la distanza
del cammino minimo da $u$ a $v$ nel grafo residuale $G_f$, considerando che ogni
arco ha distanza unitaria.

\begin{myblockquote}
    \begin{minipage}{\textwidth}
        \begin{theorem}
            Se l'agloritmo di Edmonds-Karp viene eseguito su una rete di flusso
            $G = (V, E)$ con sorgente $s$ e pozzo $t$, allora per ogni
            vertice $v \in V - \{s,t\}$, la distanza de cammino minimo
            $\delta_f(s,v)$ nella rete residua $G_f$ aumenta \linebreak
            monotonicamente per ogni aumento di flusso.
        \end{theorem}
    \end{minipage}
\end{myblockquote}

\begin{proof}
    Supponiamo per assurdo che per qualche vertice $v \in V
        - \{s, t\}$ ci sia un aumento di flusso che provoca una diminuzione della
    distanza del cammino minimo da $s$ a $v$. Sia $f$ il flusso appena prima del
    primo aumento che riduce una distanza del cammino minimo; sia $f^{'}$ il flusso
    subito dopo. Se $v$ è il vertice con il minimo $\delta_{f^{'}}(s,v)$ la cui
    distanza è stata ridotta dall'aumento, allora $\delta_{f^{'}}(s,v) <
        \delta_{f}(s,v)$. Se $p = s \rightsquigarrow u \rightarrow v$ è un cammino
    minimo da $s$ a $v$ in $G_{f{'}}$, allora $(u,v) \in E_{f^{'}}$ e:

    $$
        \delta_{f^{'}}(s,u) = \delta_{f^{'}}(s,v) -1
    $$

    Per il modo in cui abbiamo scelto $v$, sappiamo che la distanza del vertice
    $u$ dalla sorgente $s$ non è \textbf{aumentata(?)}, ovvero:
    $$
        \delta_{f^{'}}(s,u) \ge \delta_{f}(s,u)
    $$
    Noi asseriamo che $(u,v) \notin E_f$. \textbf{Perchè?}

    Se avessimo $(u,v) \in E_f$ allora dovremmo avere anche:

    $$
        \delta_{f}(s,v) \le \delta_{f}(s,u) +1 \\
        \le \delta_{f^{'}}(s,u) -1 \\
        = \delta_{f^{'}}(s,v) -2
    $$
    Questo contraddice l'ipotesi che $\delta_{f^{'}}(s,v) < \delta_{f}(s,v)$.

    Come è possibile avere $(u,v) \notin E_f$ e $(u,v) \in E_f^{'}$? L'augment
    deve avere incrementato il flusso da $v$ a $u$. L'algoritmo di Edmondo-Karp
    aumenta sempre il flusso lungo tutto i cammini minimi, e quindi il cammino
    minimo da $s$ a $u$ in $G_f$ ha $(v,u)$ come suo ultimo arco, per tanto si
    ha:

    $$
        \delta_{f}(s,v) = \delta_{f}(s,u) -1 \\
        \le \delta_{f^{'}}(s,u) -1 \\
        = \delta_{f^{'}}(s,v) -2
    $$

    Questo contraddice l'ipotesi che $\delta_f^{'}(s,v) < \delta_f(s,v)$, quindi
    l'ipotesi dell'esistenza di un tale vertice $v$ non è corretta.\\
\end{proof}

%\newpage
\textbf{Il prossimo teorema limita il numero di iterazioni dell'algoritmo}\\

\begin{minipage}{\textwidth}
    \begin{myblockquote}
        \begin{theorem}
            Se l'algoritmo viene eseguito su una rete di flusso $G = (V, E)$ con
            sorgente $s$ e pozzo $t$, allora il numero totale di aumenti di flusso
            effettuati dall'algoritmo è $O(VE)$.
        \end{theorem}
    \end{myblockquote}
\end{minipage}

\begin{proof}
    Diciamo che un arco $(u,v)$ di una rete residua $G_f$ è \textbf{critico} in un
    cammino aumentante $p$ se la capacità residua di $p$ è la capacità residua di
    $(u,v)$. Dopo aver aumentato il flusso lungo un cammino aumentante ogni arco
    critico scompare dalla rete residua. Inoltre in ogni cammino aumentante ci deve
    essere almeno un arco critico.

    Dimostreremo che ogni arco può diventare critico al più $\frac{|V|}{2}$ volte.\\

    Siano $u$ e $v$ due vertici collegati da un arco, poichè i cammini aumentanti
    sono i cammini minimi, quando $(u,v)$ diventa \textbf{critico} si ha:
    $$
        \delta_f(s,v) = \delta_{f^{'}}(s,u) +1
    $$
    Una volta che il flusso viene aumentato l'arco $(u,v)$ scompare dalla rete
    residua e non potrà riapparire successivamente in un altro cammino aumentante
    fino a che il flusso da $u$ a $v$ non diminuirà, e questo accade solo se $(u,v)$
    apppare in un cammino aumentante.

    Se consideriamo che $f^{'}$ è il flusso quando si verifica questo evento, allora
    si ha

    $$
        \delta_{f^{'}}(s,u) = \delta_{f^{'}}(s,v) +1
    $$
    Poichè $\delta_f(s,v) \le \delta_{f^{'}}(s,v)$ per la dimostrazione precedente,
    si ha:

    $$
        \delta_{f^{'}}(s,u) =  \delta_{f^{'}}(s,v) + 1 \\
        \ge \delta_f(s,v) + 1
        = \delta_f(s,v) +2
    $$


    Di conseguenza, dall'istante in cui $(u,v)$ diventa critico, all'istante in cui
    diventa di nuovo critico, la distanza di $u$ dalla sorgente aumenta almeno di 2
    unità.

    La distanza di $u$ dalla sorgente inizialmente è almeno pari a 0.

    I vertici intermedi in un cammino minimo da $s$ a $u$ non possono contenere $s,
        u$ o $t$. Pertanto, fino al momento in cui u diventa irraggiungibile dalla
    sorgente, se mai si verifica la cosa, la sua distanza sarà al massimo $|V| -2$.
    Quindi dopo la prima volta che diventa critico esso può ridiventarlo al massimo
    altre $\frac{|V|}{2}$ volte.

    Poichè ci sono $O(E)$ coppie di vertici che possono essere connessi da un arco
    in un grafo residuale, il numero di archi critici durante l'intera esecuzione
    dell'algoritmo è  $O(VE)$.
\end{proof}

\section{Fat Flow}
Migliora la complessità di Ford-Fulkerson scegliendo il cammino aumentante con
il più grande bottleneck disponibile.\\

È relativamente facile far vedere come il cammino con il massimo bottleneck da
$s$ a $t$ in un grafo direzionato può essere calcolato in tempo di $O(E\log{V})$
usando una variante dell'algoritmo di Dijkstra. Semplicemente si crea uno
spanning tree direzionato $T$, con radice in $s$. Ripetutamente trovando l'arco
con capacità maggiore uscente da $T$ e lo si aggiunge a $T$ stesso finchè $T$
non conterrà un cammino da $s$ a $t$.

\subsection{Analisi dell'algoritmo}

Possiamo ora analizzare l'algoritmo in termini di maximum flow $f^{*}$. Sia $f$
un qualsiasi flow in $G$, e $f'$ il maximum flow nel grafo residuale corrente
$G_f$. (All'inizio dell'algoritmo $G_f = G$ e $f' = f^{*}$.) Sia $e$ l'arco del
bottleneck nel prossimo augmenting path. Sia $S$ il set di nodi raggiungibili da
$s$ attraverso archi in $G$ con capacità maggiore di $c(e)$ e sia $T = V$
\textbackslash $S$. Per costruzione, $T$ non è vuoto, e ogni arco da $S$ a $T$
ha capacità al massimo $c(e)$. Perciò la capacità del taglio $(S,T)$ è al
massimo $c(e) \cdot E$. D'altro canto, il teorema del Maxflow-Mincut implica che
$||S,T|| \geq |f|$. Da ciò si evince che $c(e) \geq |f|/E$.\\

La dimostrazione precedente implica che aumentando $f$ attraverso il path con il
maggior bottleneck in $G_f$ moltiplica il valore del flusso massimo in $G_f$ di
un fattore di al massimo $1 - 1/E$. In altre parole il flusso residuale
\textit{decade esponenzialmente} all'aumentare delle iterazioni. Dopo $E \cdot
    \ln{|f^{*}|}$ iterazioni il valore del flusso massimo in $G_f$ sarà al amssimo:
$$
    |f^{*}|\cdot(1-1/E)^{E\cdot \ln{|f^{*}|}} < |f^{*}|e^{-\ln{|f^{*}|}} = 1
$$
(La $e$ in questione è la costante di Eulero, non l'arco) In particolare,
\textit{se tutte le capacità sono interi}, allora dopo $E\cdot \ln{|f^{*}|}$
iterazioni la capacità massima del grafo residuale sarà \textit{zero} e $f$ sarà
il flusso massimo.\\

Concludiamo che per grafi con capacità intere l'algoritmo \textbf{Fat Flow}
richiede $O(E^2\log{E}\log{|f^{*}|})$.