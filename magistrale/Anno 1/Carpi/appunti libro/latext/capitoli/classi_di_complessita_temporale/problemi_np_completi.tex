\section{Problemi NP-completi}

Iniziamo con una
\paragraph{Definizione.} Siano $S, S^{\prime}$ due problemi su alfabeti $A,
    A^{\prime}$ rispettivamente. Diciamo che $S^{\prime}$ si \textit{riduce in tempo
    polinomiale} a $S$ e scriviamo $S^{\prime} \leq_p S$ se e solo se c'è una funzione $f$
da $A^{\prime \star}$ in $A^{\star}$ che si computa in tempo al più
polinomiale rispetto alla lunghezza dell'input e soddisfa, per ogni $w \in
    A^{\prime \star}$,
$$
    w \in S^{\prime} \text { se e solo se } f(w) \in S \text {. }
$$

\paragraph{Esempi.}
\begin{enumerate}
    \item Nel precedente paragrafo 1, esempio 2, abbiamo visto che $2 C O L
              \leq_p 2 S A T$ tramite la funzione $G \mapsto I(G)$.
    \item $I S, V C, C L I Q U E$ sono l'uno riducibile all'altro tramite
          $\leq_p$ (si ricordi quanto osservato nel paragrafo 8.2, esempio 4).
\end{enumerate}

Adesso diciamo:

\paragraph{Definizione.} Un problema $S$ è
\begin{itemize}
    \item $N P$-\textit{arduo} se, per ogni $S^{\prime} \in N P, S^{\prime}
              \leq_p S$,
    \item $N P$-\textit{completo} se è $N P$-\textit{arduo} e sta in $P$.
\end{itemize}

È semplice osservare

\paragraph{Proposizione 8.5.1} \textit{Sia S un problema NP-completo. Allora $S
        \in P$ se esolo se $P=N P$.}

\begin{proof}
    È chiaro che, se $P=N P$, allora $S \in P$.\\
    Viceversa, supponiamo $S \in P$, così $S$ ha un algoritmo deterministico di
    decisione che lo accetta in tempo al più polinomiale rispetto alla lunghezza
    dell'input. Per ogni $S^{\prime} \in N P$, dobbiamo trovare un analogo
    algoritmo per $S^{\prime}$. D'altra parte $S^{\prime} \leq_p S$, cioè esiste una
    procedura deterministica $f$ che traduce in tempo polinomiale parole
    dell'alfabeto $A^{\prime}$ di $S^{\prime}$ in parole dell'alfabeto $A$ di
    $S$ in modo tale che le parole di $S^{\prime}$ corrispondono esattamente a
    quelle di $S$. L'algoritmo cercato per $S^{\prime}$ si ottiene allora
    combinando quest'ultima procedura e l'algoritmo per $S$ : per $w \in
        A^{\prime \star}$, si computa $f(w)$ e si controlla se $f(w) \in S$ o no.
    Corrispondentemente si deduce che $w \in S^{\prime}$ o no.
\end{proof}

Così un problema $N P$-completo $S$ può rappresentare adeguatamente rutta la
classe $N P$ a proposito della questione $P=N P$. Ovviamente, se $S \notin  P$,
allora $P \neq N P$. Ma, sorprendentemente, se $S \in P$, allora $P=N P$. Per
usare un famoso riferimento letterario, vale tra i problemi $N P$-completi il
classico patto dei moschettieri di Dumas: "uno per tutti, e tutti per uno".\\
Va però mostrato che problemi NP-completi esistono davvero. Eccone allora un
esempio (assai artificiale e tecnico, ma comunque sufficiente a confermare che
la nozione ha senso).

\paragraph{Proposizione 8.5.2} \textit{Ci sono problemi NP-completi.}

\begin{proof}
    Consideriamo il seguente problema $\mathcal{U}$. Gli input di $\mathcal{U}$
    sono terne costituite da

    \begin{itemize}
        \item una MdT $M$ su $\{0,1\}$,
        \item una parola $w \in\{0,1\}$,
        \item due interi positivi $l, t$
    \end{itemize}
    (tutti proposti come parole di un opportuno alfabeto, per esempio come
    numeri naturali): l'output è $SI$ o $NO$ secondo che esista una parola $y
        \in\{0,1\}$ tale che $l(y) \leq l$ e $M$ converge sull'input $(w, y)$ in al
    più $t$ passi.\\
    La definizione stessa di $N P$ assicura che $\mathcal{U} \in N P$. Resta
    allora da provare che $\mathcal{U}$ è $N P$-arduo. Sia $S^{\prime} \in N P$.
    Senza perdita di generalità possiamo supporre che l'alfabeto di $S^{\prime}$
    sia $\{0,1\}$. Dunque esistono una MdT deterministica $M_S$ su $\{0,1\}$ e
    polinomi $q_S, t_S$ tali che, per ogni $w \in\{0,1\}^{\star}$,
    $$
        w \in S
    $$
    se e solo se
    \begin{center}
        esiste $y \in\{0,1\}^* \text { tale che } l(y) \leq q_S(l(w))$\\
        e $M_S$ accetta $(w, y)$ in al più $t_S(l(w)+l(y))$ passi.
    \end{center}

    Poniamo allora, per $w \in\{0,1\}^*$,
    $$
        f(w)=\left(M_S, w, t_S\left(l(w)+q_S(l(w))\right), q_S(w)\right)
    $$
    È facile controllare che $f$ riduce $S$ ad $\mathcal{U}$, come richiesto. In
    conclusione, $\mathcal{U}$ è NP-completo.

\end{proof}

Quindi ci sono problemi $N P$-completi. Il prossimo paragrafo ce ne darà esempi
meno artefatti e più "concreti".