\section{Un'altra caratterizzazione di NP: le macchine di Turing non deterministiche}

Conosciamo già le macchine di Turing $M$ su un alfabeto $A$. Ricordiamone ancora
la definizione. Una MdT $M$ su $A$ nell'insieme degli stati $Q$ può essere
intesa come una sequenza di istruzioni (dettate dalla funzione di transizione
$\delta$ di $M$ ) che riguardano possibili coppie $(q, a)$ con $q \in Q$ e con
$a$ elemento di $A$, oppure simbolo muto $\star$, e associa a ciascuna di esse
una terna univocamente determinata
$$
    \left(q^{\prime}, a^{\prime}, \pm 1\right)
$$
dove $q^{\prime} \in Q, a^{\prime} \in A \cup\{\star\},-1$ (spostarsi a
sinistra), $+1$ (spostarsi a destra) sono comandi di spostamento.\\ Quel che
conviene nuovamente ribadire è che, ad ogni data coppia $(a, q)$, o non è
associata alcuna terna di istruzioni o, se una terna è associata, essa è unica.
Ammettiamo adesso di far cadere quest'ultima condizione. Otteniamo allora il
concetto di \textit{macchina di Turing non deterministica} $M$ su $A$ nell'insieme degli
stati $Q$, da intendersi come insieme di 5 -uple
$$
    \left(q, a, q^{\prime}, a^{\prime}, \pm 1\right)
$$
$\operatorname{con} q, q^{\prime} \in Q, a, a^{\prime} A \cup\{\star\}, \pm 1$
come sopra. Ne abbiamo già parlato nella prima parte. Sottolineiamo ancora che
la novità è che ad una data coppia $(q, a)$ possono corrispondere più possibili
istruzioni tra le quali scegliere quella da eseguire. In effetti una MdT non
deterministica $M$ su $A$ può talora svolgere sullo stesso input più
computazioni, anche di esito opposto.\\ Ovviamente non è vietato, ed è anzi
ammesso, che la MdT $M$ abbia al più una istruzione su ogni coppia. Così le
"vecchie" macchine di Turing, corrispondenti a questo comportamento, rientrano
tra quelle non deterministiche ora introdotte. Per distinguere il loro preciso
rigore, le chiameremo \textit{macchine di Turing deterministiche}. Occorre però
accogliere con molta attenzione questa terminologia. Le MdT deterministiche non
sono quelle che non sono non deterministiche, formano anzi una sottoclasse della
classe di tutte le macchine di Turing non deterministiche, caratterizzata,
appunto, dalla condizione che ogni possibile coppia $(q, a)$ ammette al più una
terna di istruzioni che la riguardano.\\
È anche da dire che, come già mostrato nella prima parte del libro, ogni MdT non
deterministica $M$ su $A$ si può accompagnare con una MdT $M^{\prime}$
deterministica su $A$ (eventualmente in un maggior numero di stati) con la
seguente proprietà: per ogni $w \in A^{\star}$,

$$
    M^{\prime} \text { converge su } w
$$

se e solo se

\begin{center}
    esiste una computazione convergente di $M$ su $w$.
\end{center}

Sostanzialmente, $M^{\prime}$ segue in parallelo tutte le possibili computazioni
di $M$ su $w$, accompagnandole passo per passo, e si arresta non appena una di
esse si arresta. Scrivere le istruzioni dettagliate di $M^{\prime}$ è esercizio
pesante, ma possibile. Ovviamente i tempi di lavoro di $M^{\prime}$ sono
destinati ad essere più lunghi di quelli di $M$.\\
Possiamo ora dare una nuova caratterizzazione di $N P$, riferita proprio alle
MdT non deterministiche. In questo contesto, la caratterizzazione richiama il
modo in cui $P$ è definito (tramite le MdT deterministiche).

\paragraph{Teorema 8.3.1} \textit{Sia $S$ un insieme di parole su un alfabeto finito A.
    Allora $S \in$ $N P$ se e solo se ci sono una $M d T$ non deterministica $M$ su
    $A$ e un polinomio a coefficienti interi (e valori positivi) $q_S$, tali che, per
    ogni $w \in A^{\star}$,}
\begin{itemize}
    \item \textit{quando $w \in S$, c'è almeno una computazione di $M$ su w che converge
              impiegando al più $q_S(l(w))$ passi;}
    \item \textit{quando $w \notin S$, nessuna computazione di $M$ su w converge.}
\end{itemize}
\begin{proof}
    Sia dapprima $S \in N P$. Allora esistono, in base alla definizione che
    conosciamo, una MdT deterministica $M_S$ su $A$ e due polinomi $p_S, t_S$
    tali che, per ogni $w \in A^{\star}$,
    \begin{itemize}
        \item $w \in S$ se e solo se esiste $y \in A^*$ per cui $l(y) \leq
                  p_S(l(w))$ e $M_S$ converge su $(w, y)$;
        \item su ogni possibile input, $M_S$ diverge oppure converge in tempo al
              più polinomiale limitato da $t_S$ rispetto alla lunghezza dell'input
              stesso.
    \end{itemize}

    Costruiamo allora una MdT non deterministica $M$ su $A$ come segue: per ogni
    $w \in A^*$,
    \begin{itemize}
        \item $M$ scrive prima tutte le possibili stringhe $y \operatorname{con}
                  l(y) \leq p_S(l(w))$,
        \item per ogni $y$ $M$ simula poi $M_S$ su $(w, y)$.
    \end{itemize}

    Il non determinismo deriva dal fatto che $M$ svolge più possibili
    computazioni, una per ogni testimone $y$. Comunque è chiaro che un elemento
    $w \in A^{\star}$ sta in $S$
    se e solo se, per un $y$ opportuno, c'è una computazione di $M$ su $w$ che
    converge, impiegando al più $t_S\left(l(w)+p_S(l(w))\right.$ passi.\\
    Viceversa, ammettiamo che esistano $M, q_S$ come enunciato. Per $w \in
        A^{\star}, M$ opera su $w$ come segue: se $w \in S$, c'è almeno una
    computazione di $M$ su $w$ che converge in al più $q_S(l(w))$ passi;
    altrimenti, per $w \notin S$, ogni computazione di $M$ diverge. Ricordiamo
    poi, dalla prima parte del libro, che ogni computazione di una MdT su un
    assegnato input può essere opportunamente codificata da un intero positivo,
    o anche da una parola $y$ su $A$. Questo si applica alle computazioni di $M$
    su una certa $w$, anche a quelle di lunghezza $\leq q_S(l(w))$. La lunghezza
    $l(y)$ di $y$ può essere poi opportunamente limitata da quella di $w$ e dal
    numero di passi $q_S(l(w))$ della computazione corrispondente. Possiamo
    allora usare le codifiche $y$ delle computazioni di $M$ di lunghezza $\leq
        q_S(l(w))$ su una generica $w$ per organizzare il seguente algoritmo
    deterministico $M_S$. Data $w \in A^{\star}$, consideriamo le coppie $(w,
        y)$ in $A^{\star}$ con $y$ come sopra e le proponiamo come input. La MdT
    $M_S$ si comporta come segue:
    \begin{itemize}
        \item se $M$ converge almeno una volta su $w$ in al più $q_S(l(w))$
              passi, e $y$ è la codifica di una tale computazione, $M_S$ converge su
              $(w, y)$;
        \item altrimenti, $M_S$ diverge su $(w, y)$.
    \end{itemize}
    È esercizio di pazienza scrivere in dettaglio la definizione di $M_S$.
    Risulta comunque chiaro che $w \in S$ se e solo se c'è $y \in A^{\star}$
    tale che $M_S$ converge su $(w, y)$. Inoltre la lunghezza dei testimoni $y$
    dipende polinomialmente, tramite $M$ e $q_S$, da quella di $w$. Infine il
    tempo di lavoro di $M_S$ si può anch'esso limitare polinomialmente rispetto
    a $l(w)$.
\end{proof}

Siccome le MdT deterministiche sono particolari MdT deterministiche, si ha dal
precedente teorema una nuova conferma che $P \subseteq N P$.