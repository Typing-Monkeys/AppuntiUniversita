\subsection{Problema di Matematica}

Apriamo una piccola parentesi, e parliamo ancora dei numeri interi
$0, \pm 1, \pm 2, \pm 3, ...$
e delle operazioni $+, \cdot, -$ che li riguardano. In particolare, costruiamo
attraverso $+, \cdot, -$ polinomi a coefficienti interi, come

\[
    2x + 3, \ x + 2, \ x^{2} + y^{2} + 1, \ x^{2} + y^{2} - 2, \
    x^{2} - 2, \ x^{2} + 1
\]

e così via. Notiamo che alcuni di questi polinomi (il secondo e il quarto, per la
precisione) hanno anche radici intere: ad esempio:

\begin{itemize}
    \item $x + 2 = 0$ è risolto da $x = -2$,
    \item $x^{2} + y^{2} - 2 = 0$ da $x = \pm 1, \ y = \pm 1$.
\end{itemize}

Tuttavia gli altri polinomi della lista, pur avendo coefficienti interi, non hanno
radici intere:

\begin{itemize}
    \item non $2x + 3$, perché $3$ è dispari;
    \item non $x^{2} - 2, \ x^{2} + 1$, perché $2$ e $-1$ non hanno radici quadrate tra gli interi;
    \item non $x^{2} + y^{2} + 1$ per analoghi motivi (per ogni scelta di $x, y$ interi, si ha $x^{2} +
              y^{2} + 1 > x^{2} + y{^2} \geq 0$ dunque $x^{2} + y{^2} + 1$ non si annulla mai tra gli interi).
\end{itemize}

Anzi, proprio per ovviare a queste deficienze si è portati ad allargare l'ambito
numerico degli interi e formare rispettivamente gli insiemi dei razionali (con $- \frac{3}{2}$),
dei reali (con $\sqrt{2}$), dei complessi (con $i = \sqrt{-1}$).
Comunque, rimanendo nell'ambito dei polinomi a coefficienti interi, potrebbe interessarci distinguere
quali tra essi hanno anche radici intere, e quali no. Si tratta di questione tutto men che banale se
è vero, come è vero, che un matematico illustre come David Hilbert la segnalò nel 1900 in una lista di
23 problemi matematici meritevoli a suo avviso di lavoro e di attenzione.
Più precisamente la questione era collocata al posto 10 nella lista.\\
\ \\
\textbf{Decimo problema di Hilbert.} Determinare un procedimento capace di stabilire,
per ogni polinomio a coefficienti interi, se esso ha o no radici intere.

Nel linguaggio odierno ci viene dunque richiesto un algoritmo che abbia

\begin{itemize}
    \item INPUT: un polinomio a coefficienti interi;
    \item OUTPUT: SÌ/NO, secondo che il polinomio abbia o no radici intere.
\end{itemize}

Si noti che il polinomio non ha limitazioni né sul grado (1, o 2, o anche maggiore)
né sul numero delle variabili coinvolte ($x$, eventualmente $y$, ma anche $z, t, ...$).

Il decimo problema di Hilbert, formulato, come detto, nel 1900, non ebbe certo
soluzione nel 1936. Anzi, si dovette attendere il 1970 perché potesse trovare
una risposta per molti versi inattesa e sorprendente. Ma gli sviluppi del 1936
ebbero un ruolo fondamentale in questa soluzione del 1970. Vediamo perché,
e collochiamoci idealmente negli anni '30, tre decenni dopo la proposizione del
quesito di Hilbert.

Di fronte alla difficoltà di determinare l'algoritmo richiesto, si poteva reagire in
due modi:

\begin{itemize}
    \item pensare che i tempi (e le conoscenze matematiche) non erano ancora così maturi
          da permetterne la soluzione, e che ulteriori progressi scientifici erano richiesti;
    \item  dubitare che l'algoritmo potesse davvero trovarsi e che nessuno, nel 1930 e
          negli anni successivi, fosse effettivamente in grado di concepirlo.
\end{itemize}

La prima reazione sembra più ragionevole e realistica; la seconda, più pessimista,
aveva comunque all'epoca fondate motivazioni. Infatti altre questioni di matematica e logica condividevano
la stessa situazione del Decimo Problema di Hilbert: una soluzione che tardava a venire,
ostacolata da formidabili complicazioni tecniche e teoriche. Ad esempio, era questo il caso del fondamentale
problema di logica che lo stesso Hilbert aveva sollevato, chiamandolo Entscheidungsproblem
(problema di decisione). Chi ha dimestichezza con la Logica Matematica ne conosce i termini;
gli altri lettori saranno, se non altro, soddisfatti di sapere che esso chiede di riconoscere quali affermazioni
di un usuale linguaggio logico (quello denominato "del primo ordine") sono da ritenersi valide
(cioè universalmente accettabili) e quali no.
Del resto, a favore della visione più "pessimista" contribuiva in quegli anni '30
la fresca dimostrazione (proprio del 1931) di due famosissimi risultati di
Logica Matematica - i Teoremi di Incompletezza di Kurt Gidel - che stabilivano in qualche modo l'incapacità
umana di cogliere i reali fondamenti dei numeri interi
con le loro usuali operazioni $+, \cdot, -$ e conseguentemente l'impossibilità di delegare completamente la ricerca
matematica alle macchine, secondo le idee accennate nel paragrafo precedente.
Non ci attardiamo comunque su queste pur affascinanti connessioni e torniamo al
nostro problema delle radici intere, e alla possibilità di una risposta negativa del
tipo: "l' algoritmo richiesto non esiste". Ora, per convincere che un algoritmo per
il Decimo problema di Hilbert c'è, basta proporlo: stabilirne le istruzioni e verificarne la correttezza.
Invece, per escluderne l'esistenza, non ci basta formulare
qualche tentativo più o meno maldestro e poi denunciarne il fallimento. Dobbiamo
infatti stabilire una volta per tutte che nessuno, nel 1930 o oggi o in futuro, saprà
mai escogitare un procedimento di calcolo. Ma allora dobbiamo anticipatamente
convenire una qualche definizione generale di

\begin{itemize}
    \item ciò che è computabile (il problema)
\end{itemize}

e, parallelamente, di

\begin{itemize}
    \item chi computa (l'algoritmo che affronta il problema e la macchina che lo svolge),
    \item come si computa (le regole che algoritmo e macchina devono rispettare);
\end{itemize}

dimostrare conseguentemente che il Decimo problema di Hilbert, \\o l'Entscheidungsproblem non rientrano in questa
classe di questioni "calcolabili" (se davvero questa è la loro sorte).
Così la questione si allarga dai contesti particolari
dei numeri interi o delle proposizioni logiche, e va ad introdurre temi basilari di
informatica:

\begin{itemize}
    \item che cosa è un "calcolatore"?
    \item quali sono i problemi che i calcolatori sanno (o non sanno) risolvere?
\end{itemize}

Fu nel 1936 che queste domande ebbero una prima risposta grazie a personaggi illustri degli albori
dell'informatica moderna, come Turing, Church, Kleene, Gòdel e altri.