\section{Caratteristiche di un Algoritmo}

Il grande scrittore di fantascienza Isaac Asimov fissò nei suoi racconti le 3 leggi
fondamentali che dovevano regolare il comportamento dei robot nei loro rapporti
con l'Uomo, ispirandole a principi inderogabili di rispetto, obbedienza e collaborazione.
Il nostro obiettivo qui è per certi versi analogo. Come già detto, vogliamo
infatti descrivere in modo rigoroso come avviene una computazione e precisare:

\begin{itemize}
    \item anzitutto chi computa (l'algoritmo che la svolge e la macchina su cui si svolge),
    \item ma anche e soprattutto come si computa\\
          (le regole che algoritmo e macchina
          devono rispettare nel loro lavoro);
\end{itemize}

formulare in conclusione una specie di decalogo del bravo algoritmo e del bravo
calcolatore. Dunque il nostro contesto richiama in qualche senso le leggi dei robot androidi di Asimov.
Ma per ottenere il nostro obiettivo seguiamo, più che i romanzi di fantascienza, le riflessioni che gli
autorevoli scienziati citati alla fine dello scorso paragrafo, e tra essi segnatamente Alan Turing,
svolsero intorno al 1936.
Notiamo anzitutto che, per calcolare un dato problema, serve anzitutto un alfabeto appropriato con cui formalizzame i
contenuti (una lingua characteristica, per dirla alla Leibniz): una sequenza finita di simboli che descrivano i termini
della questione. Questo alfabeto può essere semplicemente quello della lingua italiana, o magari quello della lingua
scientifica dominante (l'inglese), ma può anche talora allargarsi in ragione del contesto.
Ad esempio, se trattiamo i polinomi a coefficienti interi, come nel caso del Decimo Problema di Hilbert,
è bene che prevediamo anche i simboli $0, 1, 2, 3, 4, 5, 6, 7, 8, 9, +, \cdot, -$. Altri ambiti ci potranno
poi suggerire alfabeti ancor più specialistici e mirati. Concordato dunque il nostro alfabeto,
possiamo formulare in modo adeguato il nostro problema e trasmetterlo all'algoritmo e alla macchina calcolatrice
per la loro computazione.
A questo punto dobbiamo finalmente pensare a concordare con precisione a quali
modelli di algoritmo e di macchina intendiamo ricorrere. In questo possiamo fare
momentaneo riferimento alla nostra esperienza personale e a un'idea intuitiva di
algoritmo e di calcolatore, per immaginare quali regole di buon comportamento ci
aspettiamo che essi rispettino durante il loro lavoro. Semmai, nei casi più delicati
ed incerti, proviamo a riferirci al modo di agire della mente umana come possibile
termine di paragone. Possiamo allora convenire quanto segue.

\begin{enumerate}
    \item Un algoritmo è di lunghezza finita.
    \item Esiste un agente di calcolo (la macchina calcolatrice, appunto) che sviluppa la
          computazione eseguendo le istruzioni dell' algoritmo.
    \item L'agente di calcolo ha a disposizione una memoria, dove vengono immagazzinati i risultati intermedi
          del calcolo.
    \item Il calcolo avviene per passi discreti.
    \item Il calcolo non è probabilistico.
\end{enumerate}

I punti 1-3 hanno un'ovvia interpretazione. Il punto 4 afferma che il calcolo non
avviene mediante dispositivi analogici, ma si svolge ordinatamente, un passo dietro l'altro.
Il punto 5 afferma che il calcolo non è soggetto a fattori casuali, ma si
svolge in modo assolutamente preciso e rigoroso (o anche, come in genere si dice,
"deterministico"). Altre caratteristiche degli algoritmi sono le seguenti.

\begin{enumerate}
    \setcounter{enumi}{5}
    \item Non deve esserci alcun limite finito alla lunghezza dei dati di ingresso.
    \item Allo stesso modo, non deve esserci alcun limite alla quantità di memoria
          disponibile.
\end{enumerate}

Il punto 6 ci dice che l'input del problema può essere arbitrariamente lungo.
La condizione è assolutamente ragionevole: ad esempio, un algoritmo di somma tra
gli interi si deve applicare ad ogni possibile addendo, quindi ad ogni numero intero, comunque grande.
Anche il punto 7 è meritevole di qualche chiarimento.
Asserisce la possibilità di accedere ad una memoria potenzialmente illimitata.
Per sostenere la plausibilità di questa assunzione, proviamo a pensare che cosa accade in caso contrario,
se ammettiamo di limitare preliminarmente ad un livello uniforme prefissato le potenzialità di memoria.
In queste condizioni può capitare che algoritmi anche elementari, costruiti per eseguire semplici computazioni,
si trovino talora nell'impossibilità di completarle. Ad esempio, la funzione che ad
ogni intero associa il suo quadrato $f(x) = x^{2}$ non è più calcolabile, perchè lo
spazio di memoria necessario per computare il quadrato di $x$ dipende ovviamente
da $x$ e dunque, per $x$ molto grande, si trova ad infrangere i limiti eventualmente
prefissati. Così 7 risulta del tutto plausibile.
Anche le seguenti osservazioni sono essenziali per cogliere la natura del calcolo.

\begin{enumerate}
    \setcounter{enumi}{7}
    \item  Deve esserci un limite finito alla complessità delle istruzioni eseguibili dal
          dispositivo.
    \item Il numero di passi della computazione è finito ma non limitato.
    \item Sono ammesse computazioni senza fine.
\end{enumerate}

Il punto 8 ribadisce, insieme al punto 1, l'intrinseca finitezza del dispositivo di calcolo.
Ci deve essere una limitazione finita non solo per il numero delle istruzioni
di un algoritmo (come 1 sostiene), ma anche per la complessità delle singole istruzioni; in altre parole,
solo una porzione finita della memoria dell'agente di calcolo
deve essere occupata da queste istruzioni iniziali al momento in cui la computazione su un dato input si avvia.
La condizione non contraddice la condizione 7,
che riguarda i successivi passi del calcolo e afferma la possibilità di accoglierne
senza limite le informazioni nella restante porzione di memoria. D' altra parte 8
si può intuitivamente giustificare proprio in riferimento alle caratteristiche della
mente umana, considerandone l'intrinseca limitatezza (ma anche le possibilità potenzialmente illimitate di crescita).
Il punto 9, poi, ci dice che non è possibile stabilire a priori un limite massimo per
Il numero dei passi richiesti per eseguire un generico algoritmo su un certo input.
11 punto 10, infine, fa riferimento all'eventualità di dover affrontare problemi senza soluzione,
come quelli accennati nei precedenti paragrafi: in questi casi una
computazione è destinata a prolungarsi indefinitamente senza riuscire a produrre
risposte soddisfacenti.
Queste sono le condizioni che possiamo ragionevolmente richiedere ad un qualunque sistema di computazione
(che includa sia il programma di calcolo che la macchina che lo esegue).
Sono 10, e quindi completano il nostro decalogo nel
senso stretto del termine. Sono anche sufficienti ad individuare il nostro obiettivo. Infatti, proprio a partire da
queste riflessioni, Alan Turing affrontò nel 1936
il problema di definire rigorosamente che cosa possa intendersi per computabile,
proponendo un semplicissimo modello di calcolatore ante litteram (la macchina
di Turing, appunto) e sostenendo che computabile è esattamente quanto una macchina di Turing riesce a computare;
dimostrò conseguentemente l'impossibilità
di risolvere, ad esempio, l'Entscheidungsproblem di Hilbert, perchè non ci sono
macchine di Turing che lo soddisfino e, di conseguenza, neppure algoritmi di alcun genere con questa capacità.
Tra parentesi, potrà essere interessante anticipare che, qualche anno dopo, nel 1970, anche il Decimo Problema di Hilbert
dte. qualche anno dopo, nel 1970, anche il Decimo Problema di Hilbert ottenne la
ottenne la stessa risposta negativa, come racconteremo in maggior dettaglio nelle prossime pagine.
In realtà, nei capitoli che verranno, avremo modo di presentare, descrivere e discutere vari modelli di computazione,
anzitutto quello di Turing, ma anche altri,
classici come quello delle funzioni ricorsive o delle grammatiche, ed altri relativamente più recenti,
collegati ai linguaggi di programmazione. Avremo modo
di confrontare questi approcci e di mostrarne la sostanziale equivalenza. Incontreremo anche
(come già anticipato nelle scorse righe) casi di funzioni che non
si possono calcolare e di problemi che non si possono risolvere (perché estranei
a qualunque trattamento mediante i nostri modelli di calcolo). La teoria della
computabilità si interessa principalmente a tutte queste tematiche; in questo senso precorre ed inaugura,
come già accennato, la moderna Informatica Teorica;
concorre ancora sostanzialmente al suo sviluppo.