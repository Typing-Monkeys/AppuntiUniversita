\section{La Tesi di Church-Turing}

Come già accennato nel Capitolo 1, la macchina di Turing non è l'unico modello
astratto di computabilità finora comparso. Anzi, durante gli anni '30 e all'inizio
degli anni '40, molte importanti caratterizzazioni della calcolabilità hanno visto la
luce: oltre a quella di Turing, anche quella di Kleene, basata sulle equazioni
funzionali, quella di Church, fondata sul $\lambda$-calcolo, ed altre ancora. In ogni caso,
gli autori hanno elaborato modelli elementari di computazione e hanno formulato
l'ipotesi che la loro nozione di calcolabilità fosse la più generale possibile e che
non potessero esistere procedimenti algoritmici di calcolo non esprimibili nel loro
formalismo. Alcuni esempi di questi modelli alternativi saranno presentati nei
capitoli successivi. D'altra parte, tutti i tentativi fatti nell'ambito della logica
matematica e dell'informatica teorica di definire modelli e paradigmi alternativi
alle MdT hanno condotto a caratterizzazioni di insiemi decidibili, semidecidibili e
di funzioni calcolabili equivalenti a quelle di Turing (esempi di queste equivalenze
verranno dimostrati nei prossimi capitoli). Questa circostanza ha condotto alla
formulazione di quella che oggi è nota come \textit{Tesi di Church-Turing} e viene ancora
ritenuta universalmente valida. Questa tesi afferma che ogni procedimento
algoritmico, espresso in un qualunque modello di calcolo, è realizzabile mediante una
macchina di Turing. In realtà il contributo di Alonzo Church alla formulazione di
questa affermazione può sembrare misterioso, e si potrebbe ritenere più consono
chiamarla semplicemente \textit{Tesi di Turing}. Spiegheremo nel Capitolo 4 il ruolo di
Church. In conclusione si ha:

\paragraph{Tesi di Church-Turing.} \textit{Una funzione è calcolabile se e solo se
    esiste una macchina di Turing che la calcola (e cioè se e solo se la funzione è
    calcolabile secondo Turing).}\\

Che ogni funzione calcolata da una MdT sia effettivamente calcolabile è banale. Ciò
che è invece rilevante e problematico è l'implicazione inversa, per la quale ogni
procedimento algoritmico è riconducibile a una MdT. "Algoritmo" e "funzione
calcolabile" sono concetti intuitivi, non specificati in modo formale, per cui non è
possibile una dimostrazione rigorosa di equivalenza con il concetto di macchina di
Turing. La Tesi di Church-Turing non è dunque una congettura che, in linea di
principio, potrebbe un giorno diventare un teorema. Tuttavia, la nozione intuitiva di
funzione calcolabile è contraddistinta da un insieme di caratteristiche (quali
determinismo, finitezza di calcolo, etc.), che possono essere considerate in larga
misura "\textit{oggettive}". Questo fa sì che sia praticamente sempre possibile una
valutazione concorde nel decidere se un dato procedimento di calcolo possa essere
considerato algoritmico o meno. Quindi, almeno in linea di principio, è ammissibile
che venga "scoperto" un controesempio alla Tesi di Church-Turing: che si individui
cioè una funzione che sia effettivamente calcolabile secondo questi parametri
informali, ma che non sia computata da nessuna MdT. Almeno finora, però, nessun
controesempio alla tesi di Church-Turing è stato trovato nonostante gli ovvi
progressi teorici e pratici che l'Informatica ha avuto dagli anni '30. In effetti,
gli argomenti a favore della Tesi di Church-Turing possono essere raccolti in tre
gruppi.

\begin{enumerate}
    \item \textit{Evidenza euristica.}
          \begin{itemize}
              \item Per ogni singola funzione calcolabile che sia stata esaminata, è
                    sempre stato possibile trovare una MdT in grado di computarla. Lo
                    stesso si può dire dei procedimenti che producono effettivamente
                    nuove funzioni a partire da altre. Eppure questa indagine è stata
                    condotta per un gran numero di funzioni, di classi di funzioni e
                    di procedure, con l'intento di renderla la più esaustiva
                    possibile.
              \item I metodi per dimostrare che le funzioni computabili sono
                    calcolabili secondo Turing sono stati sviluppati con un grado di
                    generalità tale da far ritenere improbabile che possa essere
                    scoperta una funzione calcolabile cui l'approccio di Turing non
                    possa essere applicato.
              \item I vari metodi tentati per costruire funzioni effettivamente
                    calcolabili, ma non computabili da MdT, hanno condotto tutti al
                    fallimento, nel senso che le funzioni ottenute si dimostrano a
                    loro volta calcolabili secondo Turing, oppure in realtà non
                    calcolabili.
          \end{itemize}
    \item \textit{Equivalenza delle diverse formulazioni proposte.} Tutti i tentativi
          che sono stati elaborati per caratterizzare in modo rigoroso la classe di
          tutte le funzioni effettivamente calcolabili si sono rivelati tra loro
          equivalenti. Ciò che è particolarmente rilevante è la diversità degli
          strumenti e dei concetti impiegati nelle diverse formulazioni; in molti
          casi esse traggono la loro origine da concetti matematici preesistenti. Che
          punti di vista talmente diversi e vasti convergano concentricamente alla
          stessa conclusione (e siano comunque equivalenti al modello di Turing) può
          essere inteso come forte argomento a sostegno di ognuno di loro e in
          particolare della tesi di Church-Turing.
    \item \textit{L'impiegato diligente.} In realtà, i precedenti ragionamenti si
          applicano indifferentemente a tutti i vari approcci equivalenti alla
          calcolabilità. Ma c'è un'ulteriore riflessione che serve in particolare a
          sostenere l'approccio di 'Turing. In effetti, le caratteristiche che lo
          distinguono dagli altri sono la sua naturalezza, la sua minore astrazione,
          e anzi il suo dichiarato proposito di simulare meccanicamente il
          comportamento di un essere umano che esegue un calcolo (l'impiegato
          diligente). In questo senso il modello di Turing si lascia preferire. Del
          resto, avremo modo nei prossimi capitoli di presentare altri approcci alla
          calcolabilità e di constatare direttamente le loro maggiori complicazioni
          teoriche. Anzi, ci furono matematici illustri, come Kurt Gòdel, che, pur
          perplessi di fronte all'astrazione di questi metodi alternativi, si
          dichiararono tuttavia pienamente convinti e conquistati dalla naturalezza
          del modello di Turing.
\end{enumerate}

Queste argomentazioni hanno indotto la teoria della computabilità a sviluppare la
tendenza di considerare la Tesi di Church-Turing come una sorta di "legge empirica",
piuttosto che come un enunciato a carattere "\textit{logico-formale}". In altre
parole, si è autorizzati a procedere come segue. Se un certo linguaggio ammette un
qualche algoritmo (di qualunque natura) che lo decide o accetta, oppure se una certa
funzione ha un algoritmo che la calcola, allora, sulla base della tesi di
Church-Turing, possiamo assumere che ci sia una macchina di Turing che decide o
accetta quel linguaggio, o calcola quella funzione: non c'è bisogno di descrivere in
dettaglio la MdT, possiamo affidarci all'autorevolezza della tesi e non attardarci in
ulteriori verifiche.\\
La tesi di Church-Turing permette allora, per
ogni algoritmo che sia intuitivamente tale (anche se espresso in maniera informale),
di dedurre l'esistenza di una MdT che lavora in modo equivalente. L'uso della tesi di
Church-Turing consente di esprimere rapidamente risultati che un approccio più
formale costringerebbe a raggiungere più faticosamente. Per esempio, lavoriamo con
numeri naturali $x, y, z$ e con un alfabeto che li rappresenta, e consideriamo la
funzione così definita:

\[
    f(x, y, z) =  \begin{cases}
        1 & \textrm{se l'}x\textrm{-esima macchina } M_x \textrm{ converge in } z \textrm{ passi sull'input } y, \\
        0 & \textrm{altrimenti;}                                                                                 \\
    \end{cases}
\]

sulla base della tesi di Church-Turing, possiamo dire che essa è calcolabile senza
fornire necessariamente una MdT che la computa. È infatti sufficiente fornire un
algoritmo che la calcoli anche ricorrendo a procedure di più alto livello. Per
esempio: si manda in esecuzione $M_x$ sull'input $y$ per $z$ passi di computazione;
a ogni passo si incrementa di 1 un contatore $C$ (che inizialmente viene posto a
0) e si controlla se la macchina di Turing ha terminato la sua esecuzione (cioè
ha raggiunto una configurazione terminale). Se $M_x$ si arresta quando il contatore
non ha ancora raggiunto $z$ allora l'algoritmo dà in output il valore 1. Se invece
il contatore supera $z$ senza che $M_x$ si sia fermata, l'algoritmo termina e
restituisce in output 0.