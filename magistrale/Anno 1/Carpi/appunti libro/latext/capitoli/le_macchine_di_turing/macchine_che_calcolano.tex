\section{Macchine che calcolano}

Un processo di calcolo è finalizzato ad elaborare informazioni.
Può essere semplice quanto una stima della durata di un percorso tra due città,
e complicato quanto ana previsione metereologica.
Lo studio della calcolabilità si prefigge l'obiettivo
di fornire l'intuizione degli elementi e delle proprietà salienti che
caratterizzano il processo.
Tale intuizione può essere usata per predire la complessità della computazione,
per scegliere in modo consapevole un tipo di calcolo piuttosto che un
altro, e per sviluppare strumenti che facilitino il progetto del processo stesso.
Lo studio della computabilità rivela addirittura che ci sono problemi che non
possono avere risposta; inoltre, ci mostra che tra i problemi che invece possono
essere risolti, ce ne sono alcuni che richiedono risorse così ingenti
(ad esempio, un tempo di calcolo dell'ordine del milione di anni) da scoraggiare
ogni tentativo di soluzione pratica. Lo studio della computabilità riesce a
identificare queste situazioni; permette così di determinare con opportuni
strumenti i casi "positivi" in cui la soluzione si ottiene, magari anche a
costi accessibili.
In questa prima parte degli appunti ci concentreremo in realtà solo sul primo aspetto,
ci dedicheremo cioè a distinguere quali problemi si possono risolvere, e quali no.
Presenteremo quattro modelli astratti di calcolo che permettono questa
identificazione: nell'ordine, macchine di Turing, funzioni ricorsive,
grammatiche, prop-ammi. Non faremo invece alcun alcun riferimento al tema
della complessità computazionale, non ci interesseremo dunque dei costi dei
problemi che hanno soluzione: questo aspetto sarà ampiamente trattato nella
seconda parte degli appunti.
In particolare, in questo capitolo verranno inizialmente introdotti le nozioni di
stringa e di linguaggio, e il ruolo che essi hanno nel rappresentare l'informazione.
Verrà poi presentato il formalismo delle Macchine di Turing: dapprima sarà
definita la versione più semplice, si passerà poi a varianti più complesse,
utili per approfondire aspetti particolari del concetto di calcolo.
Verranno poi presentate le nozioni di calcolabilità e di decidibilità
indotte dal modello della macchina di Turing.