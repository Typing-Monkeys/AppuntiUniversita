\subsection{Numeri e coppie}

Visto che stiamo trattando coppie, o terne, o $n$-uple di elementi, varrà la pena
di citare un'osservazione importante sulle coppie di numeri naturali, che ci sarà
presto utile nel seguito. Si tratta di un risultato dimostrato dal matematico
Georg Cantor nella seconda metà dell'Ottocento.

\paragraph{Teorema 2.7.1}
\textit{C'è una corrispondenza biunivoca $r$ tra ${\mathbb{N}}^2$ e $\mathbb{N}$:
    specificamente, per ogni scelta di $n,m$ naturali,}

\[
    r(n,m) = \frac{(n+m)(n+m+1)}{2} + n.
\]
