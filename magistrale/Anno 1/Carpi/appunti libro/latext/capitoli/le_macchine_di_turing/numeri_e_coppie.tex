\section{Numeri e coppie}

Visto che stiamo trattando coppie, o terne, o $n$-uple di elementi, varrà la pena
di citare un'osservazione importante sulle coppie di numeri naturali, che ci sarà
presto utile nel seguito. Si tratta di un risultato dimostrato dal matematico
Georg Cantor nella seconda metà dell'Ottocento.

\paragraph{Teorema 2.7.1}
\textit{C'è una corrispondenza biunivoca $r$ tra ${\mathbb{N}}^2$ e $\mathbb{N}$:
    specificamente, per ogni scelta di $n,m$ naturali,}

\[
    r(n,m) = \frac{(n+m)(n+m+1)}{2} + n.
\]

L'affermazione è per certi versi sorprendente: si può infatti osservare che ogni
naturale $n$ compare infinite volte come ascissa di una coppia $(n, m) \in
    \mathbb{N}^2$, al variare di $m$. Così $\mathbb{N}^2$ "sembra" infinitamente più grande
di $\mathbb{N}$. Eppure è in corrispondenza biunivoca con $N$ tramite $r$ e quindi,
in questo senso, ci sono tante coppie ordinate di naturali quanti naturali. Al di là
della complessità della formula che definisce $r$, cerchiamo di capire l'idea che
Cantor seguì per ottenerla. Supponiamo allora di disporre le coppie ordinate di
numeri naturali in una tabella (infinita):

$$
    \begin{array}{cccccc}
        (0,0)  & (0,1)  &
        (0,2)  & \ldots & (0, n) & \ldots                   \\ (1,0) & (1,1) & (1,2) & \ldots & (1, n) & \ldots
        \\ (2,0) & (2,1) & (2,2) & \ldots & (2, n) & \ldots \\ \vdots & \vdots & \vdots &
        \vdots & \vdots & \vdots                            \\ (n, 0) & (n, 1) & (n, 2) & \ldots & (n, n) & \ldots \\
        \vdots & \vdots & \vdots & \vdots & \vdots & \vdots
    \end{array}
$$
Le righe
corrispondono alle ascisse, le colonne alle ordinate. Per enumerare le coppia di
numeri naturali si percorra la tabella sulle diagonali come indicato nella figura che
segue:

$$
    \begin{array}{cccccc}
        \swarrow & \swarrow & \swarrow & \swarrow & \ldots & \ldots \\
        \swarrow & \swarrow & \swarrow & \ldots   & \ldots & \ldots \\
        \swarrow & \swarrow & \ldots   & \ldots   & \ldots & \ldots \\
        \swarrow & \ldots   & \ldots   & \ldots   & \ldots & \ldots \\
        \vdots   & \vdots   & \vdots   & \vdots   & \vdots & \vdots
    \end{array}
$$

La prima diagonale contiene solo $(O, O)$, la seconda $(0, 1)$ e $(1, 0), ...$,
la $k$-esima contiene $(0, k - 1), (1, k - 2), ..., (k - 2, 1)$ e $(k - 1, 0)$ e
così via. La generica diagonale $k$ contiene $k$ coppie, precisamente quelle in
cui la somma delle coordinate è $k - 1$. In questo modo, le coppie $(n, m)$ di
$\mathbb{N}^2$ risultano enumerate come segue:

$$
    (0, 0), (O, 1), (1, 0), (0, 2), (1, 1), (2, O), (O, 3), (1, 2), (2, 1), (3, 0), \ldots
$$

e così via, dunque \textit{prima} secondo $m + n$ e poi secondo $n$.
Non è difficile controllare che la funzione $r$ associa alle coppie sopra elencate
nell'ordine

$$
    0, 1, 2, 3, 4, 5, 6, 7, 8, 9, \ldots
$$

cioè proprio la sequenza dei naturali. In generale $r$ associa ad ogni coppia $(n, m)$
il numero di passi necessari per raggiungerla meno 1. Dalla costruzione segue allora
che la funzione $r$ è una biezione: ogni coppia $(n, m)$ compare una sola volta
nella tabella e quindi è univocamente determinato il numero di passi necessario
per raggiungerla lungo le varie diagonali. Inoltre $r$ è data effettivamente e
ammette, come ogni biiezione, la sua funzione inversa: una corrispondenza biunivoca
da $\mathbb{N}$ a $\mathbb{N}^2$, ancora effettiva. Sono quindi anche note le
proiezioni di tale inversa su $\mathbb{N}$ cioè le due funzioni (sempre effettive)
$\pi_1 : \mathbb{N} \rightarrow \mathbb{N}$ e $\pi_2 : \mathbb{N} \rightarrow \mathbb{N}$\
tali che, per ogni $t$ naturale, $\pi_1(t)$, $\pi_2(t)$ sono rispettivamente quei
naturali $n, m$ per cui $r(n, m) = t$. A proposito, un insieme che - come $\mathbb{N}^2$ -
è in corrispondenza biunivoca con $\mathbb{N}$ si dice \textit{numerabile}.