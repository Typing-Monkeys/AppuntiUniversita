\section{La Macchina di Turing}

La Macchina di Turing, qui abbreviata MdT, prende il nome dal matematico
inglese Alan Turing, che la introdusse nel 1936 precedendo di almeno un decennio
l'era del computer.
È il primo dei modelli di calcolo che vogliamo trattare.
L'idea di Turing è infatti
quella di immaginare un semplice meccanismo astratto che riassuma e
simuli tutte le potenzialità computazionali dell'uomo comune.
Così Turing prende a esplicito
modello l'"\textit{impiegato diligente}", che svolge con ordine e cura gli
incarichi assegnatigli, ma non fa niente di più:
all'ora stabilita timbra il cartellino e torna a casa.
Vediamo come la MdT traduce questo comportamento.
Da un punto di vista fisico la MdT può essere pensata come composta da una unità
di controllo a stati finiti, un nastro di lunghezza infinita e una testina di
lettura e scrittura, che permette la comunicazione tra controllo e nastro.
Il nastro è infinito e suddiviso in celle (anche chiamate quadri).
Ogni cella è in
grado di memorizzare un simbolo di un certo alfabeto $A = \{a_0, a_1, ..., a_n,\}$,
oppure un simbolo "bianco" $*$ che denota l'assenza di scrittura in una cella.
Il nastro contiene solo un numero finito di simboli di $A$;
tutte le altre celle contengono comunque il simbolo bianco $*$.
La testina di lettura e scrittura permette all'unità di
controllo di leggere e scrivere un simbolo per volta dal nastro,
quindi a ogni istante la testina può indicare una sola cella del nastro
(e leggere o scrivere il simbolo che la riguarda).
L'unità di controllo, oltre ad avere gli organi meccanici per lo
spostamento del nastro e della testina, contiene il programma secondo cui verrà
eseguito il calcolo e mantiene lo stato della macchina.
L'insieme di possibili stati della macchina è un insieme finito
$Q = \{q_0, q_1, ..., q_m\}$.
Se vogliamo riprendere il paragone con l'impiegato diligente, $A$
rappresenta l'insieme di lettere con cui egli legge la domanda di partenza e
svolge i suoi calcoli successivi;
gli stati di $Q$ corrispondono invece alle possibilità che l'impiegato ha
di scrivere la stessa lettera in più modi distinti, sottolineandola, o
evidenziandola, o colorandola per darle maggiore risalto.
La computazione della MdT avviene per passi discreti. Si concorda che, all'avvio
di ogni sua computazione, la macchina si trovi in uno stato iniziale prefissato
$q_0$.
Ad ogni passo l'unità di controllo prende atto dello stato in cui si trova e dal
simbolo contenuto nella cella che la testina indica, e di conseguenza esegue le
operazioni sotto elencate:

\begin{itemize}
    \item rivede il suo stato;
    \item scrive un simbolo nella cella indicata dalla testina, sostituendo il
          simbolo esistente (ricordiamo che tra i simboli ammessi c'è anche $*$);
    \item sposta la testina di una posizione a sinistra o a destra.
\end{itemize}

Il nuovo stato assunto dall'unità di controllo, il simbolo da scrivere sulla cella
indicata dalla testina e lo spostamento della testina a sinistra o a destra sono
determinati dal programma della MdT, che stabilisce il comportamento della macchina
stessa. Il programma di una MdT può quindi essere pensato come un insieme
di quintuple della forma $(q, a, q', a', x)$, dove $q$ indica lo stato dell'unità
di controllo, $a$ il simbolo nella cella indicata dalla testina mentre $q', a' , x$
specificano l'azione che la MdT deve intraprendere:
in dettaglio $q'$ rappresenta il nuovo stato
dell'unità di controllo, $a'$ il simbolo da scrivere nella cella esaminata
dalla testina e $x = \pm 1$ lo spostamento della testina: una posizione a sinistra
se $x = -1$, a destra se $x = +1$. $+1$ sarà talora denotato semplicemente con $1$
nel seguito.
Ovviamente ogni singola MdT si riconosce, non tanto dal suo aspetto esteriore di
nastri, celle e testine, quanto proprio da queste quintuple, e cioè dal programma
che le si richiede di svolgere e dalle istruzioni che esso prevede,
tutte ridotte a
semplici ordini di cambiamento di stato, di scrittura e di spostamento: così Turing
schematizzava il comportamento della mente dell'impiegato diligente,
sintetizzandone le capacità operative minime, e secondo questo modello si
prevede di
riconoscere linguaggi o computare funzioni.
Notiamo poi che un impiegato diligente esegue ordinatamente le sue istruzioni
finché ne ha e purché non ne abbia troppe tra loro in concorrenza. Nel primo caso,
quando gli ordini mancano, l'impiegato si ferma e ritiene concluso il suo lavoro;
nel secondo, invece, resta nell'imbarazzo della scelta: responsabilità che non si
può scaricare sulle sue spalle.
È dunque importante che le istruzioni relative ad una coppia $(q, a)$,
quando esistono, siano univoche. Si può ammettere che manchino, non che si
moltiplichino. Il
passaggio da $(q, a)$ alla corrispondente terna $(q', a', x)$ deve essere
assolutamente deterministico e lontano da ogni ambiguità.
Il programma di una MdT può essere descritto mediante una tabella a righe e
colonne, la cosiddetta matrice funzionale; le righe corrispondono ai possibili
stati $q$ della macchina mentre le colonne corrispondono ai possibili simboli $a$
dell'alfabeto (incluso $*$). All'incrocio di ciascuna coppia $q$ e $a$ si pone
l'istruzione che
deve essere eseguita dalla macchina quando l'unità di controllo si trova nello
stato $q$ e la testina legge il simbolo $a$: la terna che descrive in che stato
entrare, che
cosa scrivere, dove spostarsi. È ammesso che una casella della matrice funzionale
sia bianca. In tal caso, si conviene che la macchina non ha azioni da compiere e
quindi termina il calcolo.
La matrice funzionale della macchina rappresenta una specie di tabella in cui
sono concentrate tutte le istruzioni: l'impiegato diligente esegue il suo lavoro
applicandola pedissequamente.
Ma è tempo di dare una definizione formalmente rigorosa delle MdT, che ne
riassuma i caratteri essenziali, l'alfabeto, gli stati, le istruzioni.
Possiamo allora dire:

\paragraph{Definizione.}
Una Macchina di Turing $M$ è una quadrupla $(Q, A, \delta, q_0)$, dove:

\begin{itemize}
    \item $Q$ è un insieme finito di stati;
    \item $A$ è un alfabeto, cui si aggiunge il simbolo bianco $*$;
    \item $\delta$ è una funzione da $Q \times (A \cup \{*\})$ a
          $Q \times (A \cup \{*\}) \times \{ -1, +1\}$: $\delta$ è chiamata
          \textit{funzione di transizione}, e gli elementi $(q, a, q', a', x)$
          di $\delta$ sono chiamati \textit{regole
              di transizione} o \textit{istruzioni} di $M$;
    \item $q_0 \in Q$ è lo stato iniziale.
\end{itemize}

Come già sottolineato, l'informazione cruciale su una MdT è quella fornita dalla
funzione di transizione $\delta$, quella che descrive il programma delle istruzioni.
È $\delta$ che regola il comportamento di $M$.
Così, se $M$ si trova nello stato $q \in Q$, la
testina legge il simbolo $a \in A \cup \{*\}$ e $\delta(q, a) = (q', a', x)$,
allora $M$ si sposta
nello stato $q'$, il simbolo $a'$ sostituisce $a$ nella cella in esame e la
testina si sposta
di una posizione a sinistra se $x = -1$, o a destra se $x = +1$. Se invece $M$ si
trova nello stato $q \in Q$, la testina legge il simbolo $a \in A \cup \{*\}$ ma
$\delta(q, a)$ non
è definito, allora $M$ si arresta. In questo senso, la matrice sopra descritta
altro
non è che il grafico di $\delta$. Spesso, quando chiaro dal contesto,
identificheremo una
macchina di Turing $M$ con la sua funzione di transizione $\delta$.
L'istruzione di $\delta$ che
genera la terna $(q', a', x)$ da $(q, a)$ si scriverà talora

\[
    (q, a) \xrightarrow{ \ \delta \ }  (q', a', x)
\]

invece che $\delta (q, a) = (q', a' x)$ o $(q, a, q', a', x) \in \delta$;
qualche volta, quando non ci
sono rischi di ambiguità, ometteremo $\delta$ e scriveremo semplicemente

\[
    (q, a) \rightarrow (q', a', x).
\]

Una MdT, così come l'abbiamo appena definita, viene anche detta \textit{deterministica},
a sottolineare che ogni tripla $(q', a', x)$, se esiste, è unica e univocamente
determinata dalla coppia $(q, a)$.
Tra qualche paragrafo incontreremo anche MdT non
deterministiche e ne spiegheremo le differenze rispetto a quanto appena descritto.
Torniamo adesso ad una descrizione informale del modo in cui una MdT esegue le
sue computazioni a partire da un dato input e restituisce il relativo output.
Ci è utile introdurre anche il concetto di configurazione istantanea di una MdT:
una sorta
di "fotografia" che ritrae la macchina ad un dato istante di lavoro,
illustrandone lo
stato, il contenuto del nastro e la posizione della testina.
Poichè il nastro di una MdT è sempre vuoto, salvo che per un numero finito di celle,
è possibile riassumerne il contenuto con una stringa finita di simboli e fornire
la descrizione istantanea di una MdT con una quadrupla del tipo $(\xi, q, a, \eta)$,
dove $q$ rappresenta lo stato della macchina all'istante in considerazione,
$a$ è il simbolo in lettura, $\xi \in (A \cup \{*\})^*$ è la stringa di simboli a
sinistra del simbolo in lettura ed
$\eta \in (A \cup \{*\})^*$ è la stringa di simboli a destra del simbolo in lettura
prima che il nastro diventi definitivamente bianco. Per esempio,
per $A = \{1\}$, $(11, q_0 , 1, \lambda)$ ci
dice che la macchina esamina una cella con $1$ nello stato $q_0$, che il nastro è
bianco
a destra della cella considerata, contiene $11$ e poi diventa bianco a sinistra.
Poniamo adesso:

\paragraph{Definizione.}
Una configurazione istantanea di una Macchina di Turing
$M =(Q, A, \delta, q_0)$ è un elemento dell'insieme:
$(A \in \{*\})^* \times Q \times (A \cup \{*\}) \times (A \cup \{*\})^*$.\\

Riepiloghiamo allora come avviene una computazione di una macchina $M$ su una
stringa di input $w \in A^*$.

\begin{itemize}
    \item Al passo iniziale $0$, $M$ esamina una sequenza $w$ scritta da sinistra
          verso destra sul nastro di input: la testina indica il primo carattere
          $a_0$ di $w$ e lo stato interno della macchina è quello iniziale $q_0$.
          Questa configurazione è detta
          \textit{configurazione iniziale} su $w$.
          Se $w = a_0w'$ , la 4-upla che la rappresenta è
          $(\lambda, q_0, a_0, w')$.
    \item Ammettiamo che al passo $i$ della computazione la macchina si trovi
          nella
          configurazione $C_i = (\xi, q, a, \eta)$.
          In base alla funzione $\delta$, se esiste una regola di transizione
          $(q, a) \xrightarrow{ \ \delta \ }  (q', a', x)$,
          allora la macchina passa nella nuova configurazione $C_{i+1}$ derivata
          da $C_i$ secondo questa istruzione. Se invece $\delta(q, a)$
          non è definita $M$ si arresta.
          Per esempio, se $C_i$ è come sopra
          $(11, q_0, 1, \lambda)$ e
          $(q_0, 1) \xrightarrow{ \ \delta \ }  (q_1, 1, 1)$, si ha
          $C_{i+1} = (111, q_1, *, \lambda)$; se invece
          $(q_0, 1) \xrightarrow{ \ \delta \ }  (q_1, 1, -1)$,
          allora $C{i+1}= (1, q_1, 1, 1)$.
    \item Se $M$ si ferma dopo un numero $n$ finito di passi,
          allora si dice che $M$ \textit{converge}
          sull'input $w$ (in notazione, $M \downarrow w$);
          $C_n$ si chiama \textit{configurazione finale}, la
          sequenza di configurazioni $C_0, C_1, ..., C_n$ è una computazione
          completa (finita)
          di $M$ sull'input $w$ e la stringa contenuta sul nastro di input è l'
          output di tale computazione.
          Per esempio, se $C_n = (\xi, q, a, \eta)$, allora $\xi a \eta$ è l'output.
    \item Se $M$ non si arresta mai, allora si dice che $M$ \textit{diverge}
          sull'input $w$ e si scrive
          $M \uparrow w$.
\end{itemize}

Per formalizzare rigorosamente questo comportamento computazionale possiamo
introdurre una relazione binaria $\vdash_M$ tra le configurazioni di $M$.
Sostanzialmente $\vdash_M$ associa ad ogni configurazione di $M$ quella che la
segue dopo l'esecuzione dell'istruzione di $M$ corrispondente.
Per esempio, se $C$ è, come sopra, $(11, q_0, 1, \lambda)$, $C'$ è
$(111, q_1, *, \lambda)$ e in $M$ si ha l'istruzione
$(q_0, 1) \xrightarrow{ \ \delta \ } (q_1, 1 , 1)$, si
pone $C \vdash_M C'$.
Il lettore potrà scrivere per esercizio, se ha voglia e pazienza,
tutti i dettagli della definizione di $\vdash_M$.
Se poi per una data $C$ non esiste alcuna configurazione $C$ tale che
$C \vdash_M C'$,
allora scriveremo $C \nvdash_M$.\\

Sia \(\vdash^*_M\) la chiusura riflessiva e transitiva di \(\vdash_M\); vale dire,
per \(C, C'\) configurazioni, \(C \vdash^*_M C'\) significa che esistono un
naturale $n$ e configurazioni $C_0, C_1, \ldots$ $C_n$, tali che
$C = C_0 \vdash_M C_1 \vdash_M \ldots \vdash_M C_n = C^{\prime}$.
Chiamiamo poi \textit{computazione} di una MdT $M$ su un input $w$ una sequenza
(eventualmente infinita) di configurazioni

\[
    C_0 \vdash_M C_1 \vdash_M \ldots \vdash_M C_i \vdash_M \ldots
\]

tale che $C_0$ è una configurazione iniziale su $w$. Chiaramente
$M \downarrow w$ se questa computazione è finita

\[
    C_0 \vdash_M C_1 \vdash_M \ldots \vdash_M C_n
\]

e inoltre $C_n \nvdash_M$; $C_n$ è la configurazione finale di $M$ su $w$.
Altrimenti, se $M \uparrow w$, la computazione non ha mai fine.

\paragraph{Esempio.}
Si consideri l'alfabeto $A = \{\star, 1\}$ composto dal solo 1 e dal simbolo
bianco $\star$. Definiamo una MdT $M$ che, presa come input una successione di
1 consecutivi, restituisce come output tale successione aumentata di un elemento.
In dettaglio $M$ dispone di due stati interni, lo stato iniziale $q_0$ e uno stato
di "arresto" $q_1$, e la funzione di transizione $\delta$ è composta dalle
istruzioni

\[
    \left(q_0, 1\right) \rightarrow\left(q_0, 1,+1\right), \quad\left(q_0, \star\right) \rightarrow\left(q_1, 1,+1\right) .
\]

Secondo le convenzioni stabilite, alla partenza $M$ ha lo stato $q_0$ e la testina
indica il primo simbolo a sinistra dell'input.
Fintanto che la testina trova celle segnate con 1, allora, in virtù della prima
regola di transizione, $M$ scrive 1 sulla cella osservata
(cioè il simbolo letto viene lasciato inalterato) e la testina si sposta a
destra di una cella, mantenendo lo stato $q_0$.
Quando la testina arriva ad una cella vuota, $M$ passa alla seconda regola di
transizione, in virtù della quale la macchina segna con 1 la cella osservata,
sposta la testina a destra (la scelta della direzione di spostamento è in questo
caso ininfluente) e assume lo stato $q_1$, per il quale non è definita nessuna
regola di transizione e, quindi, la macchina si arresta.

\paragraph{Esempio.}
Consideriamo adesso una MdT che ha lo stesso alfabeto $A = {1, \star}$
dell'esempio precedente, ancora due stati $q_0, q_1$, ma una sola istruzione
$(q_0, \star) \rightarrow (q_0, \star, +1)$. Ammettiamo che la macchina abbia
come input il nastro bianco, e dunque si trovi a leggere $\star$ nello
stato $q_0$ al momento in cui la computazione si avvia. È facile verificare che
la macchina scivola verso destra continuando a ristampare $\star$ su ogni nuova
cella considerata e a rimanere nello stato $q_0$. Quindi la computazione diverge.\\

In conclusione, la macchina di Turing costituisce un soddisfacente modello di
calcolo, e comunque corrisponde al decalogo fissato nel Capitolo 1. Infatti si ha
quanto segue.

\begin{itemize}
    \item La funzione di transizione di una MdT costituisce un insieme finito di
          istruzioni.
    \item La MdT così descritta è anche l'agente di calcolo che esegue le istruzioni
          di cui al punto 1).
    \item La MdT può utilizzare il nastro per memorizzare i risultati intermedi.
    \item La MdT opera in modo discreto.
    \item La MdT opera in modo deterministico, con istruzioni univocamente
          specificate.
    \item Non esiste nessuna limitazione sulla lunghezza delle stringhe di ingresso,
          in quanto si assume che il nastro è illimitato.
    \item Il nastro della MdT costituisce, appunto, una memoria di capacità non
          limitata.
    \item Le operazioni che la MdT può eseguire sono molto semplici, quindi di
          complessità finita.
    \item Non esiste nessun limite al numero delle istruzioni eseguite durante una
          computazione: infatti, le istruzioni sono di numero finito, ma è ammessa la
          possibilità di usare più volte la stessa istruzione.
    \item Esiste la possibilità di computazioni infinite, come l'esempio precedente
          ci mostra.
\end{itemize}
