\section{Alfabeti, Stringhe e Linguaggi}

L'abilità nel rappresentare l'informazione è cruciale nel processo di calcolo\\
dell'informazione stessa. Le società umane hanno creato linguaggi parlati
per comunicare ad un livello elementare, e hanno sviluppato il metodo della
scrittura per raggiungere un livello ben più sofisticato.
La lingua italiana, ad esempio, nella
sua forma parlata si basa su un insieme finito di suoni elementari,
le parole sono definite in termini di sequenze finite di tali suoni, le frasi vengono derivate da
sequenze finite di parole, i discorsi sono ottenuti da sequenze di frasi, e così via.
L'italiano scritto usa un insieme finito di simboli come insieme di primitive, le
parole sono definite da sequenze finite di simboli, i periodi vengono derivati da
sequenze finite di parole, i paragrafi da sequenze finite di periodi, e così via.
Approcci simili sono stati sviluppati anche in altri contesti; per esempio, abbiamo
già visto nel precedente capitolo che un numero naturale può essere rappresentato
come una sequenza finita di cifre decimali $0, 1, 2, 3, 4, 5, 6, 7, 8, 9$; ma anche
alternativamente in forma binaria cioè come sequenza finita di bit $0, 1$, quando

\[
    0, 1, 2, 3, 4, 5, ... , 10, 11, 12, ...
\]

diventano rispettivamente

\[
    0, 1, 10, 11, 100, 101, ... , 1010, 1011, 1100, ...
\]

Ci sono del resto altre possibili rappresentazioni dei naturali,
rispetto a basi ancora diverse;
ad esempio un naturale si può esprimere in modo molto rozzo con l'uso
del solo simbolo 1, come sequenza finita di 1: in questo caso

\[
    0, 1, 2, 3, ...
\]

diventano
\[
    1, 11, 111, 1111, ...
\]

e così via.\\
Se poi parliamo di polinomi a coefficienti interi, possiamo espandere i simboli
precedenti accogliendo $+, \cdot, -, x, y, z, t, ...$ e costruire l'oggetto della
nostra attenzione - i polinomi, appunto - come sequenze finite di tutti questi
elementi. Ovviamente ci si aspetta che i processi di calcolo trattino
l'informazione anche in ambiti più generali:
infatti essi manipolano non solo interi, ma anche grafi, programmi, e molti altri
tipi di entità. Comunque questi oggetti possono essere tutti
descritti come sequenze finite di lettere su opportuni alfabeti; dunque possiamo
assumere in astratto che i processi di calcolo considerino stringhe di simboli
scelti in un insieme finito e non vuoto (che viene in genere chiamato alfabeto).
Una sequenza finita di simboli di un dato alfabeto $A$ viene chiamata \textit{stringa} o
\textit{parola} su $A$. La stringa, formata dalla sequenza di simboli
$a_1, a_2, ..., a_n$, viene usualmente denotata $a_1a_2...a_n$.
Ammettiamo anche l'eventualità di una stringa
costituita da nessun simbolo, chiamata \textit{stringa vuota} e denotata
dalla lettere greca $\lambda$.

\paragraph{Esempio.} $A = \{a, b, c, ..., z\}$ e $B = \{0, 1, 2, ..., 9\}$ sono alfabeti.
$abb$ è una stringa su $A$, mentre $123$ è una stringa su $B$. $b12$ non è una stringa
su $A$, in quanto contiene simboli non di $A$. Analogamente,
l'allineamento delle cifre decimali di $\pi$: $1415...$
non è una stringa su $B$, perchè non è una sequenza finita.\\

Ribadiamo che un alfabeto $A$ non è mai vuoto: infatti, per costruire stringhe,
dobbiamo avere dei simboli che le compongano. Inoltre si assume di solito $A$ finito.
In realtà, già nei casi del Decimo problema di Hilbert e dei polinomi a
coefficienti interi, si ammettono infinite indeterminate $x, y, z, t,...$
e dunque infiniti simboli;
ma si può facilmente rimediare con qualche artificio, ad esempio adoperando per
le indeterminate due soli simboli $x, |$ e convenendo di rappresentare
$x, y, z, t,...$ come

\[
    x, x|, x||, x|||, ...
\]

cioè come stringhe su $x, |$.\\

La lunghezza di una stringa $\alpha$ - denotata $l(\alpha)$ - è il numero dei
simboli che la compongono: per $\alpha = a_1a_2 \cdots a_n, \ l(\alpha) = n$.

\paragraph{Esempio.}
$aa$ è una stringa di lunghezza $2$; $l(\lambda) = 0$; $l(ab) + l(b) = 3$.\\

Date due stringhe $\alpha, \beta$ se ne può formare un'altra in cui $\alpha$ è
seguita da $\beta$: essa è
denotata $\alpha\beta$ e chiamata la \textit{concatenazione} di $\alpha$ e $\beta$.
La notazione $\alpha^{i}$ è usata per la
stringa ottenuta concatenando $i$ copie della stringa $\alpha$ quando
$i$ è un intero positivo.
Così $\alpha^2 = \alpha\alpha$, $\alpha^3 = \alpha\alpha\alpha$ e via dicendo.
Si intende poi $\alpha^1 = \alpha $ e $\alpha^0 = \lambda$.
La concatenazione si applica ovviamente anche alle lettere di $A$,
viste come parole su $A$, e anzi rappresenta la via
secondo cui esse generano le altre stringhe.

\paragraph{Esempio.}
La concatenazione di $ab$ con $baa$ produce la stringa $abbaa$, quella di $baa$
con $ab$ determina $baaab$, cioè $ba^{3}b$. Le concatenazioni $\lambda\alpha$ e
$\alpha\lambda$, per ogni stringa $\alpha$, producono la stessa stringa $\alpha$.
In particolare, $\lambda\lambda = \lambda$.\\

Diciamo che una stringa $\alpha$ è una \textit{sottostringa}
di $\beta$ se $\beta = \gamma\alpha\rho$, per qualche scelta
di stringhe $\gamma$ e $\rho$. Una sottostringa $\alpha$
di una stringa $\beta$ si chiama \textit{prefisso} di $\beta$ se
$\beta = \alpha\rho$ per qualche stringa $\rho$ (dunque per $\gamma = \lambda$);
$\alpha$ è un \textit{prefisso proprio} se
$\rho \neq \lambda$. Una sottostringa $\alpha$ di una stringa $\beta$
è un \textit{suffisso} di $\beta$ se $\beta = \gamma\alpha$ per
qualche stringa $\gamma$; $\alpha$ è un \textit{suffisso proprio} se $\gamma \neq \lambda$.

\paragraph{Esempio.}
$\lambda, a, b, ab, abb$ sono sottostringhe di $abb$. $\lambda, a, ab$ sono prefissi
propri di $abb$. $\lambda, b, bb$ sono suffissi propri di $abb$.
$abb$ è prefisso e suffisso di $abb$.\\

Se $\alpha = a_1 \cdots a_n$, allora $a_n \cdots a_1$ è chiamata
l'\textit{inversa} di $\alpha$ e viene denotata $\alpha^{rev}$.
L'insieme di tutte le stringhe su un alfabeto $A$ viene indicato $A*$,
mentre $A+$ denota l'insieme $A* - \{\lambda\}$ delle stringhe non vuote su $A$.
Un alfabeto $A$, come insieme finito di simboli, si può sempre ordinare in più
modi possibili. Sarà spesso conveniente nel seguito di questi appunti fissare
un tale ordinamento totale $<$ e, per $A = \{a_1, ...,a_n\}$, assumere ad esempio

\[
    a_1 < a_2 < \cdots < a_n.
\]

In realtà è bene concordare sin d'ora un qualche ordinamento anche per
le parole su $A$. Infatti in Scienza dell'Informazione si affrontano
comunemente tecniche di ricerca sequenziale e binaria,
\textit{insertion sort}, \textit{quick sort} e \textit{merge sort} che
trattano stringhe e si riferiscono a un loro prefissato ordinamento. Una strategia
frequentemente usata è quella \textit{lessicografica}, definita nel modo che segue.

\paragraph{Definizione.}
Sia $A$ un alfabeto (ordinato da qualche relazione $\le$) e siano $\alpha$ e
$\beta$ stringhe in $A*$. Si dice che $\alpha$ è \textit{lessicograficamente minore}
di $\beta$, $\alpha < \beta$, o
equivalentemente $\beta$ \textit{lessicograficamente maggiore} di $\alpha$,
$\beta > \alpha$, se vale uno dei due casi:

\begin{enumerate}
    \item $\alpha$ è prefisso di $\beta$,
    \item $\alpha$ ha un prefisso $\gamma a$,
          $\beta$ ha un prefisso $\gamma b$, per lo stesso $\gamma \in A*$,
          con $a, b \in A$ e $a < b$ in $A$.
\end{enumerate}

\paragraph{Esempi.}
\begin{enumerate}
    \item Se $A$ è l'alfabeto della lingua italiana, ordinato ponendo
          $a < b < c < \cdots$,
          l'ordine lessicografico in $A*$ non è quello comune dei vocabolari;
          infatti $aba < abb$ perché $a < b$, e $abb$ precede $abba$ perché ha lunghezza
          minore, proprio
          come nei dizionari, ma $z$ precede $ab$ perché ha lunghezza minore.
    \item Consideriamo ora l'alfabeto $A = \{0, 1, 2, 3, 4, 5, 6, 7, 8, 9\}$
          dove si fissa, come di consueto, $0 < 1 < 2 < \cdots < 9$.
          L'ordine che ne deriva sui naturali, intesi
          come stringhe in $A*$ è il loro ordine abituale: ad esempio
          \begin{itemize}
              \item $151 < 153$ perché $1 < 3$,
              \item $151 < 1327$ perché $151$ ha lunghezza minore.
          \end{itemize}
\end{enumerate}

Consideriamo ancora i numeri naturali $\mathbb{N}$,
intesi come parole su $A = \{0, 1, 2, ..., 9\}$ rispetto alla loro tradizionale
rappresentazione in base $10$. È ben noto che
$N$ è primo se $N \ge 2$ e gli unici divisori di $N$ sono $1$ e $N$,
e che ogni $N \ge 2$ si decompone in modo unico nel prodotto di fattori primi.
Nascono così due classici problemi: entrambi hanno per input il nostro $N \ge 2$,

\begin{itemize}
    \item il primo chiede se $N$ è primo o no,
    \item il secondo di decomporre $N$ nei suoi fattori primi.
\end{itemize}

Allora la prima domanda intende riconoscere tra le stringhe su $A$
quelle che corrispondono ai numeri primi; la seconda vuole calcolare,
per ogni $N$, quelle stringhe
che rappresentano numeri primi che dividono $N$.
Nel primo caso si vuole individuare un sottoinsieme di $A*$, nel secondo
computare una funzione che da parole
su $A$ genera nuove sequenze di parole su $A$. Sotto questo punto di vista,
il Decimo problema di Hilbert presenta molte analogie col quesito di riconoscere
i primi. È vero che tratta polinomi piuttosto che
numeri, e che adopera tutt'altro criterio di selezione
(l'esistenza di radici intere piuttosto che la primalità);
tuttavia ha ancora a che fare con un alfabeto $A$
(quello dei polinomi a coefficienti interi) e con parole su $A$
(i polinomi, appunto) e vuole scegliere quelle parole che soddisfano il criterio
fissato (l'esistenza di radici intere),
individuare quindi un particolare sottoinsieme di $A*$.
In generale, dato un alfabeto $A$, un sottoinsieme $L$ di $A*$ si chiama \textit{linguaggio
    formale}, o più semplicemente \textit{linguaggio}, o anche \textit{problema}.
La prima notazione fa riferimento alla sua natura di insieme di parole,
come ogni lingua tradizionale; la seconda alla questione computazionale che $L$
ovviamente propone, e cioè
riconoscere le stringhe su $A$ che stanno in $L$ e escludere le altre.

\paragraph{Esempio.}
L'insieme dei numeri primi è un linguaggio su $A = \{0, 1, 2, ..., 9\}$
(e su qualunque alfabeto per i naturali). L'insieme dei polinomi a coefficienti e
radici intere è un linguaggio sull'alfabeto del Decimo problema di Hilbert.\\

Non dimentichiamo che, accanto al "problema" di riconoscere un linguaggio, c'è
anche quello di computare funzioni su stringhe su un alfabeto $A$, come il caso
della decomposizione in fattori primi ci ha sopra illustrato.