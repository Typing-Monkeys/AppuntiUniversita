\section{Codifiche di Stringhe}

Abbiamo sin qui considerato stringhe su alfabeti $A$ arbitrari. In realtà è
possibile ridurre il nostro ambito ai numeri naturali e a uno degli alfabeti che
servono a rappresentarli, come $\{1\}$ o $\{0, 1\}$ o $\{0, 1, 2, \ldots\}$.
L'idea che si può seguire è la stessa in base alla quale

\begin{itemize}
    \item in un teatro ogni poltrona ha un numero di riconoscimento,
    \item o in una biblioteca ogni volume riceve un suo numero di etichetta.
\end{itemize}

I numeri non sono né poltrone né libri, ma servono a identificarli senza ambiguità.
Procedimenti effettivi di codifica relativamente elementari svolgono la stessa
funzione per le stringhe su un qualunque alfabeto $A$. Vediamone alcuni. Per $A$
arbitrario, cerchiamo di definire in modo esplicito una funzione di numerazione $\sharp$
da $A^*$ a $\mathbb{N}$.

\begin{enumerate}
    \item Una prima strategia può fare riferimento alla rappresentazione dei
          naturali in base 10, o 2, o qualunque valore prefissato $>0$. Ammettiamo, ad
          esempio, per semplicità, che l'alfabeto $A$ abbia nove elementi. È immediato
          identificarli con le cifre $1,2, \ldots, 9$ e magari indicarli con
          $a_1, a_2, \ldots, a_9$. Così

          \[
              j \mapsto a_j \text{ per ogni } j=1, \ldots, 9
          \]

          è una corrispondenza biunivoca tra $\{1,2, \ldots, 9\}$ e $A$. A questo
          punto ad ogni parola non vuota $w=a_{j_1} \cdots a_{j_k}$ su $A$ possiamo
          associare $j_1 \cdots j_k$ in base 10, cioè porre

          $$
              \sharp(w)=j_1 \cdot 10^{k-1}+j_2 \cdot 10^{k-2}+\cdots+j_{k-1} \cdot 10^1+j_k \cdot 10^0
          $$

          gli indici dei simboli di $w$ determinano i coefficienti di questa
          rappresentazione, la lunghezza $k$ di $w$ ne regola il numero degli addendi
          ($k$, appunto, da $10^0$ a $10^{k-1}$). Si noti che $\sharp(w) \neq 0$ perché
          $j_1, \ldots, j_k>0$ e $k>0$. Alla parola vuota $\lambda$, che non ha
          simboli e ha lunghezza 0 , associamo allora 0:

          $$
              \sharp(\lambda)=0 .
          $$

          Parole distinte ricevono così in modo effettivo numeri distinti.
          Per esempio

          $$
              \sharp\left(a_1 a_2 a_3\right)=123, \ \ \sharp\left(a_2 a_4\right)=24,
          $$

          e via dicendo. Viceversa 2154 è il numero di $a_2 a_1 a_5 a_4$, mentre
          1004 non corrisponde a nessuna parola perché contiene 0, ma non è 0.
          Quindi $\sharp$ non è suriettiva. Tuttavia per ogni naturale $n$ è
          possibile stabilire in modo effettivo se $n$ è o no nell'immagine
          di $\sharp$, se sì, a quale parola corrisponde.\\
          Il procedimento si estende facilmente al caso in cui l'alfabeto $A$ ha non
          più nove, ma un numero arbitrario $N>0$ di simboli. Si scrivono gli
          elementi di $A$ come $a_1, \ldots, a_N$ e poi si fa riferimento a $N+1>1$
          e alla rappresentazione dei naturali in base $N+1$, al fatto cioè che ogni
          $j$ si scrive in modo unico come

          $$
              j=j_1 \cdot(N+1)^{k-1}+j_2 \cdot(N+1)^{k-2}+\cdots+j_{k-1} \cdot(N+1)^1+j_k \cdot(N+1)^0
          $$

          per $k>0$, $j_1, \ldots, j_k$ naturali opportuni. Si pone allora, per
          $w \in A^*$

          \begin{itemize}
              \item se $w=\lambda$ è vuota, $\sharp(w)=0$,
              \item se $w=a_{j_1} \cdots a_{j_k}$ non è vuota,
          \end{itemize}

          $$
              \sharp(w)=j_1 \cdot(N+1)^{k-1}+j_2 \cdot(N+1)^{k-2}+\cdots+j_{k-1} \cdot(N+1)^1+j_k \cdot(N+1)^0 .
          $$

          Le proprietà osservate per $N=9$ si preservano. In particolare ogni
          parola su $A$ riceve il "suo" numero $\sharp(w)$ e, viceversa, da ogni
          naturale si può recuperare in modo effettivo la parola corrispondente,
          se esiste.
          I pignoli potrebbero semmai obiettare che la numerazione data da
          $\sharp$ ha il difetto di non saper distinguere simboli da parole.
          Per spiegarci meglio, riferiamoci per un attimo alla lingua italiana, dove
          $a$ ha ruolo sia di lettera (vocale) che di parola (preposizione: andare
          "$a$" Roma). Questa duplice veste meriterebbe un duplice numero di codice,
          uno come simbolo e uno come stringa. Ma $\sharp$ le assegna soltanto il valore 1,
          come prima lettera dell'alfabeto.

    \item Cerchiamo di ovviare all'ultima obiezione proponendo un'altra strategia
          di numerazione che fa stavolta ricorso al teorema fondamentale
          dell'Aritmetica, quello secondo cui ogni naturale $\geq 2$ si decompone
          in modo unico nel prodotto di fattori primi: ad esempio
          $12=2^2 \cdot 3$,$15=3 \cdot 5$, e cosi via. Cominciamo allora col
          numerare i simboli dell'alfabeto $A$, cui assegniamo
          valori \textit{dispari} (per motivi che saranno chiari più tardi).
          Definiamo dunque una funzione $\sharp_0$ da $A$ a $\mathbb{N}$ ponendo,
          per $A=\{a_1, \ldots, a_N\}$ e per $j=1, \ldots, N$,

          $$
              \sharp_0\left(a_j\right)=2 \cdot j+1
          $$

          cioè $\sharp_0(a_1)=3, \sharp_0(a_2)=5$ e via dicendo.\\
          Passiamo poi a numerare le stringhe di $A^*$, introducendo la seguente
          funzione $\sharp$ da $A^*$ a $\mathbb{N}$. Sia $w \in A^*$:

          \begin{itemize}
              \item se $w=\lambda$, conveniamo $\sharp(w)=1$,
              \item altrimenti sia $w=a_{j_1} \cdots a_{j_k}$; in riferimento ai valori
                    $2 \cdot j_j+1, \ldots .2 \cdot j_k+1$ già associati da $\sharp_0$
                    a $j_1, \ldots, j_k$, poniamo

                    $$
                        \sharp_1(w)=2^{2 \cdot j_1+1} \cdot 3^{2 \cdot j_2+1} \cdots p_k^{2 \cdot j_k+1}
                    $$

                    dove $p_k$ denota il $k$-mo numero della sequenza dei primi.
          \end{itemize}

          Assegniamo quindi le codifiche (tramite $\sharp_0$) dei simboli di $w$
          come esponenti a $\sharp_1(w)$ nella sua decomposizione in fattori primi.
          Per esempio

          $$
              \sharp_1\left(a_1 a_4 a_2 a_1 a_1\right)=2^3 \cdot 3^9 \cdot 5^5 \cdot 7^3 \cdot 11^3 \text {. }
          $$

          In questo modo, stringhe distinte ricevono in modo effettivo valori
          distinti. Viceversa, per ogni numero naturale $n$, è possibile stabilire
          in modo effettivo se $n$ è o no immagine di $\sharp_1$, se sì, risalire
          alla parola corrispondente: basta fare riferimento alla decomposizione di
          $n$ in fattori primi, oppure constatare che $n=1$. Per esempio, il numero
          dei fattori primi coinvolti rivela la lunghezza della eventuale stringa
          corrispondente. Stavolta, poi, gli elementi $a_j$ di $A$ ricevono due
          numeri distinti

          \begin{itemize}
              \item l'uno $2 \cdot j+1$ come simboli,
              \item l'altro $2^{2 \cdot j+1}$ come parole.
          \end{itemize}

          Né c'è pericolo di confondere questi valori, perché i numeri dei simboli
          sono dispari $\geq 3$, mentre quelli delle parole sono 1 oppure pari,
          perché esplicitamente divisibili per 2.
          Questo secondo procedimento di numerazione fu ideato da Kurt Gödel nella
          sua dimostrazione dei Teoremi di Incompletezza cui si è accennato nel
          primo capitolo.
          Una volta trattati simboli e stringhe, esso permette anche di numerare
          coppie, o terne, o $n$-uple ordinate di stringhe. A questo proposito,
          abbiamo già visto una possibile strategia: tradurre queste sequenze in
          stringhe in un alfabeto più lungo che utilizzi anche il simbolo bianco
          $\star$. Così $\left(w_1, w_2\right)$ diviene
          $w_1 \star w_2$, $\left(w_1, w_2, w_3\right)$ diviene
          $w_1 \star w_2 \star w_3$ e via dicendo; in questo modo
          $\left(w_1, w_2\right)$, $\left(w_1, w_2, w_3\right), \ldots$ vengono a
          condividere il numero che $w_1 \star w_2, w_1 \star w_2 \star w_3, \ldots$
          ricevono in $A \cup\{\star\}$.
          Il procedimento di Gödel consente tuttavia di evitare il ricorso al simbolo
          estraneo $\star$. Infatti, di fronte alla $n$-upla $(w_1, w_2, \ldots, w_n)$
          di stringhe di $A$ (con $n \ge 2$), possiamo

          \begin{itemize}
              \item ricordare che $w_1, w_2, \ldots, w_n$ hanno già un loro numero
                    di codice, $\sharp_1(w_1)$, $\sharp_1\left(w_2\right), \ldots, sharp_1\left(w_n\right)$
                    rispettivamente,
              \item ricorrere allora nuovamente al teorema fondamentale dell'Aritmetica,
              \item associare a $\left(w_1, w_2, \ldots, w_n\right)$ il numero
                    $$
                        \sharp_2\left(w_1, w_2, \ldots, w_n\right)=2^{\sharp_1\left(w_1\right)} \cdot 3^{\sharp_1\left(w_2\right)} \cdots p_k^{\sharp_1\left(w_n\right)} .
                    $$
          \end{itemize}

          Si ottiene così una funzione effettiva $\sharp_2$ dall'insieme delle
          sequenze finite di stringhe su $A$ di ogni possibile lunghezza $n \geq 2$
          a $\mathbb{N}$. Sequenze distinte ricevono da $\sharp_2$ numeri distinti,
          per il teorema fondamentale dell'Aritmetica e l'unicità della
          decomposizione in fattori primi. Viceversa, per ogni naturale $m$, è
          possibile effettivamente riconoscere se $m$ è o no immagine di $\sharp_2$
          di qualche sequenza di stringhe e, se sì, di quale: di nuovo basta
          ricorrere alla decomposizione di $m$ in fattori primi, il numero dei
          fattori primi distinti coinvolti rivela la lunghezza della sequenza, gli
          esponenti di questi fattori dicono le stringhe della sequenza. Di più,
          non c'è pericolo di confondere i valori assegnati alle sequenze di stringhe
          e quelli associati in precedenza a simboli o stringhe, perché questi ultimi
          sono dispari oppure pari con esponenti dispari $\geq 3$ nella loro
          decomposizione in fattori primi, i nuovi sono pari e con esponenti pari.
\end{enumerate}