\section{Un'altra funzione non calcolabile}

Negli scorsi paragrafi
abbiamo presentato casi di funzioni che non sono calcolabile e di linguaggi che
non sono decidibili secondo la definizione di Turing. In genere, questi esempi
sono stati relativamente tecnici e poco naturali, scarsamente legati ai
familiari contesti numerici. Nei prossimi due paragrafi, introduciamo due esempi
più semplici ed accessibili, l'uno di una funzione che non è calcolabile,
l'altro di un insieme che non è decidibile. Cominciamo con il caso della
funzione. Va detto che non è affatto facile immaginare funzioni non calcolabili;
infatti le funzioni che ci vengono in mente sono per lo più calcolabili (e
proprio in quanto tali si affacciano alla nostra percezione). Nel 1962,
comunque, il matematico ungherese Tibor Rado riuscì a ideare un esempio
relativamente semplice ed anche simpatico (per quanto può essere simpatica una
funzione di Matematica). La funzione in questione viene chiamata $\Sigma$ di
Rado, va dall'insieme $\mathbb{N}$ dei naturali a $\mathbb{N}$ stesso ed è
definita nel modo che segue.\\
Sia dato un numero naturale $n$. Dobbiamo costruire
$\Sigma(n)$. Costruiamo allora tutte le macchine di Turing $M=\left(Q, A,
    \delta, q_o\right)$ dove $A$ è un alfabeto con un solo simbolo $A=\{1\}, Q$ si
compone di $n+1$ stati $q_0, q_1, \ldots, q_n$ e, inoltre,

\begin{enumerate}
    \item[(i)]$\delta$ non ha istruzioni sulle coppie $\left(q_n,
        \star\right),\left(q_n, 1\right)$ relative allo stato $q_n$,
    \item[(ii)] $M$ converge sul nastro bianco.
\end{enumerate}

Tra tutte queste MdT
organizziamo il seguente gioco (che ebbe da Rado il nome inglese di "\textit{busy beaver
    competition}", in italiano "\textit{gioco del castoro laborioso}"):
ogni MdT ottiene per punteggio il numero di 1 che riesce a scrivere sull'output della
sua computazione (convergente!) sul nastro bianco. $\Sigma(n)$ è il punteggio
vincente in questo torneo.\\
Notiamo che almeno la macchina vuota (quella priva di
istruzioni) è ammessa a giocare per $\Sigma(n)$ perchè soddisfa ovviamente le due
condizioni $(i)$ e $(i i)$; tra l'altro, essendo priva di istruzioni, si arresta
ancor prima di partire, non riesce a stampare nessun 1 e quindi ottiene punteggio 0.
Così il gioco di $\Sigma(n)$ trova almeno un concorrente. D'altra parte, perchè ci
sia un punteggio vincente e quindi $\Sigma(n)$ sia correttamente definita, è bene
escludere che i concorrenti siano troppi, cioè infiniti (in tal caso potrebbe anche
non esserci un risultato massimo). Ma, per questo, basta osservare che tutte le
macchine di Turing sull'alfabeto $\{1\}$ con $n+1$ stati sono complessivamente un
numero finito, e dunque anche quelle che rispettano (i) e (ii) e sono di conseguenza
ammesse a giocare rientrano nell'ambito finito.\\
Vediamo adesso alcuni possibili valori di $\Sigma$.
\paragraph{Esempio.}
\begin{enumerate}
    \item È facile vedere che $\Sigma(0)=0$. Infatti, per $n=0$, le macchine ammesse
          a giocare hanno un solo stato $q_0$ e nessuna istruzione che lo riguardi,
          dunque si arrestano immediatamente su qualunque input, in particolare su
          quello bianco. In altre parole, l'unica MdT concorrente è quella priva di
          istruzioni ed ottiene punteggio 0.
    \item È relativamente semplice anche controllare che $\Sigma(1)=1$. Infatti
          stavolta le $\operatorname{MdT} M$ concorrenti hanno due stati $q_0$ e
          $q_1$ e mancano di istruzioni per $q_1$. Se però la funzione di transizione
          $\delta$ di $M$ ha un'istruzione del tipo $\delta\left(q_0,
              \star\right)=\left(q_0, \ldots, \ldots\right)$ (e quindi resta nello stato
          $q_0$ quando legge il simbolo bianco nello stato $q_0$), allora $M$,
          indipendentemente da quanto le viene ordinato di scrivere e da dove le
          viene comandato di spostarsi, finisce per divergere sul nastro bianco, e
          dunque è squalificata dal gioco. Ad esempio, se l'istruzione di $\delta$ è
          $\delta\left(q_0, \star\right)=\left(q_0, 1,+1\right), M$ finisce per
          spostarsi indefinitamente verso destra stampando 1 su ogni quadro
          incontrato, e comunque diverge. Lo stesso capita se altra è l'istruzione su
          che cosa scrivere o dove spostarsi, ma si mantiene l'ordine di rimanere
          nello stato $q_0$. Dunque, tra le MdT $M$ partecipanti al gioco soltanto
          due comportamenti sono ammessi: o $\delta$ non ha istruzioni su $\left(q_0,
              \star\right)$ (nel qual caso $M$ converge subito con punteggio 0 ), oppure
          $\delta$ ha un'istruzione $\delta\left(q_0, \star\right)=\left(q_1, \ldots,
              \ldots\right)$, e allora $M$ prima la esegue, stampando $\star$ o
          1 sul nastro, poi entra nello stato $q_1$ e, non avendo
          disposizioni che lo riguardano, si arresta, con punteggio 0 o 1
          rispettivamente. In conclusione, $\Sigma(1)=1$, come detto.
    \item Tuttavia, già il calcolo di $\Sigma(2)$ comincia a complicarsi. In effetti
          si riesce a verificare che $\Sigma(2)=4$, ma si devono affrontare e
          superare seri ostacoli computazionali. Il motivo è semplice. Proviamo
          infatti a contare le MdT $M$ a 3 stati $q_0, q_1, q_2$ che sono prive di
          istruzioni relative a $q_2$ e, in questo senso, sono potenziali concorrenti
          al gioco di $\Sigma(2)$. Esse corrispondono alle funzioni di transizione
          $\delta$ che vanno da coppie ordinate con $q_0$ e $q_1$ (ma non $q_2$ )
          come prima componente e $\star$ o $1$ come seconda e giungono a terne che
          hanno $q_0, q_1$ o $q_2$ come prima componente, ancora $\star$ o 1 come
          seconda e, finalmente, $\pm 1$ come terza. Restano in questo modo coinvolte
          $2 \times 2=4$ coppie e $3 \times 2 \times 2=12$ terne. Le funzioni totali
          che vanno dalle prime alle seconde sono allora $12^4=20.736$ (e ad esse
          andrebbero aggiunte le altre funzioni parziali, quelle il cui dominio
          accoglie solo alcune delle coppie sopra elencate): tante sono,
          indicativamente, le potenziali partecipanti al gioco di $\Sigma(2)$ e per
          ognuna di esse va esaminato il comportamento sul nastro bianco per
          computare $\Sigma(2)$. La loro analisi produce comunque il risultato
          $\Sigma(2)=4$.
    \item Anche i valori di $\Sigma$ per $n = 3$ e $n = 4$ sono conosciuti, ma il loro calrcolo è formidabile.
          $\Sigma(3)=6$ fu provato da
          Lin e Rado nel 1965 , ma richiede l'esame di oltre 16 milioni di MdT (al
          posto delle $20.736$ di $\Sigma(2)$ ). È, invece, un teorema di Brady del
          1983 il fatto che $\Sigma(4)=13$ : stavolta le potenziali concorrenti da
          analizzare sono altre 25 miliardi e stabilirne il punteggio non è affatto
          facile; pur tuttavia, una accurata classificazione dei loro comportamenti
          (cui si conferiscono nomi fantasiosi, come alberi di Natale, o draghi che
          si mangiano la coda) e un largo ricorso all'ausilio dei calcolatori
          permette la conclusione $\Sigma(4)=13$.
    \item Ma, per $n \geq 5$, i valori di
          $\Sigma$ non sono ancora noti. Calcoli di Marxen e Buntrock del 2002
          permettono al massimo di stabilire che $\Sigma(5) \geq 4.098$ e che
          $\Sigma(6)$ supera addirittura $1,29 \cdot 10^{865}$.
\end{enumerate}

Del resto, non c'è da stupirsi di tutte queste complicazioni. Infatti, come
Rado osservò sin dal 1962, $\Sigma$ non è calcolabile, anzi supera
asintoticamente qualunque funzione calcolabile. Ad esempio, provate a
considerare le funzioni $f(n)=2^n$, oppure $2^{2^n}$, o ancora
$2^{2^{2^n}}$, e così via. Un attimo di riflessione assicura che queste
funzioni sono tutte calcolabili, ma crescenti rapidissimamente. Eppure il
Teorema di Rado (che adesso dimostreremo) assicura che, da un certo $n$ in
poi, $\Sigma$ le supera definitivamente. Non c'è allora da stupirsi che
$\Sigma$ finisca fuori da ogni ragionevole controllo umano. Per dimostrare
il Teorema di Rado ci serve prima premettere la semplice osservazione che
anche $\Sigma$ è una funzione crescente, nel senso che, per ogni naturale
$n$,

\[
    \Sigma(n) \leq \Sigma(n+1) .
\]

Infatti tutte le MdT $M$ a $n+1$ stati ammesse a giocare per $\Sigma(n)$
possono essere facilmente adattate per partecipare al torneo di
$\Sigma(n+1)$ : basta aggiungere loro un nuovo stato $q_{n+1}$ senza
istruzioni che lo riguardino. In tal modo $M$ resta sostanzialmente la
stessa e quindi converge ancora sull'input bianco, producendo lo stesso
output di prima; non ha istruzioni sul nuovo stato $q_{n+1}$ (anzi, non ne
ha neppure per $\left.q_n\right)$, quindi gioca anche per $\Sigma(n+1)$
ottenendo lo stesso punteggio che per $\Sigma(n)$. D'altra parte, il torneo
di $\Sigma(n+1)$ ammette anche altre partecipanti assolutamente nuove, che
possono raggiungere anche risultati superiori. Quindi $\Sigma(n) \leq
    \Sigma(n+1)$, come detto.\\
Possiamo adesso provare:

\paragraph{Teorema (Rado) 3.6.1}
\textit{$\Sigma$ non è calcolabile. Anzi, per ogni funzione calcolabile
    $f \text{ da } \mathbb{N} \text{ a } \mathbb{N}$, esiste un naturale $k(f)$ tale che, per ogni
    $k>k(f)$, $\Sigma(k)>f(k)$.}

\begin{proof}
    Rappresentiamo per semplicità i naturali $n$
    sull'alfabeto $\{1\}$, come stringhe di 1: quindi $0,1,2,3, \ldots$ diventano
    rispettivamente $1,11,111,1111$, e così via. In genere ogni $n$ viene espresso da una
    stringa di $n+1$ caratteri uguali a 1. Osserviamo poi che, se $f$ è calcolabile,
    anche la funzione $g$ che ad ogni naturale $n$ associa il valore massimo tra $f(2
        n+1)$ e $f(2 n+2)$ è chiaramente calcolabile. C'è dunque una MdT $M(g)$ su $\{1\}$
    che la computa. Supponiamo che $M(g)$ ammetta $n(g)$ stati. Consideriamo adesso un
    qualunque numero naturale $n$. Possiamo costruire una MdT $M_n$ che, a partire
    dall'input bianco, prima vi stampa $n$ e poi simula $M(g)$, computando quindi in
    conclusione $g(n)$. Le istruzioni per $M_n$ sono relativamente semplici da
    organizzare. Anzitutto dobbiamo consentirle di scrivere $n$, dunque $n+1$ cifre
    consecutive tutte uguali a 1 (dopo di che la macchina deve tornare al simbolo 1 più a
    sinistra per poter simulare $M(g)$). Dobbiamo conseguentemente prevedere per $M_n$
    $n+1$ stati $q_0, \ldots, q_n$ in aggiunta a quelli di $M(g)$ e, per quanto riguarda
    la sua funzione di transizione $\delta_n$, l'istruzione $\delta_n\left(q_i,
        \star\right)=\left(q_{i+1}, 1,-1\right)$ per ogni $i<n$: si ottiene così che $M_n$
    si muova verso sinistra e scriva esattamente $n$ cifre uguali a 1. Aggiungiamo le
    ulteriori istruzioni

    \[
        \delta_n\left(q_n, \star\right)=\left(q_n, 1,+1\right), \ \delta_n\left(q_n, 1\right)=(q, 1,-1)
    \]

    dove $q$ è lo stato di partenza di $M(g)$ : il loro effetto è di far stampare
    l'ultimo 1 a sinistra dei precedenti, e poi di portare $M$ su questo 1 nel primo
    stato di $M(g)$. A questo punto aggiungiamo a $M_n$ tutte le istruzioni di $M(g)$ nei
    relativi $n(g)$ stati, avendo cura di distinguere questi stati da quelli già
    adoperati per scrivere $n$. In questo modo $M_n$ viene ad ammettere complessivamente
    $n(g)+n+1$ stati e a convergere sull'input bianco con output $g(n)$, come richiesto.
    Comunque, se vogliamo coinvolgerla nel gioco del castoro laborioso, dobbiamo
    accrescerla di un ulteriore stato, ovviamente privo di istruzioni, e quindi prevedere
    in totale $n(g)+$ $n+2$ stati per $M_n$; in questo modo, $M_n$ concorre al torneo di
    $\Sigma(n(g)+n+1)$ con punteggio $g(n)+1$ (il numero degli 1 nella sequenza che
    rappresenta $g(n)$ ). Così $g(n)<g(n)+1 \leq \Sigma(n(g)+n+1)$. A questo punto è
    facile concludere: per $n \geq n(g)$, si ha infatti

    \[
        \begin{gathered}
            f(2 n+1) \leq g(n)<\Sigma(n(g)+n+1) \leq \Sigma(2 n+1), \\
            f(2 n+2) \leq g(n)<\Sigma(n(g)+n+1) \leq \Sigma(2 n+2)
        \end{gathered}
    \]

    (in ciascun caso, la prima diseguaglianza segue dalla definizione di $g$, la seconda
    è stata appena provata, l'ultima dipende dal fatto che $\Sigma$ è crescente). La tesi
    è quindi dimostrata: se vogliamo essere scrupolosi e riferirci precisamente
    all'enunciato del teorema, fissiamo $k(f)=2 n(g)$ e, per $k>k(f)$,
    distinguiamo i due casi in cui $k$ è dispari oppure pari (dunque $k=2 n+1$ oppure
    $k=2 n+2$ con $n \geq n(g)$ ) per concludere comunque $\Sigma(k)>f(k)$.
\end{proof}
