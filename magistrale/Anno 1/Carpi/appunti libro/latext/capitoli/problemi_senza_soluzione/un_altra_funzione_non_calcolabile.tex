\section{Un'altra funzione non calcolabile}

Negli scorsi paragrafi
abbiamo presentato casi di funzioni che non sono calcolabile e di linguaggi che
non sono decidibili secondo la definizione di Turing. In genere, questi esempi
sono stati relativamente tecnici e poco naturali, scarsamente legati ai
familiari contesti numerici. Nei prossimi due paragrafi, introduciamo due esempi
più semplici ed accessibili, l'uno di una funzione che non è calcolabile,
l'altro di un insieme che non è decidibile. Cominciamo con il caso della
funzione. Va detto che non è affatto facile immaginare funzioni non calcolabili;
infatti le funzioni che ci vengono in mente sono per lo più calcolabili (e
proprio in quanto tali si affacciano alla nostra percezione). Nel 1962,
comunque, il matematico ungherese Tibor Rado riuscì a ideare un esempio
relativamente semplice ed anche simpatico (per quanto può essere simpatica una
funzione di Matematica). La funzione in questione viene chiamata $\Sigma$ di
Rado, va dall'insieme $\mathbb{N}$ dei naturali a $\mathbb{N}$ stesso ed è
definita nel modo che segue. Sia dato un numero naturale $n$. Dobbiamo costruire
$\Sigma(n)$. Costruiamo allora tutte le macchine di Turing $M=\left(Q, A,
    \delta, q_o\right)$ dove $A$ è un alfabeto con un solo simbolo $A=\{1\}, Q$ si
compone di $n+1$ stati $q_0, q_1, \ldots, q_n$ e, inoltre, (i) $\delta$ non ha
istruzioni sulle coppie $\left(q_n, \star\right),\left(q_n, 1\right)$ relative
allo stato $q_n$, (ii) $M$ converge sul nastro bianco. Tra tutte queste MdT
organizziamo il seguente gioco (che ebbe da Rado il nome inglese di "busy beaver
competition", in italiano "gioco del castoro laborioso"):