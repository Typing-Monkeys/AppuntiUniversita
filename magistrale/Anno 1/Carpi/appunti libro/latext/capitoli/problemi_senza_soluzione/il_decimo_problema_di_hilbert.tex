\section{Il Decimo Problema di Hilbert}

Adesso proponiamo l'esempio di un classico problema di Matematica che non ha
algoritmo di soluzione perchè non è decidibile secondo Turing. Ci riferiamo al Decimo
Problema di Hilbert, già trattato proprio all'inizio del libro. Ricordiamone
l'enunciato.

\paragraph{Decimo Problema di Hilbert (1900).}
Determinare un algoritmo per decidere, per ogni intero positivo $k$ e per ogni
polinomio $p\left(x_1, \ldots, x_k\right)$ con coefficienti interi e $k$
indeterminate $x_1, \ldots, x_k$, se $p\left(x_1, \ldots, x_k\right)$ ha o no radici
intere (se cioè esistono o no interi $z_1, \ldots, z_k$ tali che $p\left(z_1, \ldots,
    z_k\right)=0$).\\

Si noti che il problema non pone limitazioni nè sul numero delle indeterminate nè sul
grado complessivo del polinomio. Nel 1970, Yuri Matijasevic, coronando il lavoro di
Julia Robinson, Martin Davis e altri negli anni precedenti, dimostrò che un algoritmo
così generale non può esistere. Discutiamo brevemente la prova di Matijasevic.\\
Anzitutto si osservi che è sufficiente mostrare che non esiste alcun algoritmo per
decidere, per intero positivo $k$ e per ogni polinomio $p\left(x_1, \ldots,
    x_k\right)$ a coefficienti interi, se $p\left(x_1, \ldots, x_k\right)$ ha o no radici
\textbf{naturali} (e cioè intere non negative). Infatti un classico teorema di Lagrange
assicura che un numero intero $z$ è $\geq 0$ se e solo se si può esprimere in
$\mathbb{Z}$ come somma di quattro quadrati (ad esempio, $4=2^2+0^2+0^2+0^2,
    7=2^2+1^2+1^2+1^2$, e così via). Di conseguenza, un eventuale algoritmo di decisione
per le radice intere ne fornisce uno per $\mathbb{N}$: infatti un arbitrario
polinomio $p\left(x_1, \ldots, x_k\right)$ ammette radici naturali se e solo se il
polinomio

\[
    p\left(x_{1,1}^2+x_{1,2}^2+x_{1,3}^2+x_{1,4}^2, \ldots, x_{k, 1}^2+x_{k, 2}^2+x_{k, 3}^2+x_{k, 4}^2\right)
\]

(in cui ogni vecchia indeterminata $x_i(1 \leq i \leq k)$ è sostituita dalla sua
espressione come somma dei quadrati di 4 nuove indeterminate $x_{i, 1}^2+x_{i,
    2}^2+x_{i, 3}^2+x_{i, 4}^2$, dunque si hanno $4 k$ indeterminate e, ancora,
coefficienti interi) ha radici intere. A questo punto diciamo che un insieme $S$ di
naturali è \textit{Diofanteo} se e solo se esistono un naturale $k$ ed un polinomio $p\left(x,
    x_1, \ldots, x_k\right)$ a coefficienti interi in $k+1$ indeterminate $x, x_1,
    \ldots, x_k$ tali che $S$ coincide con l'insieme di quei naturali $n$ per cui
$p\left(n, x_1, \ldots, x_k\right)$ ha radici naturali. L'aggettivo Diofanteo è un
tributo a Diofanto, matematico alessandrino dell'antichità che si interessò della
soluzione dei polinomi a coefficienti interi.\\
Il risultato chiave della ricerca di
Matijasevic è la seguente, sorprendente caratterizzazione dei sottoinsiemi Diofantei
di $\mathbb{N}$.

\paragraph{Lemma (Matijasevic) 3.7.1} \textit{$S \subseteq \mathbb{N}$ è Diofanteo
    se e solo se è semidecidibile.}\\

Ovviamente, pensiamo qui $S$ come un insieme finito di
stringhe su un qualche alfabeto per i naturali, ad esempio su $\{0,1, \ldots, 9\}$.
Il risultato precedente è la chiave fondamentale di tutta la dimostrazione di
Matijasevic. La difficoltà degli argomenti usati ci impediscono di darne resoconto
qui.
Accettiamolo dunque per buono, e consideriamo l'insieme semidecidibile ma non
decidibile $K \subseteq \mathbb{N}$ incontrato nella trattazione del problema
dell'arresto per le MdT. Per il lemma, $K$ è Diofanteo, dunque esiste un polinomio

\[
    p_K\left(x, x_1, \ldots, x_k\right)
\]

a coefficienti interi tale che $K$ coincide con l'insieme di quei naturali $n$ per i
quali $p_K\left(n, x_1, \ldots, x_k\right)$ ammette radici naturali; il polinomio
$p_K\left(x, x_1, \ldots, x_k\right)$ può essere effettivamente calcolato conoscendo
$K$. A questo punto, se esiste un algoritmo per stabilire se un polinomio a
coefficienti interi ammette o non radici naturali, si può facilmente derivarne un
algoritmo di decisione per $K$ : per ogni $n$ naturale, si costruisce il polinomio
$p_K\left(n, x_1, \ldots, x_k\right)$ nelle indeterminate $x_1, \ldots, x_k$ e si
controlla se $p_K\left(n, x_1, \ldots, x_k\right)$ ha o no radici naturali. Ma allora
$K$ è decidibile, e questo è assurdo.\\
In conclusione si ha:

\paragraph{Teorema (Matjiasevic, 1970) 3.7.2} \textit{Non esiste alcun algoritmo per
    risolvere il Decimo Problema di Hilbert.}\\

Come già accennato, il risultato negativo di Matijasevic si applica a polinomi di
grado e numero di indeterminate arbitrari. Ove si pongano limitazioni sul grado dei
polinomi oppure sul numero delle loro indeterminate, un algoritmo per la esistenza di
radici intere può ben essere trovato. Ad esempio, possiamo attingere dai nostri
ricordi di Matematica elementare e rammentare che, se poniamo $k=1$ e quindi ci
restringiamo ad una sola indeterminata $x_1$, allora c'è un semplice algoritmo per
gestire soddisfacentemente la situazione. Accenniamolo brevemente.\\
Sia dunque

\[
    p\left(x_1\right)=a_0+a_1 x_1+\ldots+a_m x_1^m
\]

il nostro polinomio a coefficienti interi nell'unica indeterminata $x_1$. Se $a_0=0$,
allora $p\left(x_1\right)$ ha la radice 0. Se invece $p\left(x_1\right)=a_0 \neq 0$,
allora $p\left(x_1\right)$ non può avere radici. Assumiamo allora $a_0 \neq 0,
    p\left(x_1\right)$ di grado $m \geq 1$ (e dunque $a_m \neq 0$). Se $z \in
    \mathbb{Z}$ è radice di $p\left(x_1\right)$, si ha

\[
    a_0=-z \cdot\left(a_1+\ldots+a_m z^{m-1}\right),
\]

da cui segue che $z$ divide $a_0$. Perciò le eventuali radici intere di
$p\left(x_1\right)$ si trovano tra i divisori del suo termine noto $a_0$ (che sono al
più un numero finito). Allora un algoritmo in questo caso particolare è il seguente:
determinare i divisori di $a_0$ e poi controllare se tra di essi si trova o no
una radice di $p\left(x_1\right)$. Per polinomi a 2 indeterminate, c'è ancora un
algoritmo abbastanza generale, mentre il problema per 3 indeterminate è ancora
aperto. Algoritmi parziali di soluzione del Decimo Problema di Hilbert si hanno anche
ove si limiti il grado del polinomio (ma non il numero delle indeterminate), almeno
per grado 1 o 2 . Per polinomi di grado 3, il problema è ancora aperto; per polinomi
di grado 4 o maggiore, la soluzione, invece, è negativa come nel caso generale.