\section{Il Problema dell'Arresto}

Esistono funzioni che le macchine di Turing non riescono a calcolare, e
linguaggi che non riescono ad accettare, oppure accettano e non riescono a
decidere. Ne abbiamo già accennato alla fine del paragrafo 3.2.  %% TODO: ricontrollare
Così, se accogliamo la tesi di Church-Turing, queste funzioni e questi
linguaggi non ammettono alcun algoritmo e capaci di trattarli.\\
Il prototipo di queste situazioni è il seguente:

\paragraph{Problema dell'Arresto.} Determinare un algoritmo per stabilire, per
ogni scelta di una MdT $M$ e di un input $y \in \mathbb{N}$, se  $M  \downarrow
    y$ o no.\\

Allora, se prestiamo fede alla tesi di Church-Turing secondo cui
tutto quello che è algoritmicamente risolubile può essere trattato con successo
da MdT, il Problema dell'Arresto ci chiede una MdT che, ricevuti in input $x, y
    \in N$ (e in particolare, tramite $x$,  la MdT $M_x$ ), risponda

\begin{itemize}
    \item \textit{SÌ} se $M \downarrow y$.
    \item \textit{NO} se $M \uparrow y$.
\end{itemize}

decida cioè l'insieme

\[
    K^{\prime} = \{(x, y) \in \mathbb{N}^2 \ : \ M \downarrow y\}.
\]

Notiamo che la macchina cercata non può essere quella universale $\mathcal{U}$
Infatti $\mathcal{U}$, dati  $x,  y$, simula la computazione di $M_x$ su $y$, ma
non può
prevederne l'esito e, se $M_x$ diverge su $y$, è condannata a divergere su $(x,
    y)$,
a proseguire cioè indefinitamente la sua computazione senza potersi accorgere a
nessun passo dell'assenza di una conclusione. Invece la MdT cercata converge per
ogni scelta di $x, y$ e anzi calcola la seguente funzione totale $h^{\prime}$ da
$\mathbb{N}^2$ a $\mathbb{N}$: per $x, y \in N$,

\[
    h^{\prime}(x, y)= \begin{cases}1 & \text { se } \phi_x(y) \downarrow, \\ 0 & \text { se } \phi_x(y) \uparrow .\end{cases}
\]

Ebbene questa macchina non esiste. Anzi, vale un risultato per certi versi più
negativo.

\paragraph{Teorema 3.4.1} \textit{La funzione $h$ tale che, per ogni input $x \in
        \mathbb{N}$,}

\[
    h(x)= \begin{cases}1 & \text { se } \phi_x(x) \downarrow, \\ 0 & \text { se } \phi_x(x) \uparrow\end{cases}
\]

\textit{non è calcolabile da nessuna MdT.}\\

\textit{Dimostrazione}. Supponiamo per assurdo il
contrario, che esista cioè una MdT con questa capacità. Consideriamo la seguente
funzione $k$ per $x \in \mathbb{N}$,

\[
    k(x)= \begin{cases}\uparrow & \text { se } h(x)=1, \\ 0 & \text { se } h(x)=0 .\end{cases}
\]

Dalla calcolabilità della funzione $h$ segue facilmente quella di $k$. Dunque
c'è una MdT che calcola $k$; sia $j \in \mathbb{N}$ il suo codice; in altre
parole sia $\phi_j=k$. Ma allora abbiamo

\[
    \phi_j(j) \downarrow \Longleftrightarrow k(j)=0 \Longleftrightarrow h(j)=0 \Longleftrightarrow \phi_j(j) \uparrow,
\]

il che costituisce una evidente contraddizione. Quindi $h$ non è calcolabile.\\

Il
metodo usato per trovare una contraddizione nella prova del precedente teorema è
un esempio della tecnica della diagonalizzazione già introdotta nel paragrafo
3.1; dall'elenco

\[
    \phi_0, \phi_1, \phi_2, \ldots
\]

delle funzioni calcolabili e dall'apparentemente innocua equivalenza, valida per
ogni naturale $x$,

\[
    k(x) \downarrow \text { se e solo se } \phi_x(x) \uparrow
\]

si deduce, per $k=\phi_j$ e per $x=j$, la contraddizione

\[
    \phi_j(j) \downarrow \text { se e solo se } \phi_j(j) \uparrow
\]

e dunque si conclude che $k$ non può essere calcolabile.

\paragraph{Corollario 3.4.2}
\textit{L'insieme $K=\left\{x \in \mathbb{N}: M_x \downarrow x\right\}$ non è
    decidibile.}\\

\textit{Dimostrazione}. La funzione $h$ del teorema 3.4.1 non è altro che la
funzione caratteristica di $K$.

\paragraph{Corollario 3.4.3} \textit{La funzione $h^{\prime}$ tale
    che, per ogni scelta di $x, y \in \mathbb{N}$,}

\[
    h^{\prime}(x, y)= \begin{cases}1 & \text { se } \phi_x(y) \downarrow, \\ 0 & \text { se } \phi_x(y) \uparrow\end{cases}
\]

\textit{non è calcolabile da nessuna $M d T$.}\\

\textit{Dimostrazione}. Altrimenti si vede
facilmente che anche $h$ è calcolabile; infatti $h(x)=h^{\prime}(x, x)$ per ogni
$x \in \mathbb{N}$

\paragraph*{Corollario 3.4.4} \textit{L'insieme $K^{\prime}=\left\{(x, y) \in
        \mathbb{N}^2: M_x \downarrow y\right\}$ non è decidibile.}\\

\textit{Dimostrazione}. Basta
notare che $h^{\prime}=f_{K^{\prime}}$.\\

La prova di questi corollari illustra il
metodo della riduzione introdotto sempre nel paragrafo 3.1: per esempio
l'incapacità di calcolare $h^{\prime}$ è dedotta dall'analogo risultato per $h$.
Diagonalizzazione e riduzione costituiscono entrambe utili strumenti teorici per
dimostrare l'indecidibilità di linguaggi e l'incomputabilità di funzioni.

La soluzione negativa del problema dell'arresto ha delle importanti conseguenze
anche nell'ambito della teoria della programmazione. In particolare, essa indica
chiaramente che non è possibile costruire un perfetto sistema di "debugging",
che sia in grado di stabilire se un programma termini o meno sull'input dato.
Notiamo poi che, sebbene il problema dell'arresto, nella sua piena generalità,
sia stato dimostrato indecidibile, è tuttavia possibile ottenere risultati
positivi in casi parziali. Per esempio abbiamo già affermato nel capitolo
precedente la calcolabilità della funzione $f$ che a ogni terna $(x, y, z)$ di
naturali associa

\[
    f(x, y, z)= \begin{cases}1 & \text { se } M_x \text { sull'input } y \text { termina in } z \text { passi, } \\ 0 & \text { altrimenti. }\end{cases}
\]

In questo caso, infatti, si limita con $z$ il numero di passi di computazione di
$M_x$, e questa specificazione consente il calcolo effettivo dell'immagine di
$f$ in ogni situazione.\\ Il problema dell' arresto non è l'unico esempio
indecidibile in Informatica Teorica. Altre questioni condividono la sua assenza
di soluzioni algoritmiche. Citiamo qui il caso del \textit{Problema dell'equivalenza tra
    $MdT$} che adesso enunciamo.

\paragraph{Problema dell'Equivalenza tra Macchine di Turing.} Determinare un algoritmo per
decidere, per ogni scelta di due naturali $x, y$ (e delle corrispondenti MdT
$M_x$ e $M_y$ ), se $M_x$ e $M_y$ calcolano o no la stessa funzione, cioè se
$\phi_x=\phi_y$ o no.\\

In altre parole, sulla base della tesi di Church-Turing,
si domanda se $E=\{(x, y) \in$ $\left.\mathbb{N}^2: \phi_x=\phi_y\right\}$ è o
no decidibile. Si intende con $\phi_x=\phi_y$ che $\phi_x$ e $\phi_y$ hanno lo
stesso dominio e agiscono allo stesso modo sui suoi elementi. Ma si prova che il
problema non è decidibile secondo Turing e dunque, per la tesi di Church-Turing,
non è risolubile da nessun modello di calcolo. Un caso particolarmente critico
in questo ambito più "generale" è quello dei linguaggi di programmazione cui
dedicheremo il Capitolo 6. Qui la soluzione negativa del problema
dell'equivalenza implica che non sarà mai possibile decidere se due generici
programmi calcolano la stessa funzione o, in altri termini, se hanno la stessa
semantica. Ciò preclude la speranza di trovare un algoritmo per l'analisi e la
verifica di correttezza automatica dei programmi: le tecniche di testing
rimangono così le uniche strategie per scoprire errori in un dato programma. La
nostra dimostrazione dell'indecidibilità del problema dell'equivalenza per MdT
procede in tre passi.

\begin{itemize}
    \item[a)] Dapprima mostriamo che non è possibile decidere se una funzione
        calcolabile è totale (cioè converge su tutti gli input) o meno.
    \item[b)] Proviamo poi che non è possibile riconoscere quali funzioni
        calcolabili coincidono con l'identità $i d$ di $\mathbb{N}$, cioè con
        la funzione di $\mathbb{N}$ in $\mathbb{N}$ che lascia fisso ogni
        naturale.
    \item[c)] Dimostreremo infine l'indecidibilità dell'equivalenza tra MdT.
\end{itemize}

Si noti che anche i passi a), b) costituiscono per proprio conto risultati di indecidibilità.
Procediamo con la prova di a). Usiamo un argomento di \textit{diagonalizzazione}:

\paragraph{Teorema 3.4.5} \textit{L'insieme}

\[
    T=\left\{x \in \mathbb{N}: \phi_x \text { è totale (cioè } \phi_x \text { converge su ogni input) }\right\}
\]

\textit{non è decidibile.}\\

\textit{Dimostrazione}. Ammettiamo per assurdo che $T$ sia decidibile.
Allora siamo capaci di riconoscere effettivamente nell'enumerazione $\phi_0,
    \phi_1, \phi_2, \ldots$ di tutte le funzioni calcolabili quelle che sono anche
totali. C'è dunque una funzione totale e calcolabile $f$ tale che, per ogni $x
    \in \mathbb{N}$,
\[
    f(x)= \text{indice della } x \text{-ma funzione \textit{totale} nell'enumerazione}.
\]

Così ogni funzione calcolabile totale è della forma $\phi_{f(x)}$ per un unico
naturale $x$. Poniamo adesso

\[
    g(x)=\phi_{f(x)}(x)+1, \text { per ogni } x \in \mathbb{N} .
\]

Allora, per ogni $x$, $g$ calcola l'output $\phi_{f(x)}$ di $M_{f(x)}$ su $x$ e
poi gli aggiunge 1. Ovviamente $g$ è calcolabile totale. Così esiste un unico
$x_0 \in \mathbb{N}$ tale che $g=\phi_{f\left(x_0\right)}$. Ma allora

\[
    \phi_{f\left(x_0\right)}\left(x_0\right)+1=g\left(x_0\right)=\phi_{f\left(x_0\right)}\left(x_0\right),
\]

e questo è assurdo. Segue che $T$ è indecidibile.\\

Passiamo alla prova di b).
Usiamo qui una tecnica di \textit{riduzione} ad a).

\paragraph{Teorema 3.4.6} \textit{L'insieme} $I=\left\{x
    \in \mathbb{N}: \phi_x=i d\right\}$ \textit{non è decidibile.}\\

\textit{Dimostrazione}. Mostriamo
che, se $I$ è decidibile, anche l'insieme $T$ del precedente teorema lo è; così
possiamo concludere che $I$ non è decidibile. Consideriamo la funzione parziale
$f$ da $\mathbb{N}^2$ a $\mathbb{N}$ tale che, per ogni scelta di $x, y$
naturali,

\[
    f(x, y)= \begin{cases}y & \text { se } \phi_x(y) \downarrow, \\ \uparrow & \text { altrimenti. }\end{cases}
\]

Non è difficile vedere che $f$ è calcolabile. Supponiamo di fissare il primo
argomento $x$ della funzione $f$. Otteniamo allora per ogni $x$ una funzione
calcolabile $g_x$ da $\mathbb{N}$ a $\mathbb{N}$ così definita: per ogni $y$,

\[
    g_x(y)= \begin{cases}y & \text { se } \phi_x(y) \downarrow \\ \uparrow & \text { altrimenti. }\end{cases}
\]

Tutte le funzioni $g_x$ sono calcolabili e quindi compaiono nell'enumerazione
delle funzioni calcolabili. Così c'è una funzione calcolabile $r$ da
$\mathbb{N}$ a $\mathbb{N}$ tale che, per ogni $x \in \mathbb{N}$,

\[
    \phi_{r(x)}=g_x .
\]

Osserviamo poi che, per ogni $x$,

\[
    g_x = id \text{ se e solo se } \phi_x \text{ è totale.}
\]

Segue che, per ogni $x, x \in T$ se e solo se $\phi_x$ è totale, cioè se e solo
se $\phi_{r(x)}=i d$ e in conclusione se e solo se $r(x) \in I$. Così un
algoritmo che decide $I$ ne produce uno che decide $T$: basta che, per ogni $x
    \in \mathbb{N}$, prima si calcoli $r(x)$ e poi si applichi a $r(x)$ l'algoritmo
di $I$; se esso ci dice $r(x) \in I$, deduciamo $x \in T$, e viceversa.

\paragraph{Teorema 3.4.7} \textit{L'insieme} $E=\left\{(x, y) \in \mathbb{N}^2:
    \phi_x=\phi_y\right\}$ \textit{non è decidibile.}\\

\textit{Dimostrazione}. Di nuovo usiamo una tecnica di riduzione, stavolta
riferita a b). Osserviamo che la funzione $i d$ è certamente calcolabile e
dunque $i d=\phi_j$ per qualche $j$. Ma allora, per ogni $x$,

\[
    x \in I \text { se e solo se }(x, j) \in E .
\]

Quindi $E$ non è decidibile perché non lo è neppure $I$.