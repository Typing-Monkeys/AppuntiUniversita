\section{Insiemi decidibili e semidecidibili}

Nel Capitolo 2 abbiamo introdotto non solo linguaggi
decidibili, ma anche semidecidibili. Ricordiamo la definizione di entrambi
questi concetti (assumendo semmai per semplicità di trattare linguaggi $L
    \subseteq \mathbb{N}$ ). Allora si ha: - $L$ è decidibile se e solo se c'è una
MdT che sa decidere, per ogni $x \in \mathbb{N}$, se $x \in L$ o no (e dunque,
in senso lato, se e solo se c'è un algoritmo che sa svolgere questo compito), -
$L$ è semidecidibile se e solo se c'è una MdT che lo accetta e cioè converge su
tutti e soli gli elementi di $L$. Cerchiamo di cogliere meglio il senso di
questa seconda nozione e mostriamo: Teorema 3.5.1 Sia $L \subseteq \mathbb{N}$.
Allora sono equivalenti le proposizioni (a) Lè semidecidibile, (b) L è il
dominio di una funzione calcolabile, (c) $L$ è vuoto o c'è una funzione totale
calcolabile $f d a \mathbb{N}$ a $\mathbb{N}$ tale che $L e ̀$ l'immagine di $f$
(e dunque gli elementi di $L$ sono enumerati effettivamente come $f(0), f(1),
    f(2)$ e cosi via).

Dimostrazione. L'equivalenza tra (a) e (b) è diretta conseguenza della
definizione. Se $M=M_x$ è la MdT che accetta $L, L$ è il dominio di $\phi_x$, e
viceversa. Passiamo allora a confrontare (a) e (c).