\section{Insiemi decidibili e semidecidibili}

Nel Capitolo 2 abbiamo introdotto non solo linguaggi
decidibili, ma anche semidecidibili. Ricordiamo la definizione di entrambi
questi concetti (assumendo semmai per semplicità di trattare linguaggi $L
    \subseteq \mathbb{N}$ ). Allora si ha:

\begin{itemize}
    \item $L$ è decidibile se e solo se c'è una MdT che sa decidere, per ogni
          $x \in \mathbb{N}$, se $x \in L$ o no (e dunque, in senso lato, se e
          solo se c'è un algoritmo che sa svolgere questo compito),
    \item $L$ è semidecidibile se e solo se c'è una MdT che lo accetta e cioè
          converge su tutti e soli gli elementi di $L$.
\end{itemize}

Cerchiamo di cogliere meglio il senso di
questa seconda nozione e mostriamo:

\paragraph{Teorema 3.5.1} \textit{Sia $L \subseteq \mathbb{N}$.
    Allora sono equivalenti le proposizioni}

\begin{itemize}
    \item[(a)] \textit{L è semidecidibile,}
    \item[(b)] \textit{L è il dominio di una funzione calcolabile,}
    \item[(c)] \textit{$L$ è vuoto o c'è una funzione totale calcolabile $f$ da $\mathbb{N}$ a
            $\mathbb{N}$ tale che $L$ è l'immagine di $f$ (e dunque gli elementi
            di $L$ sono enumerati effettivamente come $f(0), f(1), f(2)$ e cosi
            via).}
\end{itemize}

\textit{Dimostrazione}. L'equivalenza tra (a) e (b) è diretta conseguenza della
definizione. Se $M=M_x$ è la MdT che accetta $L$, $L$ è il dominio di $\phi_x$, e
viceversa.\\ Passiamo allora a confrontare (a) e (c).
(a) $\Rightarrow$ (c). Supponiamo dapprima $L$ semidecidibile e fissiamo una MdT
$M$ che converge su tutti e soli gli elementi di $L$. Enumeriamo $L$ seguendo le
computazioni di $M$ sui naturali $0,1,2, \ldots, m, \ldots$: non appena ci si
accorge che la computazione di $M$ su uno di questi elementi $m$ ha termine, si
aggiunge $m$ all'elenco di $L$. C'è però un'ovvia obiezione che si può sollevare
per questo procedimento, e cioè come controllare contemporaneamente il lavoro di
$M$ sugli infiniti input $m$, tanto più che ogni computazione richiede i suoi
passi $0,1,2, \ldots, n, \ldots$ e talora, nei casi di divergenza, infiniti
passi. Per esempio, se $M$ diverge su 0 come facciamo ad avviare la computazione
su $1,2, \ldots$ e verificarne l'esito? La situazione richiama però le
difficoltà già affrontate nel Capitolo 2 per costruire una biiezione tra coppie
$(m, n)$ di naturali e naturali, e risolte tramite la sequenza
$(0,0),(0,1),(1,0)$ e così via. Proprio con l'uso della stessa idea si riesce a
gestire il nuovo ambito: si segue il passo 0 di $M$ su 0, poi il passo 1 su 0,
il passo 0 su 1, e via dicendo. In questo modo, se $L \neq \emptyset$, si
generano i suoi elementi

\[
    l_0, l_1, l_2, \ldots, l_j, \ldots
\]

(se $L=\emptyset$, $L$ già soddisfa (c)). La lista così ottenuta è finita per $L$
finito ma, ammettendovi ripetizioni, possiamo supporla infinitamente lunga, cioè
con $j$ che varia tra tutti i naturali. Poniamo adesso, per ogni $j \in
    \mathbb{N}$,
\[
    f(j)=l_j .
\]

Allora $f$ è una funzione totale calcolabile, e la sua immagine coincide proprio
con $L$.\\

(c) $\Rightarrow$ (a). Supponiamo viceversa che ci sia una funzione $f$
con queste proprietà. Costruiamo una MdT $M$ che converge su
tutti e soli gli elementi $x$ di $L$ procedendo come segue. Per ogni $x \in
    \mathbb{N}$, $M$ computa finché necessario $f(0), f(1)$, $f(2), \ldots$ e per
ognuno controlla contestualmente se coincide o no con $x$. Se la verifica è
positiva, $M$ si arresta, se no prosegue il suo lavoro. Grazie a $M$, $L$ è
semidecidibile. Lo stesso accade, ovviamente, se $L=\emptyset$: in tal caso $L$
è accettato da ogni MdT che diverge sempre (il lettore può provare a definirne
una).\\
In conclusione, $L$ (se non vuoto), è semidecidibile se e solo se c'è un
programma (e dunque una $\mathrm{MdT}$ ) che enumera effettivamente i suoi
elementi. Abbiamo così un'ulteriore conferma del seguente fatto, già osservato
nel Capitolo 2.

\paragraph{Osservazione.} Se $L$ è decidibile, allora $L$ è semidecidibile.
Infatti una MdT che sa riconoscere gli elementi di $L$ e scartare gli altri si
adatta facilmente a elencare gli elementi di $L$ (se ve ne sono).\\

Non sembra invece chiaro il contrario, se cioè un insieme semidecidibile non
vuoto sia anche decidibile. Ammettiamo infatti di avere un algoritmo che produce
successivamente tutti gli elementi di $L$. Ogni suo output è ovviamente in $L$.
Tuttavia non c'è verso di riconoscere per suo tramite gli elementi $x$ fuori di
$L$: chi osserva gli elementi prodotti dall'algoritmo, prende atto che, dopo
10, o $10^2$, o $10^n$ passi, $x$ non è ancora presente tra essi, ma non può
dedurre che $x$ non comparirà nei passi successivi.\\
Visto che ci siamo, notiamo anche:

\paragraph{Osservazione.} $L$ è decidibile se e solo se
$\mathbb{N}-L$ lo è. Infatti, per ogni $x \in \mathbb{N}, x \in \mathbb{N}-L$ se
e solo se $x \notin L$. Così una MdT che decide $L$ si può facilmente adattare a
decidere $\mathbb{N}-L$ (e viceversa): basta scambiare le risposte finali, dire
$N O$ (a $x \in \mathbb{N}-L$ ?) quando si diceva $S \dot{I}$ (a $x \in L$ ?), e
viceversa.\\

Di nuovo, non sembra affatto chiaro che altrettanto valga per la
semidecidibilità. Infatti, se $L$ è semidecidibile e dunque sappiamo riconoscere
gli elementi di $L$ come quelli su cui un'opportuna MdT $M$ converge, non è
detto che sappiamo riconoscere per tal via anche gli altri, quetli su cui $M$
diverge. In effetti si ha:

\paragraph{Teorema (Post) 3.5.2} \textit{Sia $L \subseteq \mathbb{N}$.
    Allora $L$ è decidibile se e solo se tanto $L$ quanto $\mathbb{N}-L$ sono
    semidecidibili.}\\

\textit{Dimostrazione}. Sappiamo già che, se $L$ è decidibile, allora $L$ è
semidecidibile. Del resto, se $L$ è decidibile, anche $\mathbb{N}-L$ lo è,
quindi $\mathbb{N}-L$ è semidecidibile. Viceversa, assumiamo $L$, $\mathbb{N}-L$
semidecidibili. Ci sono due MdT $M(L)$ e $M(\mathbb{N}-L)$ che
accettano $L$, $\mathbb{N}-L$ rispettivamente. Per ogni $x \in \mathbb{N}$,
seguiamo le computazioni di $M(L)$ e $M(\mathbb{N}-L)$ su $x$. Una e una sola
delle due converge, perché $x$ è in uno e uno solo degli insiemi $L$,
$\mathbb{N}-L$. Se converge $M(L)$, si dichiara $x \in L ;$ se converge
$M(\mathbb{N}-L)$, si conclude $x \in \mathbb{N}-L$.

Il precedente paragrafo ci ha fornito vari esempi di insiemi decidibili, e tra
questi

\[
    K=\left\{x \in \mathbb{N}: M_x \downarrow x\right\} .
\]

D'altra parte si ha:

\paragraph{Teorema 3.5.3} $\mathrm{K}$ \textit{è semidecidibile. In particolare
    ci sono linguaggi semidecidibili e non decidibili.}\\

\textit{Dimostrazione}. Ecco la descrizione informale di una MdT $M$ che accetta
$K$: per ogni $x \in \mathbb{N}$, $M$ costruisce $M_x$ e ne segue la computazione
su $x$; se e quando $M_x$ converge, anche $M$ converge. Altrimenti, $M$ diverge
insieme a $M_x$.

\paragraph{Corollario 3.5.4} $\mathbb{N}-K$ \textit{non è semidecidibile (e dunque esistono
    linguaggi non semidecidibili).}

\begin{proof}
    Altrimenti dal teorema di Post segue che $K$ è decidibile.
\end{proof}

Il lettore potrà controllare per esercizio
l'eventuale semidecidibilità degli altri insiemi indecidibili incontrati nel
precedente paragrafo

\[
    T=\left\{x \in \mathbb{N}: \phi_x \text { è totale }\right\}, I=\left\{x \in \mathbb{N}: \phi_x=i d\right\}
\]

e dei loro complementari $\mathbb{N}-T$, $\mathbb{N}-I$; potrà anche allargare
l'esame in $\mathbb{N}^2$ a

\[
    K^{\prime}=\left\{(x, y) \in \mathbb{N}^2: M_x \downarrow y\right\}, E=\left\{(x, y) \in \mathbb{N}^2: \phi_x=\phi_y\right\} .
\]

Noi dedichiamo invece la parte finale del paragrafo a un'appendice del problema
dell'arresto. Abbiamo stabilito che la questione di decidere, per ogni MdT $M$ e
per ogni input $x$, se $M \downarrow x$ o no è priva di algoritmi di soluzione.
Si può pensare che, se si fissa $M$ e ci si limita a controllare per quali $x$ $M
    \downarrow x$ e per quali no, la situazione si semplifichi e ammette risposta
positiva. Certe volte è davvero così. Per esempio, se $M$ è una MdT senza
istruzione, allora $M$ converge su ogni input e quindi

\[
    \{x \in \mathbb{N}: M \downarrow x\}
\]

coincide con $\mathbb{N}$ ed è ovviamente decidibile. Ma talora la situazione
resta così complicata come in generale. Infatti si ha:

\paragraph{Corollario 3.5.5} \textit{Ci sono
    $M d T M$ tali che $\{x \in \mathbb{N}: M \downarrow x\}$ è indecidibile.}

\begin{proof}
    Basta sfruttare la semidecidibilità di $K$ e prendere una MdT che
    accetta $K$. Allora $\{x \in \mathbb{N}: M \downarrow x\}=K$ è indecidibile. Lo
    stesso ragionamento si applica, ovviamente, a ogni linguaggio semidecidibile e
    indecidibile.
\end{proof}