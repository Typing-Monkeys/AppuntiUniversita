\section{Diagonalizzare e ridurre}

Nel capitolo precedente abbiamo introdotto il modello di calcolo della macchina
di Turing e abbiamo conseguentemente definito la computabilità secondo Turing
(per linguaggi o per funzioni). Nel presente capitolo approfondiamo questa teoria.
Uno dei suoi aspetti più rilevanti è l'esistenza di problemi non computabili: per la
precisione, emergono dal modello di Turing

\begin{itemize}
    \item funzioni che non si possono calcolare,
    \item linguaggi che non si possono accettare, o che si accettano ma non si
          decidono.
\end{itemize}

Esperimenteremo d'altra parte la possibilità di costruire per ogni alfabeto $A$ una $MdT$
"\textit{universale}", capace di svolgere ogni computazione proponibile su $A$.
In realtà, limiteremo per semplicità la nostra analisi ai numeri naturali, visti come parole su
qualche opportuno alfabeto $\{O, 1, . , 9\}$ o $\{O, 1\}$, o $\{1\}$:
questa assunzione è
implicita in tutto il capitolo, salvo avviso contrario. Del resto, le strategie di
codifica apprese nel Capitolo 2 ci mostrano come ogni alfabeto $A$ si riduca
effettivamente al contesto dei naturali. Ricordiamo poi che le $MdT$ sull'alfabeto
fissato si enumerano (in più possibili modi)

$$
    M_0, M_1, M_2, \ldots, M_n, \ldots \ \ (n \in \mathbb{N}),
$$

così come le funzioni da esse calcolate

$$
    \phi_0, \phi_1, \phi_2, \ldots, \phi_n, \ldots \ \ (n \in \mathbb{N});
$$

($\phi_n$ è in particolare la funzione calcolata da $M_n$). Nel nostro caso, quando
lavoriamo su $\mathbb{N}$, possiamo assumere che ciascuna $\phi_n$ si una funzione,
eventualmente parziale, da $\mathbb{N}$ a $\mathbb{N}$. Tanto l'enumerazione delle
$MdT$ quanto quella conseguente delle funzioni calcolabili sono da ritenersi fissate
per tutto il capitolo. I risultati che otterremo non dipendono comunque dalla loro
scelta. Ricordiamo anche la seguente notazione: per $f$ funzione parziale da
$\mathbb{N}$ a $\mathbb{N}$ e $n$ naturale, $f(n) \downarrow$, significa che $n$ è
nel dominio di $f$, cioè $f(n)$ è definita, $f(n) \uparrow$ il contrario. Il
riferimento ai naturali permette alcune utili semplificazioni del contesto. Per
esempio, per $L \subseteq \mathbb{N}$, e quindi $L$ linguaggio sull'alfabeto fissato, è definita la
\textit{funzione caratteristica} di $L$, cioè la funzione totale $f_L$ da $\mathbb{N}$
a $\mathbb{N}$ tale che, per ogni $x \in \mathbb{N}$,

$$
    f_L(x)= \begin{cases}1 & \text { se } x \in \mathbb{N}, \\ 0 & \text { altrimenti. }\end{cases}
$$

Si ha allora:

\paragraph{Lemma 3.1.1} \textit{Per $L \subseteq \mathbb{N}, L$ è decidibile se e solo se
    $f_L$ è calcolabile.}\\

II risultato si estende facilmente al
caso in cui $L$ è sottoinsieme di $\mathbb{N}^2, \mathbb{N}^3$ e così via, tramite le
biiezioni effettive tra questi insiemi e $\mathbb{N}$.\\
Quanto ai risultati negativi
di incomputabilità sopra menzionati, incontreremo due tecniche particolari che
permettono di provarli:

\begin{enumerate}
    \item La prima si chiama \textit{diagonalizzazione} e riesce a produrre,
          ad esempio, funzioni non calcolabili manipolando opportunamente la lista di tutte le
          funzioni calcolabili;
    \item la seconda prende il nome di \textit{riduzione}: essenzialmente
          deduce l'incapacità di calcolare una funzione $f$ o di decidere un linguaggio $L$
          riferendoli a situazioni negative già note e provando che queste ultime avrebbero
          soluzione positiva se altrettanto valesse per $f$ e $L$.
\end{enumerate}