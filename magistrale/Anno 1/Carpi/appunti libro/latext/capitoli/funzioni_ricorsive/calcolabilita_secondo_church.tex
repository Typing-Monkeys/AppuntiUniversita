\section{Calcolabilità secondo Church}

II nostro proposito in questo capitolo è quello di presentare un approccio alla
computabilità

\begin{itemize}
    \item contemporaneo a quello di Turing,
    \item formalmente diverso da
          quello di Turing, anzi più astratto e matematico,
    \item tuttavia completamente equivalente a quello di Turing.
\end{itemize}


Si basa sulla nozione di \textit{funzione ricorsiva}. Ma
prima di introdurre questo nuovo concetto, inquadriamo nuovamente e brevemente
il nostro obiettivo generale: per un fissato alfabeto finito $A$ ed in
riferimento all' insieme $A^{\star}$ di tutte le stringhe di simboli
$\operatorname{di} A$, intendiamo

\begin{enumerate}
    \item riconoscere quei linguaggi su $A$ che
          ammettono un algoritmo di decisione,
    \item riconoscere quelle funzioni da
          $A^{\star}$ (o da sottoinsiemi delle sue potenze cartesiane) ad $A^{\star}$ che
          ammettono un algoritmo di calcolo.
\end{enumerate}


L'analisi di Turing, sviluppata negli scorsi capitoli, ci ha allora
rispettivamente condotto alle corrispondenti nozioni di:

\begin{enumerate}
    \item linguaggio \textit{decidibile},
    \item funzione \textit{calcolabile}.
\end{enumerate}


Come detto, vogliamo adesso esplorare un
approccio alternativo. Ricordiamo che non è restrittivo per i nostri propositi
supporre di lavorare con numeri naturali, dunque riferirci ad un qualche
alfabeto per i numeri naturali (ad esempio $A=$ $\{0,1,2, \ldots, 9\}$, ma anche
$A=\{0,1\}$ se preferiamo la rappresentazione dei naturali in base 2) ed
assumere conseguentemente, senza perdita di generalità, che $A^{\star}$ coincida
con l'insieme $\mathbb{N}$ stesso. Opportune codifiche permettono di ridurre
qualunque $A$ a questa situazione. Consideriamo allora

\begin{enumerate}
    \item insiemi $L$ di naturali, o di $k$-uple di naturali,
    \item funzioni (eventualmente parziali) da $\mathbb{N}^k$ a $\mathbb{N}$
\end{enumerate}

per qualche intero positivo $k$ (ricordiamo che una funzione parziale ha dominio
contenuto in $\mathbb{N}^k$ ma non necessariamente coincidente con tutto
$\mathbb{N}^*$; in quest'ultimo caso la funzione si dice \textit{totale}; ricordiamo
anche che una funzione -parziale o totale- da $\mathbb{N}^k$ a $\mathbb{N}$ si
dice $k$-aria). In questo ambito vogliamo esaminare nuove possibili strategie
per definire la decidibilità degli insiemi e la calcolabilità delle funzioni.
D'altra parte possiamo ricordare che, per $L \subseteq \mathbb{N}^k$, resta
definita la \textit{funzione caratteristica} $f_L$ di $L$, quella funzione (totale) da
$\mathbb{N}^k$ a $\mathbb{N}$ che ad ogni $x \in \mathbb{N}^k$ associa

\begin{itemize}
    \item 1 se $x \in L$,
    \item 0 altrimenti.
\end{itemize}

Possiamo evidentemente affermare che

\begin{center}
    \textit{esiste un algoritmo
        per decidere, per ogni $x \in \mathbb{N}^k$, se $x \in L$ o no}
\end{center}

se e solo se

\begin{center}
    \textit{esiste un algoritmo per calcolare $f_L(x)$ per ogni $x \in \mathbb{N}^k$.}
\end{center}

Basterà allora per i nostri propositi definire con precisione per quali insiemi
$L \subseteq$ $\mathbb{N}^k$

\begin{center}
    \textit{esiste un algoritmo per calcolare $f_L(x)$ per ogni
        $x \in \mathbb{N}^k$.}
\end{center}

In altre parole, possiamo concentrare il nostro lavoro a
definire in modo appropriato la calcolabilità delle funzioni (anche parziali):
infatti tramite $f_L$ possiamo coinvolgere in questo ambito anche il problema di
decidere $L$. Il nostro obiettivo è dunque quello di precisare per quali
funzioni parziali $k$-arie $f$ esiste un algoritmo per calcolare $f(x)$ per ogni
$x$ nel dominio di $f$. Turing ci ha suggerito una possibile strategia, tramite
le macchine di Turing. Come detto, iniziamo qui a considerarne una alternativa.

