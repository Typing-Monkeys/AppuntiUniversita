\section{Church o Turing ?}

L'approccio alla calcolabilità tramite le funzioni ricorsive pare chiaramente
diverso da quello mediante le macchine di Turing. Il primo è assai più astratto,
l'altro più concreto, ed anzi dichiaratamente ispirato dal tentativo di simulare
il comportamento della mente umana. In questo ambito, anzi, il secondo approccio
risulta più convincente, se è vero, come è vero, che scienziati importanti come
Gödel, assai dubbiosi a proposito della Tesi di Church, si dichiaravano invece
assolutamente persuasi dalla calcolabilità secondo Turing, proprio per questa
sua maggiore concretezza. Possiamo allora chiederci a chi credere, se a Church
oppure a Turing, dunque quale modello di computabilità preferire. In realtà la
scelta è tutto men che drammatica, perchè è possibile dimostrare che
$\mathrm{i}$ due approcci, pur così differenti in natura, sono tuttavia tra loro
equivalenti e dunque ugualmente accettabili (o rifiutabili). Conseguentemente le
due tesi, quella sulle macchine di Turing e quella di Church, sono coincidenti,
tanto che le abbiamo già implicitamente congiunte nel Capitolo 2 quando abbiamo
parlato tranquillamente di \textit{Tesi di Church-Turing}.\\
Non va poi trascurato che
proprio questa equivalenza, e dunque la coincidenza concentrica di punti di
vista tanto diversi verso lo stesso obiettivo, è argomento che corrobora e
sostiene ulteriormente i due approcci alla calcolabilità, tanto quello di Turing
che quello di Church. Ma passiamo alla enunciazione e alla dimostrazione del
teorema di equivalenza.

\paragraph{Teorema 4.4.1} \textit{Sia $f$ una funzione parziale $k$-aria (dai naturali ai naturali).
    Allora f è calcolabile (nel senso di Turing) se e solo se f è parziale
    ricorsiva.}\\

In particolare, quando $f$ è totale, si può dedurre che $f$ è calcolabile
(secondo Turing) se e solo se è ricorsiva. Si può poi dedurre facilmente:

\paragraph{Corollario 4.4.2} \textit{Sia $L \subseteq \mathbb{N}^{k}$. Allora $L$ è
    decidibile (secondo Turing) se e solo se $L$ è ricorsivo.}\\

Basta infatti osservare che un insieme $L$ di $k$-uple di naturali è decidibile
(secondo Turing) se e solo se la sua funzione caratteristica è calcolabile
(secondo Turing). Passiamo ora alla dimostrazione del teorema: ne daremo solo i
cenni principali, evitando di disperderci in troppi dettagli.

\begin{proof}
    Assumiamo dapprima $f$ parziale ricorsiva e verifichiamo che $f$
    è calcolabile secondo Turing. Ricordiamo che le funzioni ricorsive sono quelle
    che si ottengono dalle funzioni iniziali applicando un numero finito di volte
    $\mathrm{i}$ procedimenti di composizione, recursione e minimalizzazione. Ci
    basta allora dimostrare quanto segue:

    \begin{enumerate}
        \item[(i)] le funzioni iniziali sono calcolabili
            con MdT,
        \item[(ii)] funzioni che si ottengono per composizione, oppure per recursione,
            oppure per minimalizzazione da funzioni che sono calcolabili con MdT sono a loro
            volta calcolabili con MdT.
    \end{enumerate}

    Le relative verifiche sono più noiose che difficili. Del resto, $(i)$ è già
    stato trattato nel Capitolo 2 almeno a livello di esercizi e quando abbiamo
    introdotto composizione, recursione e minimalizzazione, abbiamo anche accennato
    ad algoritmi
\end{proof}
