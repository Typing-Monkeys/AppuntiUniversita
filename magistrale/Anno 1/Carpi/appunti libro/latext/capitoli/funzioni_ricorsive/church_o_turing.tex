\section{Church o Turing ?}

L'approccio alla calcolabilità tramite le funzioni ricorsive pare chiaramente
diverso da quello mediante le macchine di Turing. Il primo è assai più astratto,
l'altro più concreto, ed anzi dichiaratamente ispirato dal tentativo di simulare
il comportamento della mente umana. In questo ambito, anzi, il secondo approccio
risulta più convincente, se è vero, come è vero, che scienziati importanti come
Gödel, assai dubbiosi a proposito della Tesi di Church, si dichiaravano invece
assolutamente persuasi dalla calcolabilità secondo Turing, proprio per questa
sua maggiore concretezza. Possiamo allora chiederci a chi credere, se a Church
oppure a Turing, dunque quale modello di computabilità preferire. In realtà la
scelta è tutto men che drammatica, perchè è possibile dimostrare che
$\mathrm{i}$ due approcci, pur così differenti in natura, sono tuttavia tra loro
equivalenti e dunque ugualmente accettabili (o rifiutabili). Conseguentemente le
due tesi, quella sulle macchine di Turing e quella di Church, sono coincidenti,
tanto che le abbiamo già implicitamente congiunte nel Capitolo 2 quando abbiamo
parlato tranquillamente di \textit{Tesi di Church-Turing}.\\
Non va poi trascurato che
proprio questa equivalenza, e dunque la coincidenza concentrica di punti di
vista tanto diversi verso lo stesso obiettivo, è argomento che corrobora e
sostiene ulteriormente i due approcci alla calcolabilità, tanto quello di Turing
che quello di Church. Ma passiamo alla enunciazione e alla dimostrazione del
teorema di equivalenza.

\paragraph{Teorema 4.4.1} \textit{Sia $f$ una funzione parziale $k$-aria (dai naturali ai naturali).
    Allora f è calcolabile (nel senso di Turing) se e solo se f è parziale
    ricorsiva.}\\

In particolare, quando $f$ è totale, si può dedurre che $f$ è calcolabile
(secondo Turing) se e solo se è ricorsiva. Si può poi dedurre facilmente:

\paragraph{Corollario 4.4.2} \textit{Sia $L \subseteq \mathbb{N}^{k}$. Allora $L$ è
    decidibile (secondo Turing) se e solo se $L$ è ricorsivo.}\\

Basta infatti osservare che un insieme $L$ di $k$-uple di naturali è decidibile
(secondo Turing) se e solo se la sua funzione caratteristica è calcolabile
(secondo Turing). Passiamo ora alla dimostrazione del teorema: ne daremo solo i
cenni principali, evitando di disperderci in troppi dettagli.

\begin{proof}
    Assumiamo dapprima $f$ parziale ricorsiva e verifichiamo che $f$
    è calcolabile secondo Turing. Ricordiamo che le funzioni ricorsive sono quelle
    che si ottengono dalle funzioni iniziali applicando un numero finito di volte
    $\mathrm{i}$ procedimenti di composizione, recursione e minimalizzazione. Ci
    basta allora dimostrare quanto segue:

    \begin{enumerate}
        \item[(i)] le funzioni iniziali sono calcolabili
            con MdT,
        \item[(ii)] funzioni che si ottengono per composizione, oppure per recursione,
            oppure per minimalizzazione da funzioni che sono calcolabili con MdT sono a loro
            volta calcolabili con MdT.
    \end{enumerate}

    Le relative verifiche sono più noiose che difficili. Del resto, $(i)$ è già
    stato trattato nel Capitolo 2 almeno a livello di esercizi e quando abbiamo
    introdotto composizione, recursione e minimalizzazione, abbiamo anche accennato
    ad algoritmi capaci di calcolare le funzioni corrispondentemente ottenute. Tutti questi
    procedimenti si possono adattare con un minimo di pazienza al contesto e al
    linguaggio delle MdT.\\

    Passiamo allora a controllare l'implicazione inversa: abbiamo una funzione
    parziale $k$-aria $f$ che è calcolabile secondo Turing e vogliamo mostrare
    che $f$ è parziale ricorsiva. Sia $M$ una MdT che computa $f$. Siano poi
    $q_0, \ldots, q_n$ gli stati di $M$. Possiamo assumere che $M$ si arresti se
    e solo se $M$ entra nello stato $q_n$: se questo non è il caso, ci basta
    aggiungere a $M$ un nuovo stato (senza istruzioni che lo riguardino) e alla
    funzione di transizione di $M$ le istruzioni che conducano ogni
    configurazione di arresto di $M$ al nuovo stato. Definiamo a questo punto
    quattro funzioni totali $(k+1)$-arie

    $$
        q_M, \ \ i_M, \ \ s_M, \ \ d_M
    $$

    tali che, per ogni scelta di $\vec{x} \in \mathbb{N}^k$ e $t \in
        \mathbb{N}$,
    $$
        q_M(\vec{x}, t), \ i_M(\vec{x}, t), \ s_M(\vec{x}, t), \ d_M(\vec{x}, t)
    $$

    codificano rispettivamente

    \begin{itemize}
        \item lo stato di $M$,
        \item che cosa è scritto sulla cella esaminata da $M$,
        \item che cosa è scritto sul nastro a sinistra della cella esaminata da $M$,
        \item che cosa è scritto sul nastro a destra della cella esaminata da $M$
    \end{itemize}

    al passo $t$ della computazione di $M$ sull'input
    corrispondente a $\vec{x}$. Ad esempio, $q_M(\vec{x}, t)$ è rappresentato
    dall'indice 0, o $1, \ldots$, o $n$ dello stato corrispondente;
    $i_M(\vec{x}, t)$ è una cifra di un qualche alfabeto dei naturali (ad
    esempio, 0, o $1, \ldots$, o 9 se adoperiamo la rappresentazione decimale)
    oppure un ulteriore numero (10, se vogliamo) per denotare il carattere
    bianco $\star$; $s_M$ e $d_M$ richiedono invece una definizione più accurata,
    che utilizza in modo opportuno le consuete tecniche di codifica ed in
    particolare il teorema di decomposizione in fattori primi: ne omettiamo qui
    i dettagli. Osserviamo poi che, se per un certo input $\vec{x}$  $M$ converge
    in $T$ passi, allora per ogni naturale $t>T$ si conviene
    $$
        \begin{aligned}
            q_M(\vec{x}, t) & =q_M(\vec{x}, T), \ i_M(\vec{x}, t)=i_M(\vec{x}, T), \\
            d_M(\vec{x}, t) & =d_M(\vec{x}, T), \ s_M(\vec{x}, t)=s_M(\vec{x}, T)
        \end{aligned}
    $$
    si cristallizza cioè la situazione al momento dell'arresto di $M$. In questo
    modo si ottiene che le nostre quattro funzioni diventino totali. A questo
    punto si prova che $q_M, i_M, x_M, d_M$ sono ricorsive. Ad esempio si
    osserva che, per ogni $\vec{x} \in \mathbb{N}^k$,
    $$
        q_M(\vec{x}, 0)=0,
    $$
    perché al passo iniziale $M$ è nello stato $q_0$, e
    $$
        i_M(\vec{x}, 0)
    $$

    è la prima cifra della prima componente $x_1$ di $\vec{x}$ (quella più a
    sinistra quando inizia la computazione), mentre, per ogni naturale $t$,
    $$
        q_M(\vec{x}, t+1), i_M(\vec{x}, t+1)
    $$
    si desumono in modo effettivo dalle istruzioni della funzione di transizione di
    $M$ sulla coppia
    $$
        \left(q_M(\vec{x}, t), i_M(\vec{x}, t)\right)
    $$
    ed inoltre da $s_M(\vec{x}, t)$ e $d_M(\vec{x}, t)$. Analogamente per
    $s_M(\vec{x}, t+1), d_M(\vec{x}, t+1)$. In questo modo, con qualche maggior
    attenzione per i dettagli, si mostra che le nostre quattro funzioni sono
    ricorsive. A questo punto si nota che, per ogni $\vec{x} \in \mathbb{N}^k$,

    \begin{itemize}
        \item $\vec{x}$ è nel dominio di $f$ se e solo se $M$ converge su $\vec{x}$, e quindi
              se e solo se, per qualche naturale $t, M$ entra nello stato $q_n$ al passo $t$
              della computazione su $\vec{x}$, ovvero $q_M(\vec{x}, t)=n$;
        \item in tal caso,
              $f(\vec{x})$ si ottiene da $i_M(\vec{x}, t), s_M(\vec{x}, t)$ e $d_M(\vec{x},
                  t)$ dove $t$ è il minimo naturale per cui $q_M(\vec{x}, t)=n$.
    \end{itemize}


    Si intravede in questa definizione il meccanismo di minimalizzazione applicato a
    funzioni ricorsive (quali $q_M, i_M, s_M$ e $d_M$ ). Se ne deduce che $f$ è, a
    sua volta, parziale ricorsiva, proprio come volevamo dimostrare.

\end{proof}

Il teorema ci aiuta a caratterizzare in termini di funzioni ricorsive anche la
semidecidibilità. Infatti con riferimento al teorema 3.5.1 possiamo considerare
linguaggi $L \subseteq \mathbb{N}$ tali che $L=\emptyset$ oppure $L$ è immagine
di qualche funzione totale ricorsiva 1aria $f$. Chiamiamo ricorsivamente
enumerabile un tale linguaggio $L$ e deduciamo banalmente:

\paragraph{Corollario 4.4.3} \textit{$L \subseteq \mathbb{N}$ è ricorsivamente enumerabile se
    e solo se è semidecidibile.}\\

I linguaggi ricorsivamente enumerabili si riconoscono allora anche come i domini
delle funzioni parziali ricorsive 1-arie (sempre per il teorema 3.5.1).
