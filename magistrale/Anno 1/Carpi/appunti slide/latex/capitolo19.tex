\chapter{La somma di sottoinsiemi è NP-completa} \label{ch:capitolo19}
\section{La somma di sottoinsiemi è NP-completa}
Il problema della somma di sottoinsiemi è il seguente: dati n numeri interi non negativi 
\begin{center}
    $w1,..., wn$
\end{center}
e una somma obiettivo W, si tratta di decidere se esiste un sottoinsieme $I \subset \{1,...,n\}$ tale che $\sum_{i\in I}$ $w_i = W$. Si tratta di un caso molto particolare del problema di \textbf{Knapsack}.
\\Nel problema di Knapsack, anche gli oggetti hanno valori $v_i$, e il problema consiste nel massimizzare $\sum_{i\in I}$ $v_ii \in I vi$ soggetto a $\sum_{i\in I}$ $w_i = W$. 
\\Se impostiamo $v_i = w_i$ per tutti gli i, la Somma di sottoinsiemi è un caso speciale del problema di Knapsack che abbiamo discusso quando abbiamo considerato la programmazione dinamica. In quella sezione, abbiamo fornito un algoritmo per il problema che viene eseguito in tempo $O(n*W)$.
\\Questo algoritmo funziona bene quando W non è troppo grande, ma notiamo che questo algoritmo non è un algoritmo a tempo polinomiale. Per scrivere un intero W, abbiamo bisogno solo di log W cifre. È naturale assumere che tutti i $w_i <= W$, e quindi la lunghezza dell'ingresso è $(n + 1) \log W$, e il tempo di esecuzione di $O(n*W)$ non è polinomiale per questa lunghezza di input.
\\In questa dispensa mostriamo che, in effetti, la somma di sottoinsiemi è NP-completa. Innanzitutto dimostriamo che
La somma di sottoinsiemi è in NP.\\\\
\textbf{Proposizione 1}
\\La somma di sottoinsiemi è in NP.\\\\
\textbf{Dimostrazione}\\ 
Dato un insieme proposto I, basta verificare se effettivamente  $\sum_{i\in I}$ $w_i = W$. Sommare al massimo n numeri, ciascuno di dimensione W, richiede un tempo $O(n \log W)$, lineare rispetto alla dimensione dell'input. Per stabilire che la somma di insiemi è NP-completa, dimostreremo che è almeno altrettanto difficile di SAT.
\\\\\textbf{Teorema}
\begin{center}
    SAT <= Somma di sottoinsiemi.
\end{center}
\textbf{Dimostrazione}\\
Per dimostrare l'affermazione è necessario considerare una formula $\phi$, un input di SAT, e trasformarla in un input equivalente alla somma dei sotto-insieme. 
\\Supponiamo che $\phi$ abbia n variabili $x_1,..., x_n$, e m clausole $c_1,...,c_m,$ dove la clausola $c_j$ ha $k_j$ letterali.
\\Definiremo il nostro problema della somma di sottoinsiemi utilizzando una base B molto grande, quindi scriveremo i numeri come $\sum_{j=0}^n+m a_j B^j$ e impostiamo la base B come $B = 2 max_j k_j$ che assicura che le addizioni tra i numeri tra i nostri numeri non causeranno mai un riporto.
\\Scritte in base B, le cifre $i = 1, . . . , n$ corrisponderanno alle n variabili $x_1,... x:n$, e l'obiettivo di queste cifre sarà quello di assicurarsi di impostare ogni variabile su vero o falso (e non su entrambi). Avremo due numeri $w_i$ e $w_{i+n}$ che corrispondono alla variabile $x_i$ che viene impostata vera o falsa, e la cifra i farà in modo di utilizzare uno dei due numeri $w_{i+n}$ in qualsiasi soluzione. Per fare ciò, impostiamo la decima cifra di W, $w_i$ e $w_{i+n}$ come 1 e impostiamo questa cifra in tutti gli altri numeri come 0. Le successive m cifre corrisponderanno alla variabile $x_i$.
Le successive m cifre corrispondono alle m clausole e la cifra obiettivo n + j serve a garantire che la j-esima clausola sia soddisfatta dalla nostra impostazione delle variabili.
\\Il valore obiettivo sarà $ W= $ $\sum_{j=1}^n B^i + \sum_{j=1}^m k_j B^{n+j}$.
\\Si inizia definendo 2n numeri, per ciascuno dei letterali $x_i$ e $x\Bar{i}$. Le cifre 1,...,n faranno in modo che ogni sottoinsieme che somma a W utilizzi esattamente solo uno dei due numeri $x_i$ e $x\Bar{i}$, mentre le successive m cifre avranno lo scopo di garantire che ogni clausola sia soddisfatta. Avremo bisogno di alcuni numeri aggiuntivi che definiremo in seguito.
\\Il numero corrispondente al letterale $x_i$ è il seguente $wi = B^i + \sum_{j:x_i \in c_j} B^{n+j}$, mentre il numero corrispondente al letterale $x\Bar{i}$ è il seguente
corrispondente al letterale $w_i = B^i + \sum_{j:x_i \in c_j} B^{n+j}$. Se aggiungiamo un insieme di n numeri corrispondente a un'assegnazione di verità soddisfacente per $\phi$, si ottiene una qualche forma di $\sum_{i=1}^n b_j B^{i} + \sum_{j=1}^m b_j B^{n+j} $ dove $b_j$ è il numero di letterali veri nella clausola $c_j$. Poiché si trattava di un'assegnazione soddisfacente, dobbiamo avere $b_j >= 1$.
\\Come ultimo dettaglio, aggiungeremo $k_j - 1$ copie del numero $B^{n+j}$ per tutte le clausole $c_j$ . In questo modo definito il nostro problema della somma dei sottoinsiemi, con l'obiettivo W e i $2n + \sum_{j} (k+j-1) $ numeri definiti, l'aggiunta di questi numeri aggiuntivi ci permetterà di raggiungere esattamente W.
Per dimostrare che si tratta di una riduzione valida, dobbiamo stabilire due affermazioni di seguito riportate che stabiliscono rispettivamente la direzione del se e del solo se della prova.\\\\
\textbf{Proposizione 2}\\
Se il problema SAT definito dalla formula $\phi$ è risolvibile, allora anche il problema della Somma di sottoinsiemi appena definito con $2n - m + \sum_{j} k_j$ numeri è anch'esso risolvibile.
\\\\\textbf{Dimostrazione}
\\Supponiamo di avere un'assegnazione soddisfacente per la formula $\phi$, prima consideriamo di aggiungere i numeri che corrispondono $a_i$ letterali veri. Abbiamo usato esattamente uno dei numeri $w_i$ e $w_{n+i}$, quindi avremo 1 nella cifra i-esima e otterremo una somma della forma $\sum_{i=1}^n B_i + \sum_{j=1}^m a_j B^{n+j}$.
\\Inoltre, si avrà $1 <= a_i <= k_i$, dove $a_i$ è almeno 1, in quanto l'assegnazione ha soddisfatto la formula, quindi almeno uno dei numeri aggiunti ha un 1 nella (n+j) terza cifra, e al massimo $k_j$ in quanto anche sommando tutti i numeri al massimo $k_j$ di essi ha un 1 nella (n+j)terza cifra. In particolare, con $B > k_j$, non ci saranno riporti.
\\Per fare in modo che la somma sia esattamente W, aggiungiamo $k_j - a_j$ copie del numero $B^{n+j}$ che abbiamo aggiunto alla fine della costruzione.
\\Dobbiamo poi dimostrare l'altra direzione:
\\\\\textbf{Proposizione 3}
\\ Se il problema della somma di sottoinsiemi che abbiamo appena definito con $2n - m + \sum_{j} k_j$ numeri è risolvibile, allora la Somma di sottoinsiemi definita dalla formula $\phi$ è risolvibile.\\\\
\textbf{Dimostrazione}\\
Si noti innanzitutto che per qualsiasi sottoinsieme si possa aggiungere, non ci sarà mai un riporto in nessuna cifra. 
\\Per capire perché, si noti che tutti i numeri da sommare hanno tutte le cifre 0 o 1; per la cifra $i = 1,..., n$ abbiamo due numeri con un 1 in quella cifra $w_i e w_{i+n}$; la cifra 0 è sempre 0; e per la cifra n + j abbiamo esattamente $2 max_j$ $k_j - 1$ numeri che hanno un 1 in quella cifra: $k_j$ corrispondenti ai $k_j$ letterali nella clausola e $k_j - 1$ numeri extra $B^{n+j}$ che abbiamo aggiunto alla fine. 
\\Quindi, anche se anche se aggiungiamo tutti i numeri, non possiamo causare un riporto in nessuna delle cifre!.
\\In base all'osservazione di cui sopra sull'assenza di riporti, per ottenere il numero W dobbiamo trovare un sottoinsieme I che abbia esattamente il numero giusto di 1 in ogni cifra. Per prima cosa concentriamoci sulle cifre 1,..., n. Questa cifra in W è un 1 e i due numeri che hanno un 1 in questa cifra sono $w_i$ e $w_{i+n}$, per sommare a W, dobbiamo usare esattamente uno di questi, sia I' $\subset$ I corrispondente ai letterali. Questo Questo dimostra che i numeri selezionati tra i primi 2n corrispondono a un'assegnazione di verità delle variabili $x_1, . . . , x_n$.
\\Infine, dobbiamo dimostrare che questa assegnazione di verità soddisfa la formula $\phi$. Consideriamo la somma $W' = \sum_{j \in I'} wj$ aggiungendo solo il sottoinsieme I che corrisponde alle variabili. Si noti che $W' = \sum_{j=1}^n B_i + \sum_{j=1}^m a'_j B^{n+j>}$ con $a'_j j <= k_j$. Dobbiamo dimostrare che $a'_j >= 1$, il che dimostrerà che abbiamo un'assegnazione soddisfacente. Ricordiamo che il sottoinsieme I somma esattamente a W. Per essere in grado di estendere I' con un sottoinsieme di numeri aggiuntivi che sommano a W, dobbiamo avere $a'_j$, poiché ci sono ci sono solo $k_j - 1$ copie di $B^{n+j}$ .