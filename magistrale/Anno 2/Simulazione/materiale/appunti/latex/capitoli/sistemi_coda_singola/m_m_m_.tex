
\section{M/M/1} \label{mm1}

Sistema aperto denotato da un singolo servente:

\begin{itemize}
    \item Distribuzione del Tempo di Inter-arrivo Esponenziale con parametro
          $\lambda$
    \item Tempo di Servizio Esponenziale di parametro $\mu$
\end{itemize}

\subsection{Parametri}

\begin{itemize}
    \item Numero di Utenti Medio: $N = \frac{\rho}{1 - \rho} = \lambda R$
    \item Numero Medio di Utenti in Coda: $W = N - \rho = \frac{\rho^2}{1-\rho}$
    \item Tempo Medio di Risposta: $R = \frac{\frac{1}{\mu}}{1 - \rho} = T_s +
              T_w = \frac{N}{\lambda}$
    \item Tempo di Attesa Medio in Coda: $T_w = \frac{\frac{\rho}{\mu}}{1 -
                  \rho} = R - T_s$
    \item Probabilità di Osservare almeno $k$ utenti in un Sistema in condizione
          di Stazionarietà: $ = \rho^k$
    \item Probabilità di avere $0$ utenti nel sistema: $\pi_0 = 1 - \rho$
    \item Probabilità di avere $k$ utenti nel sistema: $\pi_k = \rho^k \pi_0 =
              \rho^k (1 - \rho)$
\end{itemize}

\section{M/M/m}

Sistema aperto dotato di $m$ serventi:

\begin{itemize}
    \item Distribuzione del Tempo di Arrivo Poissoniano con parametro $\lambda$
    \item Distribuzione del Tempo di Servizio Esponenziale con parametro $\mu$
\end{itemize}

\subsection{Parametri}

\begin{itemize}
    \item Numero di Servienti: $m$
    \item Tempo Medio di Servizio: $T_s$ (vedi \ref{parametri-base})
    \item Tasso Medio di Arrivi $\lambda$ (vedi \ref{parametri-base})
    \item Tempo Medio di Inter-Arrivo: $\mu$ (vedi \ref{parametri-base})
    \item Intensità del Traffico: $\rho = \frac{\lambda}{m \mu}$
    \item Probabilità di avere $0$ utenti nel sistema:
          $$\pi_0 = \left [ \sum_{k=0}^{m-1} \left ( \frac{(m \rho)^k}{k!}
                  \right ) + \frac{(m \rho)^m}{m!} \frac{1}{1-\rho} \right ]^{-1}$$
    \item Probabilità di avere $k$ utenti nel sistema:
          \begin{itemize}
              \item[\emoji{orangutan}] se $1 \leq k \leq m$ $$\pi_k = \frac{(m
                          \rho)^k}{k!} \pi_0$$
              \item[\emoji{gorilla}] se $k > m$ $$\pi_k = \frac{m^m \rho^k}{m!}
                      \pi_0$$
          \end{itemize}
    \item Numero Medio di Serventi Occupati: $$E[s] = \sum_{k=0}^{m-1} \left (
              k\pi_k \right ) + \frac{m\pi_m}{1-\rho}  = m \rho = \frac{\lambda}{\mu}$$
    \item Numero di Utenti Medio: $N = m \rho + \pi_m \frac{\rho}{(1-\rho)^2}$
    \item Numero di Utenti Medio in Coda: $W = \pi_m \frac{\rho}{(1-\rho)^2}$
    \item Tempo Medio di Risposta: $R = \frac{N}{\lambda} = \frac{m \rho +
                  W}{\lambda}$
    \item Tempo di Attesa in Coda: $$T_w = \frac{\pi_m}{m\mu (1-\rho)^2}$$
    \item Tempo di Utilizzo (tempo in cui si sta bene a tenti): $U = 1 - \pi_0 =
              \rho$
    \item Tempo di Non Utilizzo: $\hat{U} = 1 - U$
    \item Probabilità che un Utente in Arrivo trovi tutti i serventi occupati:
          $$Prob_{coda} = \sum_{k=m}^{+\infty} \pi_k = \pi_0 \frac{(m\rho)^m}{m!}
              \frac{1}{1-\rho}$$
    \item Probabilità che un Utente in Arrivo Non trovi una coda:
          $$\hat{Prob_{coda}} = 1 - Prob_{coda}$$
\end{itemize}

\section{M/M/\texorpdfstring{$\infty$}{infinito}}

Sistema aperto con infiniti servienti:

\begin{itemize}
    \item Distribuzione del Tempo di Arrivo Poissoniano di parametro $\lambda$
    \item Distribuzione del Tempo di Servizio Esponenziale di parametro $\mu$
\end{itemize}

\subsection{Parametri}

\begin{itemize}
    \item Intensità del Traffico: $\rho$ (vedi \ref{parametri-base})
    \item Probabilità di avere $k$ utenti, che coincide (in questo caso
          specifico) con la Probabilità di avere $k$ serventi occupati: $$\pi_k =
              \frac{\rho^k}{k!} e^{-\rho}$$ con $k \geq 0$
    \item Numero Medio di Utenti: $N = \rho$
    \item Tempo Medio di Risposta, che coincide con il Tempo Medio di Servizio:
          $$R = T_s = \frac{1}{\mu}$$
\end{itemize}

\section{M/M/1/K (dimensione coda finita)}

Sistema $M/M/1$ dove sono ammessi al più $K$ utenti (coda finita):

\begin{itemize}
    \item Distribuzione del Tempo di Inter-arrivo vedi \ref{mm1}
    \item Tempo di Servizio vedi \ref{mm1}
\end{itemize}

\paragraph{Esempio:}
Il processo di arrivo è Poissoniano di parametro λ, ma un utente che arrivando
trova il sistema completo, cioè con K utenti già presenti, non viene accettato e
viene perso. Il sistema M/M/1/K è un sistema con perdita. Nel caso particolare
in cui K=1, al più un utente è ammesso nel sistema e di conseguenza non si forma
mai coda. La distribuzione del tempo di servizio è esponenziale di parametro µ,
vi è un singolo servente, e la disciplina di coda è FIFO.

\subsection{Parametri}

\begin{itemize}
    \item Per quelli base vedere \ref{parametri-base} e \ref{mm1}
    \item Probabilità di avere k utenti nel sistema:
          $$\pi_k = \frac{1-\rho}{1-\rho^{K+1}} \rho^K \ \ \ \ \text{con } \  \ 0 \leq k \leq K$$
          $$\pi_k = 0 \ \ \ \ \text{con} \ \ k > K$$
          dove $\rho = \frac{\lambda}{\mu}$\\

          Nel caso particolare di $K = 1$ lo spazio è formato da 2 soli stati e si ricava:
          $$\pi_0 = \frac{\mu}{\lambda + \mu}$$
          $$\pi_1 = \frac{\lambda}{\lambda + \mu}$$
    \item Utilizzazione CPU: $U = 1 - \pi_0 = \rho$
    \item Frequenza media del completamento delle richieste (throughput): $$X = \mu (1-\pi_0) = \mu U = \frac{U}{E(T_s)}$$
    \item Il tempo che ciascun utente impiega per la prossima richiesta: $R + \frac{1}{\lambda}$ secondi
    \item Frequenza media di generazione richieste: $$\frac{M}{R + \frac{1}{\lambda}}$$
    \item In stato stazionario, le frequenze di generazione e il tempo di completamento delle richieste devono essere uguali:
          $$X = \mu (1-\pi_0) = \frac{M}{R + \frac{1}{\lambda}}$$
          $$R = \frac{M}{\mu (1-\pi_0)} - \frac{1}{\lambda} = \frac{M}{X} - \frac{1}{\lambda} = \frac{\text{Numero Clienti}}{\text{Throughput Medio}}$$
\end{itemize}

\section{M/M/1//M (dimensione popolazione finita)}

Sistema $M/M/1$ dove la popolazione è finita di dimensione $M > 0$:

\begin{itemize}
    \item Distribuzione del Tempo di Inter-arrivo vedi \ref{mm1}
    \item Tempo di Servizio vedi \ref{mm1}
\end{itemize}


\paragraph{Esempio:}
Consideriamo il sistema $M/M/1$ assumendo che gli utenti provengano da una
popolazione finita, di dimensione $M>0$. Ogni utente si trova ad ogni istante o
all'interno del sistema (in coda o in servizio) o all'esterno. Assumiamo che
ogni utente, una volta che ha lasciato il sistema dopo essere stato servito, si
ripresenti al sistema stesso dopo un tempo esponenziale di parametro λ. Inoltre
assumiamo che ogni utente sia indipendente dagli altri. Questo comporta che, se
vi sono k utenti nel sistema ($0 \leq k \leq M$) ed $M-k$ all'esterno, il processo di
arrivo totale è dato dalla composizione di $M-k$ processi di Poisson indipendenti,
ognuno di parametro $\lambda$. Per la proprietà di composizione dei processi di Poisson,
anche il processo totale di arrivo al sistema è ancora un processo di Poisson di
parametro dipendente dallo stato $\lambda(k) = (M-k) \lambda, \  (0 \leq k \leq M)$.

\subsection{Parametri}

\begin{itemize}
    \item Per quelli base vedere \ref{parametri-base} e \ref{mm1}
    \item La condizione di stazionarietà è certamente verificata, poiché il processo è finito e irriducibile e la distribuzione
          stazionaria del numero di utenti nel sistema, dalle formule:
          $$\pi_k = \pi_0 \rho^k \frac{M!}{(M-k)!} \ \ \ \ \text{con } \  \ 0 \leq k \leq M$$
          con
          $$\pi_0 = \left [ \sum_{k=0}^{M} \rho^k \frac{M!}{(M-k)!} \right ]^{-1} \ \ \ \ \text{dove } \ \ \rho = \frac{\lambda}{\mu}$$

          $$\pi_k = 0 \ \ \ \ \text{con} \ \ k > M$$
\end{itemize}


\section{M/G/1} \label{mg1}

Sistema aperto con un singolo servente:

\begin{itemize}
    \item Distribuzione del Tempo di Inter-Arrivo Esponenziale con parametro
          $\lambda$
    \item Distribuzione del Tempo di Servizio degli Utenti Indipendente con
          Distribuzione Generale
\end{itemize}

\subsection{Parametri}

\begin{itemize}
    \item Per quelli di base vedere \ref{parametri-base}
    \item Numero Medio di Utenti (formula di \textit{Khintchine-Pollaczk} \emoji{man-with-veil-dark-skin-tone}): $$N
              = \rho + \frac{\rho^2 (1 + C^2_B)}{2 (1-\rho)}$$ dove :
          \begin{itemize}
              \item $C_B = \sigma \mu$ (Coefficiente di Variazione)
              \item $\sigma = \sqrt{\text{Varianza}}$ (Deviazione Standard)
          \end{itemize}
    \item Tempo Medio di Risposta di un lavoro: $R = \frac{N}{\lambda}$
    \item Tempo Medio di Attesa in Coda: $W = \lambda T_w = N - \rho$
    \item Tempo di Attesa in Coda: $T_w = \frac{N -\rho}{\lambda}$
\end{itemize}

\paragraph{N.B.}
\begin{itemize}
    \item Se $\rho = 1$ e quindi il sistema è \textbf{congestionato}, allora gli
          indici medi $N, W, R, T_w$ tendono a crescere senza limite.
\end{itemize}

\section{M/D/1}

Versione di $M/G/1$ con Distribuzione del Tempo di Servizio \textit{Deterministico}:

\begin{itemize}
    \item Distribuzione del Tempo di Inter-Arrivo Esponenziale con parametro
          $\lambda$
    \item Distribuzione del Tempo di Servizio degli Utenti Indipendente con
          Distribuzione Deterministica
\end{itemize}
\subsection{Parametri}

\begin{itemize}
    \item Valore Medio degli Utenti nel Sistema: $$N = \rho + \frac{\rho^2}{2 (1-\rho)}$$
    \item Numero di Utenti Medio in Attesa: $$W = \frac{\rho^2}{2(1-\rho)}$$
\end{itemize}

\paragraph{N.B.}
\begin{itemize}
    \item Tutti i parametri che non sono stati elencati sono calcolati come scritto in \ref{mg1}
    \item Se $\rho = 1$ e quindi il sistema è \textbf{congestionato}, allora gli
          indici medi $N, W, R, T_w$ tendono a crescere senza limite.
\end{itemize}