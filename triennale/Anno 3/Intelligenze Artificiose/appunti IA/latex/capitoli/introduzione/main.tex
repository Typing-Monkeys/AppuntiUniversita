\chapter{Introduzione}

\section*{Definizione di intelligenza:}

\begin{table}[h]
\centering
\begin{tabular}{lcc}
 & \textbf{Pensare} & \textbf{Comportarsi} \\
\hline
\textbf{Umanamente} & 1 & 2 \\
\textbf{Razionalmente} & 3 & 4 \\
\end{tabular}
\end{table}

\subsection*{1. Pensare umanamente: modello cognitivo}

Vogliamo capire se il programma pensa come un umano. Vengono comparate le sequenze e le tempistiche degli step di ragionamento del programma con quelli umani.

\subsection*{2. Comportarsi umanamente: test di Turing}

Viene testato che il programma possa comportarsi come un umano.
Per fare ciò ha bisogno di:

\begin{itemize}
  \item Natural language processing, per comunicare in un linguaggio umano.
  \item Rappresentazione dell'informazione, per mantenere quello che conosce.
  \item Ragionamento automatico, per rispondere a domande e per arrivare a nuove conclusioni.
  \item Machine learning, per adattarsi a nuovi scenari.
\end{itemize}

\subsection*{3. Pensare razionalmente: le leggi del pensiero}

La logica studia le leggi del ragionamento e della dimostrazione.
Si tenta di costruire un programma che possa risolvere un problema utilizzando il ragionamento logico.
Visto che le informazioni che abbiamo su molti fenomeni sono parziali (non conosciamo le leggi della politica), abbiamo bisogno della teoria della probabilit\`a per arrivare a conclusioni vere.

\subsection*{4. Agire razionalmente: l'approccio dell'agente razionale}

Un agente \`e qualcosa che agisce; dai computer agents ci si aspetta che operino autonomamente, percepiscano il loro ambiente, si adattino al cambiamento.
Un \textbf{agente razionale} agisce per ottenere il migliore risultato o il risultato atteso.

Questo approccio ha due vantaggi:
\begin{enumerate}
  \item \`E pi\`u generale del punto 3, perch\'e l'inferenza \`e solo uno dei possibili modi per ottenere la razionalit\`a.
  \item \`E pi\`u aperto allo sviluppo scientifico rispetto ai punti 1 e 2, perch\'e la razionalit\`a \`e verificabile.
\end{enumerate}

Il campo dell'AI si \`e concentrato sullo studio e sullo sviluppo di agenti che facciano la cosa giusta.

Il modello dell'agente razionale assume che venga fornito alla macchina un obiettivo completamente specificato mentre molto spesso quello che si deve fare \`e bilanciare l'obiettivo e i suoi possibili effetti collaterali.

Il problema \`e detto: \textbf{problema dell'allineamento dei valori}. I valori o gli obiettivi forniti alla macchina devono essere allineati a quelli degli umani.

